\documentclass[a4paper,12pt]{report}
\usepackage[utf8]{inputenc}
\usepackage{geometry}
\usepackage{setspace}
\usepackage{titlesec}
\usepackage{tocloft}
\usepackage{amsmath}
\usepackage{amssymb}
\usepackage{amsthm}
\usepackage{tikz}
\usepackage{tikz-cd}

\newtheorem{definition}{Definition}[chapter]
\newtheorem{theorem}{Theorem}[chapter]
\newtheorem{lemma}{Lemma}[chapter]
\newtheorem{corollary}{Corollary}[chapter]

% Adjust margins to look like the typewriter original
\geometry{top=1in, bottom=1in, left=1.5in, right=1in}

% Double spacing for the typewriter feel
\doublespacing

\begin{document}

% ==========================================
% TITLE PAGE
% ==========================================
\begin{titlepage}
  \begin{center}
    \vspace*{2cm}

    \textbf{GENERALISED ALGEBRAIC THEORIES}

    \vspace{0.5cm}

    \textbf{AND}

    \vspace{0.5cm}

    \textbf{CONTEXTUAL CATEGORIES}

    \vspace{3cm}

    by

    \vspace{0.5cm}

    J.W. Cartmell

    \vfill

    Submitted for the degree of D.Phil.

    Oxford University

    1978

  \end{center}
\end{titlepage}

% ==========================================
% ACKNOWLEDGEMENT
% ==========================================
\newpage
\thispagestyle{empty}
\begin{center}
  \textbf{Acknowledgement}
\end{center}

\vspace{1cm}

\noindent I am indebted to Professor Dana Scott for supervision, criticism and direction.

% ==========================================
% CONTENTS
% ==========================================
\newpage
\thispagestyle{empty}
\begin{center}
  \textbf{CONTENTS}
\end{center}

\vspace{1cm}

\noindent \textbf{PREFACE} \hfill \textbf{PAGE NO.}

\vspace{0.5cm}

\noindent \textbf{CHAPTER 1 GENERALISED ALGEBRAIC THEORIES} \hfill \textbf{1.1}

\vspace{0.3cm}
\noindent 1.1 \quad Introduction \hfill 1.2 \\
\noindent 1.2 \quad Examples of theories \hfill 1.7 \\
\noindent 1.3 \quad Predicates as types \hfill 1.13 \\
\noindent 1.4 \quad Essentially algebraic theories and categories \\
\hspace*{2.5em} with finite limits \hfill 1.16 \\
\noindent 1.5 \quad The extra generality of the algebraic semantics \hfill 1.19 \\
\noindent 1.6 \quad The formal definition \hfill 1.22 \\
\noindent 1.7 \quad The substitution lemma and other lemmas \hfill 1.28 \\
\noindent 1.8 \quad Informal syntax \hfill 1.39 \\
\noindent 1.9 \quad Models and homomorphisms \hfill 1.45 \\
\noindent 1.10 \quad A list of theories \hfill 1.49 \\
\noindent 1.11 \quad Interpretations \hfill 1.50 \\
\noindent 1.12 \quad Contexts and realisations \hfill 1.58 \\
\noindent 1.13 \quad Intended identity of denotation \hfill 1.60 \\
\noindent 1.14 \quad The Category GAT \hfill 1.69

\vspace{0.5cm}

\noindent \textbf{CHAPTER 2 CONTEXTUAL CATEGORIES}

\vspace{0.3cm}
\noindent 2.1 \quad Algebraic semantics \hfill 2.1 \\
\noindent 2.2 \quad Definition and examples \hfill 2.3 \\
\noindent 2.3 \quad Notation and basic lemmas \hfill 2.12 \\
\noindent 2.4 \quad Contextual categories = generalised \\
\hspace*{2.5em} algebraic theories \hfill 2.24 \\
\noindent 2.5 \quad Functorial Semantics, Universal algebra \hfill 2.77

\newpage
\thispagestyle{empty}
\vspace*{1cm}

\noindent \textbf{CHAPTER 3 THE ALGEBRAIC SEMANTICS OF MARTIN-LOF \\ TYPE THEORY} \hfill \textbf{3.1}

\vspace{0.3cm}
\noindent 3.1 \quad Disjoint unions and singleton object \hfill 3.2 \\
\noindent 3.2 \quad Categories with attributes \hfill 3.14 \\
\noindent 3.3 \quad Categories and fibrations \hfill 3.32 \\
\noindent 3.4 \quad Martin-Lof type theory \hfill 3.35 \\
\noindent 3.5 \quad Limit spaces -- a model of M-L type theory \hfill 3.50

% ==========================================
% PREFACE
% ==========================================
\newpage
\pagenumbering{roman}
\setcounter{page}{1}

\begin{center}
  \underline{PREFACE}
\end{center}

\vspace{0.5cm}

A notion of equational theory is introduced; more general than previous notions, equal in descriptive power to the essentially algebraic theories of Freyd [5], and hence to the logic of left exact categories, we call the theories generalised algebraic. The extra generality of these equational theories is achieved by the introduction of sort structures more general than those usually considered in that sorts may denote sets as is usual or else they may denote families of sets, families of families of sets and the like. This acceptance of variable types at the level of syntax (the idea and the form of syntax is taken directly from Martin-Lof type theory) makes the theories particularly suited to the description of the structures that occur in category theory. The basic example being the theory of categories, in which Ob appears as a sort to be interpreted as a set where as Hom appears as a sort to be interpreted as a family of sets indexed by $Ob \times Ob$. Hom$(x,y)$ appears in the syntax as a variable type.

The definition of the most general or algebraic semantics for generalised algebraic theories necessitates the introduction of the notion of a contextual category. So called because we shall see that the objects of a contextual category should be thought of as contexts.

The theory of contextual categories is seen as an algebraic description of the structure imposed on certain classes of term and type expressions by the operation of substitution of correctly typed terms for variables. Now this is something one might also say of the theory of categories. However the theory of contextual categories captures the structure of substitution at work in a more general situation, it is the structure of substitution as found in the generalised algebraic theories but not in algebraic theories, as found originally in Martin-Lof type theory but not in theories of the typed $\lambda$-calculus.

It is proved that the category of contextual categories is equivalent to the category of generalised algebraic theories and equivalence classes of interpretations. Thus we say that we have the most general possible semantics. This result is a generalisation of the result implicit in Lawvere [11] that the old syntactic notion of algebraic theory (i.e. one sorted equational) and Lawvere's algebraic notion are both one and the same (i.e. equivalent categories).

This thesis developed from the desire to develop the model theory of Martin-Lof type theory. The model theory rests on the notions of generalised algebraic theory and contextual category. It is only in these terms that we can define the notion model of Martin-Lof type theory. We also give the definition of model for a strengthened version of Martin-Lof type theory, this definition can be reworked algebraically into a hyperdoctrinal format. We briefly describe a new model of the type theory in which types are interpreted as limit spaces.

The model theory of the strengthened version of Martin-Lof type theory is a generalisation of the well known correspondence of the typed $\lambda$-calculus with cartesian closed categories.

\chapter{GENERALISED ALGEBRAIC THEORIES}

\section{Introduction}

The purpose of this chapter is to describe and to formally define the notion of generalised algebraic theory. It is hoped that it will be clear from the description that (i) the notion is a natural one formalising actual mathematical language and that (ii) the notion is a simple generalisation of the notion of a many sorted algebraic theory. Though (ii) tends to be obscured by the form of the chosen syntax no doubt the choice is correct.

The formal definition is given in \S 1.6. Most of the material that follows \S 1.6 is in preparation for Chapter Two. \S 1.8 is partially in digression and partially to explain some of the informal syntax that is used in the early sections of this Chapter.

The notion of generalised algebraic theory is a generalisation of the notion of many sorted algebraic theory in just the following manner. Whereas the sorts of a many sorted algebraic theory are constant types in the sense that they are to be interpreted as sets, the sorts of a generalised algebraic theory need not all be constant types; some of them may be nominated to be variable types in which case they are to be interpreted as families of sets. The type or types on which the variation of a variable type depends must always be specified.

Thus a generalised algebraic theory consists of (i) a set of sorts, each with a specified role either as a constant type or else as a variable type varying in some way, (ii) a set of operator symbols, each with its argument types and its value type specified (the value type may vary as the argument varies), (iii) a set of axioms. Each axiom must be an identity between similar well formed expressions, either between terms of the same possible varying type or else between type expressions.

The theory of categories is a good example. The sort symbols we shall call $Ob$ and $Hom$, the operator symbols $id$ and $o$.

$Ob$ is a constant type. $Hom$ is a symbol for a variable type depending twice on $Ob$. That is to say that if $t_1$ and $t_2$ are both terms of type $Ob$ then $Hom(t_1, t_2)$ is a type. In particular if $x$ and $y$ are both variables of type $Ob$ then $Hom(x,y)$ is a type.

The operator symbol $id$ has one argument type, namely $Ob$. The value type of $id$ varies as the argument varies, for if $x$ is a variable of type $Ob$ then $id(x)$ is of type $Hom(x,x)$.

Not all the argument types of $o$ are constant. If $x,y$ and $z$ are variables of type $Ob$, if $f$ is a variable of type $Hom(x,y)$ and if $g$ is a variable of type $Hom(y,z)$, then $o(f,g)$ is a term of type $Hom(x,z)$.

One way of setting up the syntax to deal with variables would be to assume that for every type $\Delta$ we had a supply $V_{\Delta}$ of variables of type $\Delta$. However this method would lead to complications. Instead we assume just one set $V$ of variables and then repeatedly assign types to variables as required. In a particular context the assertion or assumption that the variable $x$ is of type $\Delta$ is written shorthand as $x \in \Delta$. More generally, the assertion that an expression $t$ is a term of type $\Delta$ will be written as $t \in \Delta$. If the term $t$ has variables $x_1, \dots, x_n$ occurring within it then it will only make sense to assert $t \in \Delta$ under an assumption that $x_1, \dots, x_n$ are variables of particular types. The complete assertion will be of the form: if $x_1$ is a variable of type $\Delta_1, \dots$ and if $x_n$ is a variable of type $\Delta_n$ then $t$ is a term of type $\Delta$. This complete assertion we write shorthand as
\[ x_1 \in \Delta_1, x_2 \in \Delta_2, \dots, x_n \in \Delta_n \]
\[ \rule{4cm}{0.4pt} \]
\[ t \in \Delta \]
or else as $x_1 \in \Delta_1, x_2 \in \Delta_2, \dots, x_n \in \Delta_n : t \in \Delta$.

Similarly $x_1 \in \Delta_1, \dots, x_n \in \Delta_n$ is used to assert that if $x_1$ is a variable of type $\Delta_1, \dots$ if $x_n$ is a variable of type $\Delta_n$, then $\Delta$ is a type.
\[ \rule{4cm}{0.4pt} \]
\[ \Delta \text{ is a type} \]

These shorthands of the forms
\[ \frac{x_1 \in \Delta_1, \dots, x_n \in \Delta_n}{t \in \Delta} \quad \text{and} \quad \frac{x_1 \in \Delta_1, \dots, x_n \in \Delta_n}{\Delta \text{ is a type}} \]
we call rules. They serve to express which expressions of a given language are well formed as terms or as types. We work with these rules as units rather than with the basic expressions. For example, in the formal definition instead of defining the notions of well formed term and well formed type we define inductively a set of rules, to be called the derivable rules, which express the well formed types, the well formed terms and their types.

The axioms of a theory are also written as rules. Instead of the more usual $\forall x_1 \in \Delta_1 \forall x_2 \in \Delta_2 \dots \forall x_n \in \Delta_n, t_1 = t_2$ we write
\[ \frac{x_1 \in \Delta_1, x_2 \in \Delta_2, \dots, x_n \in \Delta_n}{t_1 = t_2} \]
There again, we might just write $t_1 = t_2$, whenever $x_1 \in \Delta_1, \dots, x_n \in \Delta_n$.

For example the theory of categories has as axioms the following:
\begin{itemize}
  \item $o(id(x), f) = f$, whenever $x,y \in Ob$ and $f \in Hom(x,y)$.
  \item $o(f, id(y)) = f$, whenever $x,y \in Ob$ and $f \in Hom(x,y)$.
  \item $o(o(f,g), h) = o(f, o(g,h))$, whenever $w,x,y,z \in Ob$, $f \in Hom(w,x)$, $g \in Hom(x,y)$ and $h \in Hom(y,z)$.
\end{itemize}

A theory is presented by specifying the language and by listing the axioms. The language is specified by listing the symbols and by specifying the role that each symbol plays within the language either as a sort symbol of some kind or as a particularly typed operator symbol. The role that a symbol plays can always be specified by way of the assertion of a single rule. In the case of a sort symbol $A$ there is a rule of the form
\[ \frac{x_1 \in \Delta_1, \dots, x_n \in \Delta_n}{A(x_1, \dots, x_n) \text{ is a type}} \]
that will correctly specify over what types $A$ is dependent. In the case of an operator symbol $f$ a rule of the form
\[ \frac{x_1 \in \Delta_1, \dots, x_n \in \Delta_n}{f(x_1, \dots, x_n) \in \Delta} \]
suffices to specify of what types its arguments are to be and of what type its values will be. In either case we call the symbol the \textit{introductory rule} associated with the symbol.

For example the sort $Hom$ of the theory of categories has introductory rule $x \in Ob, y \in Ob : Hom(x,y)$ is a type. The symbol $id$ has introductory rule $x \in Ob : id(x) \in Hom(x,x)$.

Finally, then, every theory is presented as a set of symbols each with associated introductory rule and a set of axioms. And of course everything must be well formed, but we leave all that until we give the formal definition in \S 1.6.

The theory of categories now looks like this:

\textbf{Symbol} \quad \textbf{Introductory Rule}
\begin{itemize}
  \item $Ob$: \quad $Ob$ is a type.
  \item $Hom$: \quad $x,y \in Ob : Hom(x,y)$ is a type.
  \item $o$: \quad $x,y,z \in Ob, f \in Hom(x,y), g \in Hom(y,z) : o(f,g) \in Hom(x,z)$.
  \item $id$: \quad $x \in Ob : id(x) \in Hom(x,x)$.
\end{itemize}

\textbf{Axioms.}
\begin{itemize}
  \item $o(id(x), f) = f$, whenever $x,y \in Ob$ and $f \in Hom(x,y)$.
  \item $o(f, id(y)) = f$, whenever $x,y \in Ob$ and $f \in Hom(x,y)$.
  \item $o(o(f,g), h) = o(f, o(g,h))$, whenever $w,x,y,z \in Ob$, $f \in Hom(w,x)$, $g \in Hom(x,y)$ and $h \in Hom(y,z)$.
\end{itemize}

Whenever we speak of a model of a theory $U$, without qualification, then we shall mean a model in the usual sense, that is where type symbols are interpreted as sets, symbols for families of types are interpreted as families of sets, operator symbols are interpreted as operators and so on. Later we shall be interpreting theories in algebraic structures, in which case type symbols will be interpreted as objects within a structure rather than as sets.

\section{Examples of Theories}

The first example is a theory which can be called the theory of families of elements of families of sets:

\textbf{Symbol} \quad \textbf{Introductory Rule}
\begin{itemize}
  \item $A$: \quad $A$ is a type
  \item $B$: \quad For $x \in A : B(x)$ is a type
  \item $b$: \quad For $x \in A : b(x) \in B(x)$
\end{itemize}

\textbf{Axioms} -- None

A model of this theory will consist of a set, a family indexed by this set and a distinguished element of each set in this family; which is to say that a model will consist of a set indexed family of elements of a family of sets. We are not sure of the notation that we should be using but if we denote the interpretation of a symbol in a model $\mathcal{M}$ by that symbol superscripted by $\mathcal{M}$ then a model $\mathcal{M}$ consists of i. a set $A^{\mathcal{M}}$, ii. an $A^{\mathcal{M}}$-indexed family of elements $b^{\mathcal{M}}$ of the family of sets $B^{\mathcal{M}}$.

\begin{center}
  \begin{tikzpicture}
    % Draw set A
    \draw[thick] (0,0) ellipse (1cm and 1.5cm);
    \node at (0,1.8) {$A^{\mathcal{M}}$};
    \filldraw (0,-0.5) circle (1pt) node[anchor=west] {$a$};

    % Draw set B(a)
    \draw[thick] (4,0) ellipse (0.8cm and 0.8cm);
    \node at (4.8,0.8) {$B^{\mathcal{M}}(a)$};
    \filldraw (4,-0.2) circle (1pt) node[anchor=west] {$b^{\mathcal{M}}(a)$};

    % Arrow or connection? The text implies a mapping structure.
    % Just representing the sets as blobs per the rough figure description.
  \end{tikzpicture}

  \small{Fig. 1. - for every element $a$ of the set $A^{\mathcal{M}}$ we have i. a set $B^{\mathcal{M}}(a)$ and ii. an element $b^{\mathcal{M}}(a)$ of the set $B^{\mathcal{M}}(a)$.}
\end{center}

If both $\mathcal{M}$ and $\mathcal{M}'$ are models of this theory then a homomorphism $f : \mathcal{M} \to \mathcal{M}'$ consists of a function $f_A : A^{\mathcal{M}} \to A^{\mathcal{M}'}$ and an $A^{\mathcal{M}}$-indexed family of functions $f_B$ such that for every $a \in A^{\mathcal{M}}$, $f_B(a) : B^{\mathcal{M}}(a) \to B^{\mathcal{M}'}(f_A(a))$ and such that for every $a \in A^{\mathcal{M}}$, $f_B(a)(b^{\mathcal{M}}(a)) = b^{\mathcal{M}'}(f_A(a))$.

Alternatively we can say that a homomorphism consists of a function $f_A : A^{\mathcal{M}} \to A^{\mathcal{M}'}$ and an operator $f_B$ such that for every $a \in A^{\mathcal{M}}$, for every $b \in B^{\mathcal{M}}(a)$, $f_B(a,b) \in B^{\mathcal{M}'}(f_A(a))$ and satisfying $f_B(a, b^{\mathcal{M}}(a)) = b^{\mathcal{M}'}(f_A(a))$, whenever $a \in A^{\mathcal{M}}$. Now this means that there is a generalised algebraic theory whose models are just homomorphisms between the models of the given theory (in fact this is quite generally the case). This theory of homomorphisms can be presented as follows:

The theory of families of elements of families of sets in the language $\langle A, B, b \rangle$ + the same theory in the language $\langle A', B', b' \rangle$ +

\textbf{Symbol} \quad \textbf{Introductory Rule}
\begin{itemize}
  \item $f_A$: \quad For $x \in A : f_A(x) \in A'$.
  \item $f_B$: \quad For $x \in A$, for $y \in B(x) : f_B(x,y) \in B'(f_A(x))$.
\end{itemize}

\textbf{Axiom.}
\begin{itemize}
  \item $f_B(x, b(x)) = b'(f_A(x))$, whenever $x \in A$.
\end{itemize}

An example similar to the first example we call the theory of families of families of elements of families of families of sets:

\textbf{Symbol} \quad \textbf{Introductory Rule}
\begin{itemize}
  \item $A$: \quad $A$ is a type
  \item $B$: \quad For $x \in A : B(x)$ is a type
  \item $C$: \quad For $x \in A$, for $y \in B(x) : C(x,y)$ is a type
  \item $c$: \quad For $x \in A$, for $y \in B(x) : c(x,y) \in C(x,y)$
\end{itemize}

\textbf{Axioms} -- None.

Suppose that $\mathcal{M}$ is a model of this theory. Then $A^{\mathcal{M}}$ is a set. For every element $a$ of the set $A^{\mathcal{M}}$ we have a set $B^{\mathcal{M}}(a)$ and for every element $b$ of the set $B^{\mathcal{M}}(a)$ we have a set $C^{\mathcal{M}}(a,b)$ and an element $c^{\mathcal{M}}(a,b)$ of the set $C^{\mathcal{M}}(a,b)$.

Now for every element $a$ of $A^{\mathcal{M}}$, $\lambda b . C^{\mathcal{M}}(a,b)$ is a $B^{\mathcal{M}}$-indexed family of sets. Thus $\lambda a . \lambda b . C^{\mathcal{M}}(a,b)$, i.e. $C^{\mathcal{M}}$, is an $A^{\mathcal{M}}$-indexed family of families of sets. Similarly $c^{\mathcal{M}}$ is an $A^{\mathcal{M}}$-indexed family of families of elements.

Note that in the presentation of this theory no harm is done if we replace the introductory rule for $C$ by the rule:
\[ \text{for } x \in A, \text{for } y \in B(x) : C(y) \text{ is a type} \]
this rule having the same meaning as the given rule. The expression $C(x,y)$ in the given rule depends explicitly on $x$ and $y$. We say that the expression $C(y)$ in the alternative rule depends implicitly on $x$ by virtue of its explicit dependence on $y$ and by virtue of the dependence of $y$ on $x$. In the alternative version of the theory we say that a variable has been omitted. This is one way in which a theory may be informally presented. We use this method and another in presenting the next theory -- the theory of trees.

The theory of trees has countably many sort symbols, no operator symbols and no axioms. However, we chose to write the theory informally with just two sort symbols, one of these symbols doing the work that in a formal presentation would be shared among countably many distinct symbols.

\textbf{Symbol} \quad \textbf{Introductory Rule}
\begin{itemize}
  \item $S_1$: \quad $S_1$ is a type
  \item $S$: \quad For $x_1 \in S_1 : S(x_1)$ is a type
  \item $S$: \quad For $x_1 \in S_1$, for $x_2 \in S(x_1) : S(x_2)$ is a type
  \item[] \quad \vdots
  \item $S$: \quad For $x_1 \in S_1$, for $x_2 \in S(x_1), \dots, \text{for } x_n \in S(x_{n-1}) : S(x_n)$ is a type
  \item[] \quad \vdots
\end{itemize}

\textbf{Axioms} -- None.

$S_1$, then, is a symbol denoting the set of nodes at the base of the tree. If $x$ is any node of the tree then $S(x)$ is the set of nodes immediately above $x$ in the tree, that is to say the set of successor nodes to $x$. In a formal presentation of this theory there would be symbols $S_1, S_2, S_3, \dots$ and the symbol $S_{n+1}$ would be introduced by the rule
\[ \frac{x_1 \in S_1, x_2 \in S_2(x_1), \dots, x_n \in S_n(x_1, \dots, x_{n-1})}{S_{n+1}(x_1, \dots, x_n) \text{ is a type}} \]

We use the same methods in presenting the theory of functors informally. The theory of functors consists of the theory of categories in the language $\langle Ob, Hom, id, o \rangle$ + the theory of categories in the language $\langle Ob', Hom', id', o' \rangle$ (and at this point we have used the same three symbols $Hom, id$ and $o$ in new roles) +

\textbf{Symbol} \quad \textbf{Introductory Rule}
\begin{itemize}
  \item $F$: \quad For $x \in Ob : F(x) \in Ob'$
  \item $F$: \quad For $x,y \in Ob$, for $f \in Hom(x,y) : F(f) \in Hom'(F(x), F(y))$
\end{itemize}

\textbf{Axioms.}
\begin{itemize}
  \item $F(id(x)) = id'(F(x))$, whenever $x \in Ob$.
  \item $F(o(f,g)) = o'(F(f), F(g))$, whenever $x,y,z \in Ob, f \in Hom(w,x)$ and $g \in Hom(y,z)$.
\end{itemize}

A model of this theory is just a functor. The category of models is the category $Cat^2$, which is to say that if $F : \mathcal{C} \to \mathcal{C}'$ is a functor and if $G : \mathcal{D} \to \mathcal{D}'$ is a functor then a homomorphism from $F$ to $G$ consists of a pair of functors $\langle H, H' \rangle$ such that $H : \mathcal{C} \to \mathcal{D}$ and $H' : \mathcal{C}' \to \mathcal{D}'$ and such that
\begin{center}
  \begin{tikzcd}
    \mathcal{C} \arrow[r, "H"] \arrow[d, "F"] & \mathcal{D} \arrow[d, "G"] \\
    \mathcal{C}' \arrow[r, "H'" description] & \mathcal{D}'
  \end{tikzcd}
\end{center}
commutes.

The final example is to indicate one way of axiomatising the disjoint union of a family of types.

If $U$ is a theory which includes a type symbol $A$ and a symbol $B$ for an $A$-indexed family of types then $U$ can be extended by three operator symbols, three axioms and one type symbol $\Sigma_A B$ in such a way that i. every model $\mathcal{M}$ of $U$ uniquely extends to a model of the extended theory and ii. every model $\mathcal{M}$ of the extended theory interprets the symbol $\Sigma_A B$ by the set $\{ \langle a,b \rangle \mid a \in A^{\mathcal{M}} \text{ and } b \in B^{\mathcal{M}}(a) \}$, that is to say as the disjoint union of the family of sets interpreting $B$. The extended theory is taken to be $U +$

\textbf{Symbol} \quad \textbf{Introductory Rule}
\begin{itemize}
  \item $\Sigma_A B$: \quad $\Sigma_A B$ is a type
  \item $p_1$: \quad For $z \in \Sigma_A B : p_1(z) \in A$
  \item $p_2$: \quad For $z \in \Sigma_A B : p_2(z) \in B(p_1(z))$
  \item $Pr$: \quad For $x \in A$, for $y \in B(x) : Pr(x,y) \in \Sigma_A B$
\end{itemize}

\textbf{Axioms}
\begin{itemize}
  \item $Pr(p_1(z), p_2(z)) = z$, whenever $z \in \Sigma_A B$.
  \item $p_1(Pr(x,y)) = x$, whenever $x \in A$ and $y \in B(x)$.
  \item $p_2(Pr(x,y)) = y$, whenever $x \in A$ and $y \in B(x)$.
\end{itemize}

In future we might refer to an extension of a theory by symbols for disjoint unions of specified families of types.

\section{Predicates as types}

It is possible to introduce sort symbols into a generalised algebraic theory and then axiomatise them in such a way as they are effectively predicate symbols. In this way any theory of predicate calculus all of whose axioms are of the form $\forall \bar{x} (\phi_1 \wedge \phi_2 \dots \wedge \phi_n \to \psi)$, where $\phi_1, \dots, \phi_n$ and $\psi$ are all atomic, can be expressed as generalised algebraic. Let us call such an axiom a universal condition.

We do not work with relations directly but rather with their characteristic families. If $R$ is an $n$-ary relation on a set $A$ then its characteristic family is the family $\lambda a_1 . \lambda a_2 \dots \lambda a_n . P(a_1, \dots, a_n)$, where $P(a_1, \dots, a_n) = \{ \emptyset \}$ if $R(a_1, \dots, a_n)$ and $P(a_1, \dots, a_n) = \emptyset$ otherwise.

The following theory indicates how an $n$-ary predicate symbol may be introduced into a theory. The given theory has as models just characteristic families of $n$-ary relations on a set.

\textbf{Symbol} \quad \textbf{Introductory Rule}
\begin{itemize}
  \item $A$: \quad $A$ is a type
  \item $P$: \quad For $x_1, \dots, x_n \in A : P(x_1, \dots, x_n)$ is a type
\end{itemize}

\textbf{Axiom}
\begin{itemize}
  \item $y_1 = y_2$, whenever $x_1, \dots, x_n \in A$ and $y_1, y_2 \in P(x_1, \dots, x_n)$.
\end{itemize}

It remains to show how universal conditions may be expressed as generalised algebraic. We distinguish three forms that such a condition might take. The first case is when each of $\phi_1, \dots, \phi_n$ and $\psi$ are instances of a predicate other than the equality predicate. In this case $\forall \bar{x} (\phi_1 \wedge \dots \wedge \phi_n \to \psi)$ cannot be expressed as an axiom but can be expressed merely by the introduction of a new operator symbol. For example the transitivity of a binary predicate $P$ is expressed by the introduction of a new operator symbol $t$ by the rule: for $x_1, x_2, x_3 \in A$, for $y_1 \in P(x_1, x_2)$ and for $y_2 \in P(x_2, x_3) : t(y_1, y_2) \in P(x_1, x_3)$. The point is that once $P$ is interpreted then $t$ is interpretable in at most one way and then only in case the predicate is transitive.

The second case to consider is the case where each of $\phi_1, \dots, \phi_n$ are instances of a predicate other than the equality predicate and $\psi$ is an instance of the equality predicate. In this case $\forall \bar{x} (\phi_1 \wedge \dots \wedge \phi_n \to \psi)$ can be expressed as an axiom of the theory. For example the anti-symmetry of a binary predicate $P$ can be expressed by the axiom: $x_1 = x_2$, whenever $x_1, x_2 \in A$ and $y_1 \in P(x_1, x_2), y_2 \in P(x_2, x_1)$.

Lastly we must consider the case when one of the $\phi_i$'s is an instance of the equality predicate. In this case a new binary predicate must be added to the language and axiomatised to be the equality predicate. The axiom $\forall \bar{x} (\phi_1 \wedge \dots \wedge \phi_n \to \psi)$ can then be dealt with by one of the first two cases. The following theory indicates the way in which the binary predicate $Eq$ can be added to a theory and axiomatised to be the equality predicate.

\textbf{Symbol} \quad \textbf{Introductory Rule}
\begin{itemize}
  \item $A$: \quad $A$ is a type
  \item $Eq$: \quad For $x_1, x_2 \in A : Eq(x_1, x_2)$ is a type
  \item $r$: \quad For $x \in A : r(x) \in Eq(x,x)$
\end{itemize}

\textbf{Axioms}
\begin{itemize}
  \item $y_1 = y_2$, whenever $x_1, x_2 \in A$ and $y_1, y_2 \in Eq(x_1, x_2)$
  \item $x_1 = x_2$, whenever $x_1, x_2 \in A$ and $y \in Eq(x_1, x_2)$.
\end{itemize}

One final example. The theory of a 1-1 function is the theory of equality in the language $\langle B, Eq, r \rangle$ +

\textbf{Symbol} \quad \textbf{Introductory Rule}
\begin{itemize}
  \item $A$: \quad $A$ is a type
  \item $f$: \quad For $x \in A : f(x) \in B$.
\end{itemize}

\textbf{Axioms}
\begin{itemize}
  \item $x_1 = x_2$, whenever $x_1, x_2 \in A$ and $y \in Eq(f(x_1), f(x_2))$.
\end{itemize}

\section{Essentially Algebraic Theories and Categories with Finite Limits}

The essentially algebraic theories of Freyd [5] can be seen to have the same descriptive power as generalised algebraic theories, at least as far as the usual set valued models are concerned. In this section we look at the relationship between these two notions and also at the relationship between essentially algebraic theories and categories with all finite limits. In the next section we point out the way in which generalised algebraic is a more general notion than essentially algebraic.

Essentially algebraic theories are introduced and very briefly discussed in Freyd [5]; they are many sorted partial algebraic theories such that the domain of every partial operation is specified as the extension of some conjunction of identities between terms compounded from previously introduced operators.

Thus the theory of categories as an essentially algebraic theory has two sorts, $Ob$ and $morph$, three total operations, $dom : Morph \to Ob$, $cod : Morph \to Ob$ and $id : Ob \to Morph$, and one binary partial operation $o$ from $Morph \times Morph$ to $Morph$ whose domain is specified by asserting that $o(x,y)$ is defined iff $cod(x) = dom(y)$.

In order to write an essentially algebraic theory as generalised algebraic, all the equality predicates used in defining domains of partial operations must be introduced. For example if $f$ is to be a partial $C$-valued function defined on $\{ x \in A \mid t_1 = t_2 \}$, where for $x \in A : t_1 \in B$ and for $x \in A : t_2 \in B$, then the equality predicate on $B$ must be introduced and axiomatised. Then $f$ can be introduced by the rule for $x \in A$, for $y \in Eq(t_1, t_2) : f(x,y) \in C$.

In this way every essentially algebraic theory can be rewritten as generalised algebraic. The converse is also the case, at least in so far as that to every generalised algebraic theory there corresponds an essentially algebraic theory with the same category of models. This is the case because of the equivalence between $A$-indexed families of sets and morphisms in the category \underline{Set} with codomain $A$. This equivalence holds for any set $A$ and is given by the following:
\begin{enumerate}
  \item If $\{ B(a) \mid a \in A \}$ is an $A$-indexed family of sets then $proj : \bigcup B(a) \to A$ is a morphism of \underline{Set} with codomain $A$ (remember that $\bigcup B(a) = \{ \langle a,b \rangle \mid a \in A, b \in B(a) \}$).
  \item If $f : A' \to A$ is a map in \underline{Set} with codomain $A$ then $\{ f^{-1}(a) \mid a \in A \}$ is an $A$-indexed family of sets.
\end{enumerate}
1. and 2. establish an isomorphism between the class of $A$-indexed families of sets and the class of functions with codomain $A$. Thus, if in a generalised algebraic theory there is a sort symbol $B$ introduced as an $A$-indexed family of types then in the corresponding essentially algebraic theory there is introduced a new sort symbol $A'$ and a map $p : A' \to A$.

The notion of an essentially algebraic theory can be seen as a notion of type theory in which the only type forming principles are for the formation of product types and for the formation of types of the form $\{ x \in \Delta \mid t_1 = t_2 \}$, where $\Delta$ is a type and $t_1$ and $t_2$ are terms of the same type. Now if we think of the objects of an arbitrary category as types then to have these two type forming principles is just to have finite products and equalisers of pairs. Since a category with finite products and equalisers of pairs is precisely a category with finite limits, the notions of essentially algebraic theory and category with finite limits are closely connected.

In fact for every essentially algebraic theory $U$ there is a category with finite limits $\mathcal{C}(U)$ such that the category of models of $U$ is equivalent to the category $LEX(\mathcal{C}(U), \underline{Set})$ of all finite limit preserving functors from $\mathcal{C}(U)$ to \underline{Set}, with all natural transformations between them as morphisms.

This is the content of a remark made by Lawvere, pages 8-9 of Lawvere [17], though the remark does not actually use the term essentially algebraic.

\section{The Extra Generality of the Algebraic Semantics}

One of the advantages of generalised algebraic over essentially algebraic is to be found in the syntax particularly with regard to the presentation of theories. In presenting theories as essentially algebraic there is a coding process in that, in general, families of sets indexed by a set are represented by functions with codomain that set. On the other hand in presenting a theory as generalised algebraic there need be no such coding. This distinction whereby families in a generalised algebraic theory can have a life of their own goes through into the algebraic semantics. The algebraic semantics of generalised algebraic theories is more general than any possible such semantics for essentially algebraic theories. There are perfectly coherent interpretations of generalised algebraic theories into structures in which the elements of the structure that are there to interpret type-indexed families of types are distinct from the elements that are there to interpret functions with codomain. This can never be the case with essentially algebraic theories because already in the syntax families of types are coded as functions with codomain.

The notion of type is adequately captured by the notion of object of a category. However having decided to think of the objects of a particular category $\mathcal{C}$ as types and in particular having decided to think of an actual object $A$ of $\mathcal{C}$ as a type then it is incorrect then to suppose that the notion of $A$-indexed family of types should always be taken to be the notion morphism of $\mathcal{C}$ with codomain $A$. This is just one possibility. In general, there will be other possibilities some of which may be more attractive. The example that we have in mind is when $\mathcal{C}$ is taken to be the category \underline{Cat} of all (small) categories. Now the question is what shall we chose to mean by a category indexed family of categories? In particular if $A$ is a category then what shall we mean by an $A$-indexed family of categories? Well, what we would like to mean by that is any functor $B : A \to \underline{Cat}$. This is not the same as taking it to mean a morphism of \underline{Cat} with codomain $A$. The idea that a family of categories indexed by the category $A$ should just be a functor $B : A \to \underline{Cat}$ arises because there is a category of all (small) categories just as the fact that there is a class $\mathbb{U}$ of all sets (= small classes) leads to the definition of a family of sets indexed by a set $A$ as a function $B : A \to \mathbb{U}$.

A functor $B : A \to \underline{Cat}$ can be thought of as a structure of the general kind (for example take sort symbols $ObA, HomA, ObB$ and $HomB$ introduced by rules $ObA$ is a type; for $x,y \in ObA : HomA(x,y)$ is a type; for $x \in ObA : ObB(x)$ is a type; for $x \in ObA$, for $y,z \in ObB(x) : HomB(x,y,z)$ is a type). It follows that there is a category of category indexed families of categories and structure preserving homomorphisms (it turns out that a homomorphism from $B : A \to \underline{Cat}$ to $B' : A' \to \underline{Cat}$ is describable just as a pair $F, N$ where $F : A \to A'$ is a functor and $N : B \to F \circ B'$ is a natural transformation). But now there follows the notion of a category indexed family of category indexed families of categories. This procedure can be iterated. We get a huge structure of categories, category indexed families of categories, category indexed families of category indexed families of categories and so on. It is a structure into which generalised algebraic theories can be interpreted -- by interpreting types as categories, type indexed families of types as category indexed families of categories and so on.

A model of the theory of families of elements of families of sets in this structure will consist of a category $A$, a functor $B : A \to \underline{Cat}$, for each $a \in |A|$, an object $b(a)$ of $B(a)$ and for each $f : a \to a'$ in $A$, a morphism $b(f) : B(f)(b(a)) \to b(a')$ in $B(a')$. Such that $b(id(a)) = id(b(a))$, for all objects $a$ of $A$ and such that $B(f')(b(f)) \circ b(f') = b(f \circ f')$, whenever $a \xrightarrow{f} a' \xrightarrow{f'} a''$ in $A$.

\section{The Formal Definition}

We must insist that the introductory rule for a symbol into a language be well formed. In order that we may say what it is for a rule to be well formed we require a notion of derivability. Since the notion of derivability depends upon the introductory rules there is a difficulty in giving the formal definition.

The difficulty is that we need knowledge of the derived rules of a theory when we are still in the process of defining the possible languages in which the theory may be written. We choose to overcome the difficulty by leaving aside the question of wellformedness until we have available the complete set of derived rules of the theory. For this reason the theories that are admitted by the definition below may not be well formed; we call them pretheories and accept that they might make little sense. Later we shall define a theory to be a well formed pretheory.

We assume throughout that we have a set $V$ of variables which has countably many distinct members. We begin by giving a definition of rule, more precisely a definition of rule of the alphabet $W$. The definition is crude in that most of the permitted rules are meaningless in all circumstances; it does suffice, though, for the purpose of turning rules into objects. Suppose then that $W$ is a set. We consider the set $W$ to be an alphabet and its elements to be symbols. The following definition is relative to $W$ (and, of course, it is relative to the set of variables $V$, but $V$ will remain fixed throughout).

The set of \underline{expressions} is defined inductively in such a way that every expression is a finite sequence of elements of $W \cup V \cup \{ (, ), , \}$ by the clauses:
1. If $x \in V$ then $x$ is an expression.
2. If $L \in W$ then $L$ is an expression.
3. If $L \in W$ and $e_1, \dots, e_n$ are expressions then $L(e_1, \dots, e_n)$ is an expression.

A \underline{premise} is defined to be any finite sequence of elements of $V \times$ the set of expressions. The empty sequence is included as a premise, called funnily enough, the empty premise. The premise determined by $((x_1, \Delta_1), \dots, (x_n, \Delta_n))$ is written as $x_1 \in \Delta_1, \dots, x_n \in \Delta_n$.

A \underline{T-conclusion} is determined by a single expression $\Delta$ and is written as `$\Delta$ is a type'.

An \underline{$\in$-conclusion} is determined by a pair of expressions $(t, \Delta)$ and is written as `$t \in \Delta$'.

A \underline{T=-conclusion} is determined by a pair of expressions $(\Delta, \Delta')$ and is written as `$\Delta = \Delta'$'.

An \underline{$\in=$-conclusion} is determined by a triple of expressions $(t, t', \Delta)$ and is written as `$t=t' \in \Delta$'.

A \underline{rule} is determined by a premise $P$ and a conclusion $C$ and is written as $\frac{P}{C}$. A rule is said to be a T-rule, an $\in$-rule, a T=-rule or an $\in=$-rule according as to the form of its conclusion.

If $\Delta, t_1, \dots, t_n$ are expressions and if $x_1, \dots, x_n$ are distinct variables then the expression $\Delta[t_1|x_1, \dots, t_n|x_n]$ is that expression which results from simultaneously replacing every occurrence of the variables $x_1, \dots, x_n$ in the expression $\Delta$ by $t_1, \dots, t_n$. Please note that $\Delta[t_1|x_1, \dots, t_n|x_n]$ and $\Delta[t_1|x_1] \dots [t_n|x_n]$ are not usually the same, indeed $\Delta[t_1|x_1, t_2|x_2]$ and $\Delta[t_1|x_1][t_2|x_2]$ are distinct whenever $x_1$ appears in $\Delta$, $x_2$ appears in $t_1$ and $t_2$ is distinct from $x_2$.

We can now give the main definitions.

\begin{definition}
  A \underline{pretheory} consists of 1. a set $S$, called the set of sort symbols. 2. A set $Z$ called the set of operator symbols. 3. To each sort symbol $A$, an associated rule of the alphabet $S \cup Z$, called the introductory rule for $A$ and of the form
  \[ \frac{x_1 \in \Delta_1, \dots, x_n \in \Delta_n}{A(x_1, \dots, x_n) \text{ is a type}} \]
  for some $n \ge 0$. 4. To each operator symbol $F$, an associated rule of the alphabet $S \cup Z$ called the introductory rule for $F$ and of the form
  \[ \frac{x_1 \in \Delta_1, \dots, x_n \in \Delta_n}{F(x_1, \dots, x_n) \in \Delta} \]
  for some $n \ge 0$. 5. A set of axioms. Each axiom is either a T=-rule or an $\in=$-rule of the alphabet.
\end{definition}

Taken together, definition 2(a) and 2(b) define the derived rules of a pretheory. The definition is of an inductive nature.

\begin{definition}[2(a)]
  If $U$ is a pretheory then (i) a \underline{context} is a premise $x_1 \in \Delta_1, \dots, x_n \in \Delta_n$ such that the rule
  \[ \frac{x_1 \in \Delta_1, \dots, x_{n-1} \in \Delta_{n-1}}{\Delta_n \text{ is a type}} \]
  is a derived rule of $U$.
  (ii) (a) The rule $\frac{x_1 \in \Delta_1, \dots, x_n \in \Delta_n}{\Delta \text{ is a type}}$ is \underline{wellformed} iff $x_1 \in \Delta_1, \dots, x_n \in \Delta_n$ is a context.
  (b) The rule $\frac{x_1 \in \Delta_1, \dots, x_n \in \Delta_n}{t \in \Delta}$ is \underline{wellformed} iff $\frac{x_1 \in \Delta_1, \dots, x_n \in \Delta_n}{\Delta \text{ is a type}}$ is a derived rule of $U$.
  (c) The rule $\frac{x_1 \in \Delta_1, \dots, x_n \in \Delta_n}{\Delta = \Delta'}$ is \underline{wellformed} iff $\frac{x_1 \in \Delta_1, \dots, x_n \in \Delta_n}{\Delta \text{ is a type}}$ and $\frac{x_1 \in \Delta_1, \dots, x_n \in \Delta_n}{\Delta' \text{ is a type}}$ are both derived rules of $U$.
  (d) The rule $\frac{x_1 \in \Delta_1, \dots, x_n \in \Delta_n}{t=t' \in \Delta}$ is \underline{wellformed} iff $\frac{x_1 \in \Delta_1, \dots, x_n \in \Delta_n}{t \in \Delta}$ and $\frac{x_1 \in \Delta_1, \dots, x_n \in \Delta_n}{t' \in \Delta}$ are both derived rules of $U$.
\end{definition}

\begin{definition}[2(b)]
  The set of derived rules of $U$ is the set of rules derivable by the following \underline{principles of derivation}.

  \begin{enumerate}
    \item[LI1.] From $\frac{P}{\Delta \text{ is a type}}$ derive $\frac{P}{\Delta = \Delta}$

    \item[LI2.] From $\frac{P}{t \in \Delta}$ derive $\frac{P}{t=t \in \Delta}$

    \item[LI3.] From $\frac{P}{\Delta_1 = \Delta_2}$ derive $\frac{P}{\Delta_2 = \Delta_1}$

    \item[LI4.] From $\frac{P}{t_1 = t_2 \in \Delta}$ derive $\frac{P}{t_2 = t_1 \in \Delta}$

    \item[LI5.] From $\frac{P}{\Delta_1 = \Delta_2}$ and $\frac{P}{\Delta_2 = \Delta_3}$ derive $\frac{P}{\Delta_1 = \Delta_3}$

    \item[LI6.] From $\frac{P}{t_1 = t_2 \in \Delta}$ and $\frac{P}{t_2 = t_3 \in \Delta}$ derive $\frac{P}{t_1 = t_3 \in \Delta}$

    \item[LI7.] From $\frac{P}{t_1 = t_2 \in \Delta_1}$ and $\frac{P}{\Delta_1 = \Delta_2}$ derive $\frac{P}{t_1 = t_2 \in \Delta_2}$

    \item[T1.] From $\frac{P}{\Delta_1 = \Delta_2}$ and $\frac{P}{t \in \Delta_1}$ derive $\frac{P}{t \in \Delta_2}$

    \item[CF1.] For $n \ge 0$, $1 \le i \le n+1$.
      From $\frac{x_1 \in \Delta_1, \dots, x_n \in \Delta_n}{\Delta_{n+1} \text{ is a type}}$ derive $\frac{x_1 \in \Delta_1, \dots, x_{n+1} \in \Delta_{n+1}}{x_i \in \Delta_i}$, providing that $x_{n+1}$ is a variable distinct from all of $x_1, \dots, x_n$.

    \item[CF2(a).] For every sort symbol $A$ with wellformed introductory rule $\frac{x_1 \in \Delta_1, \dots, x_n \in \Delta_n}{A(x_1, \dots, x_n) \text{ is a type}}$, for every context $P$, from
      \[ \frac{P}{t_1 \in \Delta_1} \quad \frac{P}{t_2 \in \Delta_2[t_1|x_1]} , \dots, \frac{P}{t_n \in \Delta_n[t_1|x_1, \dots, t_{n-1}|x_{n-1}]} \]
      derive
      \[ \frac{P}{A(t_1, \dots, t_n) \text{ is a type}} \]

    \item[CF2(b).] For every operator symbol $F$ with wellformed introductory rule $\frac{x_1 \in \Delta_1, \dots, x_n \in \Delta_n}{F(x_1, \dots, x_n) \in \Delta}$, for every context $P$, from
      \[ \frac{P}{t_1 \in \Delta_1} \quad \frac{P}{t_2 \in \Delta_2[t_1|x_1]} , \dots, \text{ and } \frac{P}{t_n \in \Delta_n[t_1|x_1, \dots, t_{n-1}|x_{n-1}]} \]
      derive
      \[ \frac{P}{F(t_1, \dots, t_n) \in \Delta[t_1|x_1, \dots, t_n|x_n]} \]

    \item[SI1.] If $Q$ is a context then from $\frac{y_1 \in \Omega_1, \dots, y_m \in \Omega_m}{\Omega = \Omega'}$ and
      \[ \frac{Q}{s_1 = s'_1 \in \Omega_1[s_1|y_1]} \dots \frac{Q}{s_m = s'_m \in \Omega_m[s_1|y_1, \dots, s_{m-1}|y_{m-1}]} \]
      derive
      \[ \frac{Q}{\Omega[s_1|y_1, \dots, s_m|y_m] = \Omega'[s'_1|y_1, \dots, s'_m|y_m]} \]

    \item[SI2.] If $Q$ is a context then from $\frac{y_1 \in \Omega_1, \dots, y_m \in \Omega_m}{s = s' \in \Omega}$ and
      \[ \frac{Q}{s_1 = s'_1 \in \Omega_1[s_1|y_1]} \dots \frac{Q}{s_m = s'_m \in \Omega_m[s_1|y_1, \dots, s_{m-1}|y_{m-1}]} \]
      derive
      \[ \frac{Q}{s[s_1|y_1, \dots, s_m|y_m] = s'[s'_1|y_1, \dots, s'_m|y_m] \in \Omega[s_1|y_1, \dots, s_m|y_m]} \]

    \item[A1.] If $\frac{x_1 \in \Delta_1, \dots, x_n \in \Delta_n}{\Delta = \Delta'}$ is an axiom then from
      $\frac{x_1 \in \Delta_1, \dots, x_n \in \Delta_n}{\Delta \text{ is a type}}$ and $\frac{x_1 \in \Delta_1, \dots, x_n \in \Delta_n}{\Delta' \text{ is a type}}$ derive $\frac{x_1 \in \Delta_1, \dots, x_n \in \Delta_n}{\Delta = \Delta'}$

    \item[A2.] If $\frac{x_1 \in \Delta_1, \dots, x_n \in \Delta_n}{t = t' \in \Delta}$ is an axiom then from
      $\frac{x_1 \in \Delta_1, \dots, x_n \in \Delta_n}{t \in \Delta}$ and $\frac{x_1 \in \Delta_1, \dots, x_n \in \Delta_n}{t' \in \Delta}$ derive $\frac{x_1 \in \Delta_1, \dots, x_n \in \Delta_n}{t = t' \in \Delta}$
  \end{enumerate}
\end{definition}

\begin{definition}
  A pretheory is \underline{wellformed} iff all of its introductory rules and axioms are wellformed. A \underline{generalised algebraic theory} is a wellformed pretheory.
\end{definition}

\section{The Substitution Lemma and Other Lemmas}

Each lemma in this section is needed at some later stage. For example the substitution lemma, which asserts that the set of derived rules of a theory is closed under the operation of the substitution of correctly typed terms for variables, is needed in the definition of the category of generalised algebraic theories.

Substitution could have been taken as one of the principles of derivation; however to have done this would have hindered the definitions by induction which surround the semantics. Compare with Lambek [19], though of course the problem is Gentzen's.

It is assumed throughout that $U$ is some generalised algebraic theory. Let us say that a derived rule of $U$ of the form
\[ \frac{y_1 \in \Omega_1, \dots, y_m \in \Omega_m}{\text{Conclusion}} \]
has the \underline{substitution property} iff for every context $Q$ of $U$, whenever $s_1, s_2, \dots, s_m$ are expressions such that
\[ \frac{Q}{s_1 \in \Omega_1}, \quad \frac{Q}{s_2 \in \Omega_2[s_1|y_1]}, \dots \]
and
\[ \frac{Q}{s_m \in \Omega_m[s_1|y_1, \dots, s_{m-1}|y_{m-1}]} \]
are all derived rules of $U$ then the rule
\[ \frac{Q}{\text{Conclusion}[s_1|y_1, \dots, s_m|y_m]} \]
is also a derived rule of $U$.

We aim to show that all derived rules of $U$ have the substitution property. We need two preliminary lemmas.

\begin{lemma}
If $\frac{x_1 \in \Delta_1, \dots, x_n \in \Delta_n}{\text{Conclusion}}$ is a derived rule of $U$ then any variables appearing in the conclusion occur among $\{x_1, \dots, x_n\}$.
\end{lemma}

\begin{proof}
By induction on the derivation of rules in $U$. Look at each principle of derivation in turn and see that it is impossible to use the principle to derive a rule without this property from rules which do have the property. This is very easy to see.
\end{proof}

\begin{lemma}
(i) The premise of a derived rule is a context. (ii) If $x_1 \in \Delta_1, \dots, x_n \in \Delta_n$ is a context then for all $i$, $1 \le i \le n$, the rule
\[ \frac{x_1 \in \Delta_1, \dots, x_{i-1} \in \Delta_{i-1}}{\Delta_i \text{ is a type}} \]
is a derived rule.
\end{lemma}

\begin{proof}
(i) is proved by induction on derivations. If each principle of derivation is checked it will be seen that the premise of the derived rule is either a context by hypothesis (CF2, SI1 and SI2), or is a premise of a previously derived rule (LI1,...LI7, T1, R1 and A2), or else satisfies the conditions necessary to be a context (CF1).

(ii) follows from an iteration of (i). If $x_1 \in \Delta_1, \dots, x_n \in \Delta_n$ is a context then $\frac{x_1 \in \Delta_1, \dots, x_{n-1} \in \Delta_{n-1}}{\Delta_n \text{ is a type}}$ is a derived rule. Hence by (i), $x_1 \in \Delta_1, \dots, x_{n-1} \in \Delta_{n-1}$ is a context. Continue until you get to $x_1 \in \Delta_1, \dots, x_i \in \Delta_i$ is a context and $\frac{x_1 \in \Delta_1, \dots, x_{i-1} \in \Delta_{i-1}}{\Delta_i \text{ is a type}}$ is a derived rule.
\end{proof}

\begin{theorem}[The Substitution Lemma]
Every derived rule of the theory $U$ has the substitution property.
\end{theorem}

\begin{proof}
The derived T=rules and $\in$=rules of $U$ have the substitution property because there are principles of derivation SI1 and SI2 which have just that effect.

The proof that T and $\in$-rules of $U$ have the substitution property is by induction on derivations in $U$. It suffices to show that no principle of derivation by which such rules are derived can be used to derive a rule without the property from rules with the property. Thus we just have to check the principles T1, CF1 and CF2.

\underline{T1.} Suppose that both $\frac{y_1 \in \Omega_1, \dots, y_m \in \Omega_m}{\Delta_1 = \Delta_2}$ and $\frac{y_1 \in \Omega_1, \dots, y_m \in \Omega_m}{t \in \Delta_1}$ are derived rules of $U$ which have the substitution property. We must show that $\frac{y_1 \in \Omega_1, \dots, y_m \in \Omega_m}{t \in \Delta_2}$ has the substitution property. So suppose that for each $j$, $1 \le j \le m$,
\[ \frac{Q}{s_j \in \Omega_j[s_1|y_1, \dots, s_{j-1}|y_{j-1}]} \]
is a derived rule of $U$. By our first assumption both
\[ \frac{Q}{\Delta_1[s_1|y_1, \dots, s_m|y_m] = \Delta_2[s_1|y_1, \dots, s_m|y_m]} \]
and
\[ \frac{Q}{t[s_1|y_1, \dots, s_m|y_m] \in \Delta_1[s_1|y_1, \dots, s_m|y_m]} \]
are derived rules of $U$. Thus, by an application of T1, so is
\[ \frac{Q}{t[s_1|y_1, \dots, s_m|y_m] \in \Delta_2[s_1|y_1, \dots, s_m|y_m]} \]
a derived rule of $U$. Hence $\frac{y_1 \in \Omega_1, \dots, y_m \in \Omega_m}{t \in \Delta_2}$ has the substitution property.

\underline{CF1.} Suppose that $\frac{x_1 \in \Delta_1, \dots, x_n \in \Delta_n}{\Delta_{n+1} \text{ is a type}}$ is a derived rule of $U$ having the substitution property. We must show that $\frac{x_1 \in \Delta_1, \dots, x_{n+1} \in \Delta_{n+1}}{x_i \in \Delta_i}$ has the substitution property. So suppose that for each $j$, $1 \le j \le n+1$,
\[ \frac{Q}{s_j \in \Delta_j[s_1|x_1, \dots, s_{j-1}|x_{j-1}]} \]
is a derived rule of $U$. By lemmas 1 and 2, $x_{j+1}, \dots, x_{n+1}$ do not occur in $\Delta_j$. Hence $\Delta_j[s_1|x_1, \dots, s_{j-1}|x_{j-1}] = \Delta_j[s_1|x_1, \dots, s_{n+1}|x_{n+1}]$. Thus
\[ \frac{Q}{s_i \in \Delta_i[s_1|x_1, \dots, s_{n+1}|x_{n+1}]} \]
is a derived rule of $U$. Which is to say $\frac{Q}{(x_i \in \Delta_i)[s_1|x_1, \dots, s_{n+1}|x_{n+1}]}$ is a derived rule of $U$. Thus $\frac{x_1 \in \Delta_1, \dots, x_{n+1} \in \Delta_{n+1}}{x_i \in \Delta_i}$ has the substitution property.

\underline{CF2(a).} (CF2(b) is very similar and we shall not bother checking it). Suppose that $A$ is a sort symbol of $U$ introduced by the rule $\frac{x_1 \in \Delta_1, \dots, x_n \in \Delta_n}{A(x_1, \dots, x_n) \text{ is a type}}$. Suppose that for each $i$, $1 \le i \le n$,
\[ \frac{y_1 \in \Omega_1, \dots, y_m \in \Omega_m}{t_i \in \Delta_i[t_1|x_1, \dots, t_{i-1}|x_{i-1}]} \]
is a derived rule of $U$ and has the substitution property. We must show that $\frac{y_1 \in \Omega_1, \dots, y_m \in \Omega_m}{A(t_1, \dots, t_n) \text{ is a type}}$ has the substitution property. So suppose that for each $j$, $1 \le j \le m$, the rule
\[ \frac{Q}{s_j \in \Omega_j[s_1|y_1, \dots, s_{j-1}|y_{j-1}]} \]
is a derived rule of $U$. Then because for each $i$, $1 \le i \le n$,
\[ \frac{y_1 \in \Omega_1, \dots, y_m \in \Omega_m}{t_i \in \Delta_i[t_1|x_1, \dots, t_{i-1}|x_{i-1}]} \]
has the substitution property and because
\[ \Delta_i[t_1|x_1, \dots, t_{i-1}|x_{i-1}][s_1|y_1, \dots, s_m|y_m] = \Delta_i[t_1[s_1|y_1, \dots, s_m|y_m]|x_1, \dots, t_{i-1}[s_1|y_1, \dots, s_m|y_m]|x_{i-1}], \]
so it is the case that for each $i$, $1 \le i \le n$,
\[ \frac{Q}{t_i[s_1|y_1, \dots, s_m|y_m] \in \Delta_i[t_1[s_1|y_1, \dots, s_m|y_m]|x_1, \dots, t_{i-1}[s_1|y_1, \dots, s_m|y_m]|x_{i-1}]} \]
is a derived rule of $U$. Thus, by an application of CF2(a), the rule
\[ \frac{Q}{A(t_1[s_1|y_1, \dots, s_m|y_m], \dots, t_n[s_1|y_1, \dots, s_m|y_m]) \text{ is a type}} \]
is a derived rule of $U$. Which is to say that the rule
\[ \frac{Q}{A(t_1, \dots, t_n)[s_1|y_1, \dots, s_m|y_m] \text{ is a type}} \]
is a derived rule of $U$. Thus $\frac{y_1 \in \Omega_1, \dots, y_m \in \Omega_m}{A(t_1, \dots, t_n) \text{ is a type}}$ has the substitution property.
\end{proof}

\begin{corollary}[Change of Variables]
If $\frac{x_1 \in \Delta_1, \dots, x_n \in \Delta_n}{\text{Conclusion}}$ is a derived rule of $U$ and if $y_1, \dots, y_n$ is a sequence of distinct variables then
\[ \frac{y_1 \in \Delta_1, y_2 \in \Delta_2[y_1|x_1], \dots, y_n \in \Delta_n[y_1|x_1, \dots, y_{n-1}|x_{n-1}]}{\text{Conclusion}[y_1|x_1, \dots, y_n|x_n]} \]
is a derived rule of $U$.
\end{corollary}

\begin{proof}
The proof is by induction on $n$. If $n=0$ then there is nothing to prove. If the result holds for $n$ then it holds for $n+1$ as follows. Suppose $\frac{x_1 \in \Delta_1, \dots, x_{n+1} \in \Delta_{n+1}}{\text{Conclusion}}$ is a derived rule of $U$ and suppose that $y_1, \dots, y_{n+1}$ is a sequence of distinct variables. Then by lemma 2 the rule $\frac{x_1 \in \Delta_1, \dots, x_n \in \Delta_n}{\Delta_{n+1} \text{ is a type}}$ is a derived rule of $U$. Hence by the inductive hypothesis so is the rule
\[ \frac{y_1 \in \Delta_1, \dots, y_n \in \Delta_n[y_1|x_1, \dots, y_{n-1}|x_{n-1}]}{\Delta_{n+1}[y_1|x_1, \dots, y_n|x_n] \text{ is a type}}. \]
By applying CF1, the rule
\[ \frac{y_1 \in \Delta_1, \dots, y_{n+1} \in \Delta_{n+1}[y_1|x_1, \dots, y_n|x_n]}{y_i \in \Delta_i[y_1|x_1, \dots, y_{i-1}|x_{i-1}]} \]
is a derived rule of $U$, for each $i$, $1 \le i \le n+1$. Therefore by the substitution lemma and since $\frac{x_1 \in \Delta_1, \dots, x_{n+1} \in \Delta_{n+1}}{\text{Conclusion}}$ is a derived rule of $U$ we can conclude that
\[ \frac{y_1 \in \Delta_1, \dots, y_{n+1} \in \Delta_{n+1}[y_1|x_1, \dots, y_n|x_n]}{\text{Conclusion}[y_1|x_1, \dots, y_{n+1}|x_{n+1}]} \]
is a derived rule of $U$. The result holds for $n+1$. Hence the result holds for all $n \ge 0$.
\end{proof}

\begin{lemma}
Every derived rule of a theory is wellformed.
\end{lemma}

\begin{proof}
By induction on derivations in the theory $U$. We check each principle in turn, showing that all rules derived from wellformed rules are wellformed. LI1-LI7 and T1 are very easy to check.

\underline{CF1.} We must show that if $\frac{x_1 \in \Delta_1, \dots, x_n \in \Delta_n}{\Delta_{n+1} \text{ is a type}}$ is a wellformed derived rule of $U$ and if $x_{n+1}$ is a variable distinct from $x_1, \dots, x_n$ then for each $i$, $1 \le i \le n$, the rule $\frac{x_1 \in \Delta_1, \dots, x_{n+1} \in \Delta_{n+1}}{x_i \in \Delta_i}$ is wellformed. That is we must show that $\frac{x_1 \in \Delta_1, \dots, x_{n+1} \in \Delta_{n+1}}{\Delta_i \text{ is a type}}$ is a derived rule of $U$. This is the case because, as above, for each $j$, $1 \le j \le i$, the rule $\frac{x_1 \in \Delta_1, \dots, x_{n+1} \in \Delta_{n+1}}{x_j \in \Delta_j}$ is derivable and because by lemma 2(ii) the rule $\frac{x_1 \in \Delta_1, \dots, x_{i-1} \in \Delta_{i-1}}{\Delta_i \text{ is a type}}$ is a derived rule. Using the substitution lemma, the rule $\frac{x_1 \in \Delta_1, \dots, x_{n+1} \in \Delta_{n+1}}{\Delta_i \text{ is a type}}$ is a derived rule.

\underline{CF2(a).} Follows immediately from lemma 2.
\underline{CF2(b), SI1 and SI2.} Follow immediately from the substitution lemma.
\underline{A1 and A2.} These state that an axiom is a derived rule only if it is wellformed.
\end{proof}

\begin{lemma}[The Derivation Lemma]
(a) Every derived T-rule of the theory $U$ is of the form
\[ \frac{y_1 \in \Omega_1, \dots, y_m \in \Omega_m}{A(t_1, \dots, t_n) \text{ is a type}} \]
for some sort symbol $A$ of $U$ with introductory rule of the form $\frac{x_1 \in \Delta_1, \dots, x_n \in \Delta_n}{A(x_1, \dots, x_n) \text{ is a type}}$ and for some expressions $t_1, \dots, t_n$ such that for each $i$, $1 \le i \le n$, the rule
\[ \frac{y_1 \in \Omega_1, \dots, y_m \in \Omega_m}{t_i \in \Delta_i[t_1|x_1, \dots, t_{i-1}|x_{i-1}]} \]
is a derived rule of $U$.

(b) Every derived $\in$-rule of $U$ is either of the form
\[ \frac{y_1 \in \Omega_1, \dots, y_m \in \Omega_m}{y_j \in \Omega} \]
for some $j$, $1 \le j \le m$, and for some $\Lambda$ such that $\frac{y_1 \in \Omega_1, \dots, y_m \in \Omega_m}{\Omega_j = \Lambda}$ is a derived rule of $U$, or else is of the form
\[ \frac{y_1 \in \Omega_1, \dots, y_m \in \Omega_m}{f(t_1, \dots, t_n) \in \Omega} \]
for some operator symbol $f$ of $U$ with introductory rule of the form $\frac{x_1 \in \Delta_1, \dots, x_n \in \Delta_n}{f(x_1, \dots, x_n) \in \Delta}$ and for some expressions $t_1, \dots, t_n$, such that for each $i$, $1 \le i \le n$,
\[ \frac{y_1 \in \Omega_1, \dots, y_m \in \Omega_m}{t_i \in \Delta_i[t_1|x_1, \dots, t_{i-1}|x_{i-1}]} \]
is a derived rule of $U$ and such that $\frac{y_1 \in \Omega_1, \dots, y_m \in \Omega_m}{\Delta[t_1|x_1, \dots, t_n|x_n] = \Omega}$ is a derived rule of $U$.
\end{lemma}

\begin{proof}
(a) Simply because the only principle of derivation that enables us to derive T-rules is principle CF2(a).
(b) The principles which allow us to derive $\in$-rules are principles T1, CF1 and CF2(b). If an $\in$-rule is derived by CF1 then it is immediately of the first of the two forms stated above, if it is derived by CF2(b) then it is immediately of the second form. It remains to consider the case of an $\in$-rule derived by T1.

First suppose that a rule $\frac{P}{t \in \Omega}$ is derived by T1 from $\frac{P}{t \in \Omega'}$ and $\frac{P}{\Omega = \Omega'}$ and also suppose that the $\in$-rule $\frac{P}{t \in \Omega'}$ is derived by T1 from some $\frac{P}{t \in \Omega''}$ and $\frac{P}{\Omega' = \Omega''}$. In this situation the rule $\frac{P}{t \in \Omega}$ could have been derived directly by T1 from $\frac{P}{t \in \Omega''}$ and $\frac{P}{\Omega = \Omega''}$, thus missing out a double application of T1. It follows that if a rule $\frac{P}{t \in \Omega}$ is derivable by an application of T1 then it is derivable by an application of T1 to some rules $\frac{P}{t \in \Omega'}$ and $\frac{P}{\Omega = \Omega'}$ such that the rule $\frac{P}{t \in \Omega'}$ is derivable by CF1 or CF2(a). It then follows that $\frac{P}{t \in \Omega}$ is of one of the two forms stated above.
\end{proof}

\begin{corollary}
If $\frac{P}{t \in \Omega}$ and $\frac{P}{t \in \Omega'}$ are both derived rules of $U$ then so to is $\frac{P}{\Omega = \Omega'}$.
\end{corollary}

The next lemma might indicate an alternative inductive definition of the notion of generalised algebraic theory.

If we say that $U'$ is a theory extending $U$ then it is meant that all the introductory rules and axioms of $U$ are included among the introductory rules and axioms of $U'$. In particular every symbol of $U$ is a symbol of $U'$.

An extension $U'$ of $U$ is said to be a \underline{simple extension} of $U$ iff all of the introductory rules and axioms of $U'$ are well formed wrt $U$. For example, the rule $\frac{x_1 \in \Delta_1, \dots, x_n \in \Delta_n}{\Delta \text{ is a type}}$ of $U'$ is wellformed wrt $U$ iff $x_1 \in \Delta_1, \dots, x_n \in \Delta_n$ is a context of $U$.

\begin{lemma}
If $U'$ is a theory extending the theory $U$ then there exists a sequence of theories $U_0, U_1, U_2, \dots$ such that for each $i \ge 0$, $U_{i+1}$ is a simple extension of $U_i$ and such that $U_0 = U$ and $\bigcup_{i \ge 0} U_i = U'$.
\end{lemma}

\begin{proof}
$U_0$ is defined to be $U$. $U_{i+1}$ is defined to be the simple extension of $U_i$ given by all those symbols of $U'$ whose introductory rules are wellformed wrt $U_i$ and all those axioms of $U'$ which are wellformed wrt $U_i$. The only problem is to show that every symbol and axiom of $U'$ is eventually in $U_i$ for some $i$. We just have to show that every introductory rule and axiom of $U'$ is wellformed wrt $U_i$ for some $i$.

Suppose then that $R$ is an introductory rule or an axiom of $U'$. Because $R$ is wellformed wrt $U'$ it must be wellformed wrt some finite number $k$ of introductory rules and axioms of $U'$. We show by induction on $k$ that $R$ is wellformed wrt $U_k$.

If $k=0$ then $R$ is wellformed wrt $U=U_0$.

If $k>0$, suppose $S$ is one of those $k$ introductory rules and axioms from which $R$ can be shown to be wellformed. In any derivation we only use an axiom or an introductory rule after it has been shown to be well formed; in particular since $S$ is used in showing that $R$ is wellformed there must be rules capable of showing that $S$ is wellformed among the $k$ rules that can be used to show $R$ is wellformed. Thus it can be shown that $S$ is wellformed from some number $p$ of rules and axioms of $U'$ where $p$ is strictly smaller than $k$. By the inductive hypothesis $S$ is wellformed wrt $U_p$. Thus $S$ is an introductory rule or an axiom of $U_k$. This is the case for any of those $k$ introductory rules and axioms of $U'$ which can be used to show that $R$ is wellformed. Thus $R$ is wellformed wrt $U_k$.
\end{proof}

\begin{lemma}
If $\frac{x_1 \in \Delta_1, \dots, x_n \in \Delta_n}{\Delta_{n+1} \text{ is a type}}$ and $\frac{x_1 \in \Delta_1, \dots, x_n \in \Delta_n, y_1 \in \Omega_1, \dots, y_m \in \Omega_m}{\text{Conclusion}}$ are both derived rules of $U$, if $z$ is a variable distinct from $x_1, \dots, x_n, y_1, \dots, y_m$ then
\[ \frac{x_1 \in \Delta_1, \dots, x_n \in \Delta_n, z \in \Delta_{n+1}, y_1 \in \Omega_1, \dots, y_m \in \Omega_m}{\text{Conclusion}} \]
is a derived rule of $U$.
\end{lemma}

\begin{proof}
By induction on $m$. If $m=0$ then from $\frac{x_1 \in \Delta_1, \dots, x_n \in \Delta_n}{\Delta \text{ is a type}}$ we can derive $\frac{x_1 \in \Delta_1, \dots, x_n \in \Delta_n, z \in \Delta}{x_i \in \Delta_i}$ for each $i$, $1 \le i \le n$. Since $\frac{x_1 \in \Delta_1, \dots, x_n \in \Delta_n}{\text{Conclusion}}$ is a derived rule by the substitution lemma so too is $\frac{x_1 \in \Delta_1, \dots, x_n \in \Delta_n, z \in \Delta}{\text{Conclusion}}$.

If $m>0$. Then since $x_1 \in \Delta_1, \dots, x_n \in \Delta_n, y_1 \in \Omega_1, \dots, y_m \in \Omega_m$ is a context, $\frac{x_1 \in \Delta_1, \dots, x_n \in \Delta_n, y_1 \in \Omega_1, \dots, y_{m-1} \in \Omega_{m-1}}{\Omega_m \text{ is a type}}$ is a derived rule. Thus by the inductive hypothesis
\[ \frac{x_1 \in \Delta_1, \dots, x_n \in \Delta_n, z \in \Delta, y_1 \in \Omega_1, \dots, y_{m-1} \in \Omega_{m-1}}{\Omega_m \text{ is a type}} \]
is a derived rule. Thus $x_1 \in \Delta_1, \dots, x_n \in \Delta_n, z \in \Delta, y_1 \in \Omega_1, \dots, y_m \in \Omega_m$,
\[ \frac{x_1 \in \Delta_1, \dots, x_n \in \Delta_n, z \in \Delta, y_1 \in \Omega_1, \dots, y_m \in \Omega_m}{x_i \in \Delta_i}, \quad 1 \le i \le n, \]
and
\[ \frac{x_1 \in \Delta_1, \dots, x_n \in \Delta_n, z \in \Delta, y_1 \in \Omega_1, \dots, y_m \in \Omega_m}{y_j \in \Omega_j}, \quad 1 \le j \le m, \]
are derived rules. By the substitution lemma
\[ \frac{x_1 \in \Delta_1, \dots, x_n \in \Delta_n, z \in \Delta, y_1 \in \Omega_1, \dots, y_m \in \Omega_m}{\text{Conclusion}} \]
is a derived rule.
\end{proof}

\section{Informal Syntax}

There is a discrepancy between the syntax adopted in the formal definition of \S 1.6 and the syntax used in informally presenting theories in other sections. We say that we have a formal syntax and an informal syntax. The informal syntax is the syntax that is used in practice. In a particular case it provides an adequate and unambiguous language for the description of the structure involved. Since the informal syntax varies non uniformly from one particular theory to another it is impossible to give a direct description of the informal syntax. In this section we classify the discrepancies that occur between formal and informal. We also state a general problem and provide a partial solution. In this section we have in mind a very practical approach to mathematical syntax.

In the first place, the actual forms of the rules are of no consequence. Thus the form that is used in the formal syntax, that is $\frac{x_1 \in \Delta_1, \dots, x_n \in \Delta_n}{\text{Conclusion}}$, has the advantage of alienation and is not significantly different from any of the other forms which have to some extent the advantage of naturality, forms such as
\begin{itemize}
    \item $x_1 \in \Delta_1, \dots, x_n \in \Delta_n : \text{Conclusion}$,
    \item for $x_1 \in \Delta_1$, for $x_2 \in \Delta_2, \dots$ and for $x_n \in \Delta_n : \text{Conclusion}$,
    \item Conclusion, whenever $x_1 \in \Delta_1, x_2 \in \Delta_2, \dots$ and $x_n \in \Delta_n$,
\end{itemize}
and such that, wherever possible, repetitions of expressions in the premise are avoided by writing $x_k, x_{k+1}, \dots, x_{k+c} \in \Delta$ instead of $x_k \in \Delta, \dots, x_{k+c} \in \Delta$.

There are two significant differences between the formal and the informal. The first of these is the omission of some of the variables which would formally have to appear in the term that introduces a symbol into a theory, and the subsequent omission of terms from the argument places of the symbol as it appears in the derived rules of the theory. Thus in the theory of categories the term $o(f,g)$ occurs in the introductory rule for $o$ instead of the term $o(x,y,z,f,g)$; the axioms differ accordingly.

The second difference between formal and informal is that informally one symbol can be made to do the work that several symbols would have to do formally. As examples of symbols that have to do the work of several we have the symbol $S$ of the theory of trees and the symbols $id, o, Hom$ and $F$ of the theory of functors (with reference to the presentations of these theories in \S 1.2).

If we arbitrarily rewrite a formal theory by these two methods, that is if we omit certain variables from certain introductory rules, altering the derived rules accordingly, and if we replace certain collections of symbols by single symbols, then ambiguities may or may not arise. There is ambiguity just when two formally distinct derived rules are rewritten as identical, for this would mean that there was an informal rule which had two meanings, was ambiguous. So the problem is -- in what ways can we rewrite a given formal theory without ambiguities arising? The answer is that it depends on the theory in question. The best general answer that we can give consists of a condition that the omission of variables must respect if ambiguities are not to arise. This condition objectifies the dropping of the variables $x,y,z$ from the term $o(x,y,z,f,g)$ in the introductory rule for $o$ and the wrongheadedness of dropping $f$ or $g$ or both from this same term. Intuitively, $o(f,g)$ depends explicitly on $f$ and $g$, $f$ depends explicitly on $x$ and $y$ because $f \in Hom(x,y)$, $g$ depends explicitly on $y$ and on $z$ because $g \in Hom(y,z)$, thus $o(f,g)$ depends implicitly on all of the variables $x,y,z,f$ and $g$ that occur in the premise of the introductory rule, so that although the variables $x,y$ and $z$ no longer appear explicitly in $o(f,g)$ there is still an implicit dependence of $o(f,g)$ on each of $x,y$, and $z$. The condition on an introductory rule which is necessary if ambiguities are not to arise is that all variables occurring in the premise must occur implicitly in the conclusion. This condition we can call the condition of implicit occurrence. It is necessary but not sufficient, as we shall show.

The definition of implicit occurrence must be given inductively. Suppose that $P$ is the premise $x_1 \in \Delta_1, \dots, x_n \in \Delta_n$, suppose $C$ is a conclusion and that $1 \le i \le n$, then we say that the variable $x_i$ \underline{occurs implicitly} in the conclusion $C$ wrt the premise $P$ iff either $x_i$ actually occurs in $C$ (in which case we also say $x_i$ occurs explicitly) or if for some $j > i$, $x_j$ appears in $\Delta_j$ and $x_j$ occurs implicitly in $C$ wrt $P$.

\begin{lemma}
If ambiguity is not to occur in an informal theory then whenever $L$ is a symbol introduced by the rule $\frac{x_1 \in \Delta_1, \dots, x_n \in \Delta_n}{C}$ then each of $x_1, \dots, x_n$ must occur implicitly in $C$ (the condition of implicit occurrence).
\end{lemma}

\begin{proof}
Assume that we have a theory and an informal presentation of that theory in which there is a symbol $L$ introduced by a rule that does not satisfy the condition of implicit occurrence. We shall suppose that $L$ is an operator symbol for if otherwise and $L$ is a sort symbol then the argument is the same. Suppose that $n \ge 1$ and that $1 \le j_1 < j_2 \dots < j_r \le n$, suppose that $L$ is introduced by the rule $x_1 \in \Delta_1, \dots, x_n \in \Delta_n : L(x_{j_1}, \dots, x_{j_r}) \in \Delta$. We are assuming that not all of $x_1, \dots, x_n$ occur implicitly in $L(x_{j_1}, \dots, x_{j_r}) \in \Delta$ wrt the premise $x_1 \in \Delta_1, \dots, x_n \in \Delta_n$. We show that there are two derived rules which are distinct in the more formal syntax, that is when $L$ is introduced by $x_1 \in \Delta_1, \dots, x_n \in \Delta_n : L(x_1, \dots, x_n) \in \Delta$, but which are indistinct in the informal syntax.

Let $x'_1, \dots, x'_n$ be a sequence of variables each one of which is distinct from each one of $x_1, \dots, x_n$. Let $J = \{ j \mid 1 \le j \le n \text{ and } x_j \text{ does not occur implicitly in } L(x_{j_1}, \dots, x_{j_r}) \in \Delta \}$. Let $y_1, \dots, y_n$ be the sequence of variables given by $y_j = x_j$ if $j \in J$, $y_j = x'_j$ if $j \notin J$.

Suppose that $j \notin J$ and $j' \in J$, then $x_j$ occurs implicitly in $L(x_{j_1}, \dots, x_{j_r}) \in \Delta$ whereas $x_{j'}$ does not, hence $x_{j'}$ does not occur in $\Delta_j$. Thus if $j \notin J$ then $\Delta_j[y_1|x_1, \dots, y_{j-1}|x_{j-1}]$ is $\Delta_j$. Also note that $\Delta[y_1|x_1, \dots, y_n|x_n]$ is $\Delta$, since if $j \in J$ then $x_j$ does not occur in $\Delta$.

By the change of variables lemma the rule
\[ y_1 \in \Delta_1, \dots, y_n \in \Delta_n[y_1|x_1, \dots, y_{n-1}|x_{n-1}] : L(y_1, \dots, y_n) \in \Delta[y_1|x_1, \dots, y_n|x_n] \]
is a derived rule. By the preceding paragraph this rule is just the rule $a_1, \dots, a_n : L(y_1, \dots, y_n) \in \Delta$, where $a_j$ is $x_j \in \Delta_j$ when $j \notin J$ and $a_j$ is $x_j \in \Delta_j[y_1|x_1, \dots, y_{j-1}|x_{j-1}]$ when $j \in J$. By lemma 4 of \S 1.7 if we extend the premise of this rule by inserting the clause $x_j \in \Delta_j$ after the clause $a_j$, whenever $j \in J$ then the new rule is still derivable. If we call this extended premise $Q$ then the rule
\[ Q : L(y_1, \dots, y_n) \in \Delta \]
is a derived rule.

Since $Q$ is a context extending the context $x_1 \in \Delta_1, \dots, x_n \in \Delta_n$, and using lemma 4 of \S 1.7 we deduce that $Q : L(x_1, \dots, x_n) \in \Delta$ is a derived rule.

Now $Q : L(y_1, \dots, y_n) \in \Delta$ and $Q : L(x_1, \dots, x_n) \in \Delta$ are distinct since we have assumed $J$ to be non empty. In the informal syntax both these rules become $Q : L(x_{j_1}, \dots, x_{j_r}) \in \Delta$. Hence informally the rule $Q : L(x_{j_1}, \dots, x_{j_r}) \in \Delta$ is ambiguous.
\end{proof}

Finally we show that the condition of implicit occurrence is not sufficient to assure that ambiguities do not arise when variables are omitted.

Consider the following theory:

\textbf{Symbol} \quad \textbf{Introductory Rule}
\begin{itemize}
    \item $A$: \quad $A$ is a type.
    \item $B$: \quad For $x \in A : B(x)$ is a type.
    \item $C$: \quad For $x \in A$, for $y \in B(x) : C(x,y)$ is a type.
    \item $D$: \quad For $y \in B(a_1)$, for $z \in C(a_2, y) : D(y,z)$ is a type.
    \item $a_1$: \quad $a_1 \in A$.
    \item $a_2$: \quad $a_2 \in A$.
\end{itemize}

\textbf{Axioms.}
\begin{itemize}
    \item $B(a_1) = B(a_2)$.
\end{itemize}

See that in this theory the rules $y \in B(a_1) : C(a_1, y)$ is a type and $y \in B(a_1) : C(a_2, y)$ is a type are both derivable.

The theory might be rewritten informally by introducing the symbol $C$ by the rule for $x \in A$, for $y \in B(x) : C(y)$ is a type, clearly the condition of implicit occurrence is respected. However the two rules $y \in B(a_1) : C(a_1, y)$ is a type and $y \in B(a_1) : C(a_2, y)$ is a type are rewritten as the rule $y \in B(a_1) : C(y)$ is a type in this informal syntax. Thus though the condition of implicit occurrence is respected ambiguity still arises.

We have included the symbol $D$ in the theory to illustrate that an ambiguous rule in a theory easily leads to an ambiguous theory. Informally the symbol $D$ is now introduced by the rule for $y \in B(a_1)$, for $z \in C(y) : D(y,z)$ is a type; the formal theory can no longer be recovered from its informal presentation since this presentation could equally well be the informal presentation of the theory that differs from the given theory in that the symbol $D$ is introduced by the rule
\[ \text{for } y \in B(a_1), \text{for } z \in C(a_2, y) : D(y,z) \text{ is a type.} \]

\section{Models and Homomorphisms}

We neglect the formal definitions of model and homomorphism. It should be quite clear what we mean by model. We have a few words to say about homomorphisms but first we wish to develop some notations which relate to the matter but are actually of more importance in the next chapter.

We begin rather distantly with trees. The theory of trees was given in \S 1.2 but from now on we want all our trees to be trees with a unique least element. If we come across a tree which does not have a unique least element then we quickly adjoin a new element 1 beneath all the other elements. If $\Theta$ is a tree and if $A$ is a node of the tree then we say that $A \in \Theta$, so confusing the tree with its set of nodes. If we wish to assert that $B$ is a node of the tree $\Theta$ and that $B$ succeeds $A$ then we just say that $A \triangleleft B$ in $\Theta$. The least node of $\Theta$ is always denoted $1$. Thus if $A$ is any node of the tree $\Theta$ distinct from $1$ then there exists a unique $n \ge 0$, there exists uniquely $A_1, \dots, A_n$ such that $1 \triangleleft A_1 \triangleleft A_2 \dots \triangleleft A_n \triangleleft A$ in $\Theta$.

We are interested in trees because for any theory $U$, the set of contexts of $U$ is structured as a tree. The least element of the tree is the empty context $\langle \rangle$. For any $n \ge 1$ the predecessor of the node $\langle x_1 \in \Delta_1, \dots, x_n \in \Delta_n \rangle$ is the node $\langle x_1 \in \Delta_1, \dots, x_{n-1} \in \Delta_{n-1} \rangle$.

We wish to identify a large tree of sets, families of sets, families of families of sets and so on. It looks as if we should call it the family tree. But we won't. Anyway it is first necessary to consider the notation that we use for families.

If $A$ is a set and if for every element $a \in A$, $B(a)$ is a set, then the corresponding $A$-indexed family of sets is denoted $\lambda a \in A . B(a)$ or just as $\lambda a . B(a)$. Now suppose that in this situation we also have a set $C(a,b)$ for every $a \in A$ and for every $b \in B(a)$. Now if $a$ is an element of $A$ then $\lambda b \in B(a) . C(a,b)$ is a $B(a)$ indexed family of sets. Thus $\lambda a \in A . \lambda b \in B(a) . C(a,b)$ is an $A$-indexed family of families of sets. We can continue in this way. The whole collection of sets, families of sets, families of families of sets and so on is structured as a (large) tree. We call it the tree of families. The next thing to do is to turn the notation about. If $A_1$ is a set and if $A_2$ is an $A_1$-indexed family of sets then we write $A_2(a_1)$ for the value of the family at an element $a_1 \in A_1$. We do the same for families of families and so on. In general if $1 \triangleleft A_1 \triangleleft A_2 \dots \triangleleft A_n \triangleleft A$ in the tree of families then for any $a_1 \in A_1, \dots,$ for any $a_n \in A_n(a_1, \dots, a_{n-1})$, $A(a_1, \dots, a_n)$ is a set.

Lastly we wish to be precise about the term operator. If $1 \triangleleft A_1 \dots \triangleleft A_n \triangleleft A$ in the tree of families and if for any $a_1 \in A_1, \dots,$ for any $a_n \in A_n(a_1, \dots, a_{n-1})$, $f(a_1, \dots, a_n) \in A(a_1, \dots, a_n)$ then we say that $\lambda a_1 \in A_1 \dots \lambda a_n \in A_n . f(a_1, \dots, a_n)$ is an operator at $A$. Thus for any node $A$ of the tree of families there is a set of operators at $A$. If we turn the notation about then we always write $g(a_1, \dots, a_n)$ for the value of the operator $g$ at arguments $a_1, \dots, a_n$. If $1 \triangleleft A_1 \dots A_n \triangleleft A$ in the tree of families, if $g$ is an operator at $A$ then the status of the operator $g$ is given by the rule for every $a_1 \in A_1, \dots$ for every $a_n \in A_n(a_1, \dots, a_{n-1})$, $g(a_1, \dots, a_n) \in A(a_1, \dots, a_n)$. For example if $\mathcal{C}$ is a real live category then $id$ is an operator whose status is given by for every $a \in |\mathcal{C}|$, $id(a) \in Hom_{\mathcal{C}}(a,a)$. Alternatively $id$ is an operator at $\lambda a \in |\mathcal{C}| . Hom_{\mathcal{C}}(a,a)$.

Now we return to the question of models of a theory. In the first place it is the derived rules of a theory that are interpreted in a model, not the expressions. If $U$ is a theory with model $M$ and if $R$ is a derived T-rule of the theory, say $\frac{x_1 \in \Delta_1, \dots, x_n \in \Delta_n}{\Delta \text{ is a type}}$, if we let $R_i$ be the rule $\frac{x_1 \in \Delta_1, \dots, x_{i-1} \in \Delta_{i-1}}{\Delta_i \text{ is a type}}$, whenever $1 \le i \le n$, if the interpretation of a rule within the model $M$ is written as that rule superscripted by $m$. Then $R_1^m$ is a set, $R_2^m$ is an $R_1^m$-indexed family of sets and in general $1 \triangleleft R_1^m \triangleleft R_2^m \dots \triangleleft R_n^m \triangleleft R^m$ in the tree of families.

Further if $R_t$ is a derived rule of $U$ of the form $\frac{x_1 \in \Delta_1, \dots, x_n \in \Delta_n}{t \in \Delta}$ then the interpretation $R_t^m$ of the rule $R_t$ by the model $M$ is an operator at $R^m$.

We note that if $M$ and $M'$ are both models of $U$ and if $f : M \to M'$ is a homomorphism then for every derived rule $R$ of $U$ of the form $\frac{x_1 \in \Delta_1, \dots, x_n \in \Delta_n}{\Delta \text{ is a type}}$ there is an operator $f_R$ whose status is given by for $a_1 \in R_1^m, \dots,$ for $a_n \in R_n^m(a_1, \dots, a_{n-1})$, for $a \in R^m(a_1, \dots, a_n)$, $f_R(a_1, \dots, a_n, a) \in R^{m'}(f_{R_1}(a_1), \dots, f_{R_n}(a_1, \dots, a_n), f_R(a_1, \dots, a_n, a))$. It is required that for all derived rules $R_t$ of the form $\frac{x_1 \in \Delta_1, \dots, x_n \in \Delta_n}{t \in \Delta}$ it is the case that for all $a_1 \in R_1^m, \dots$ for all $a_n \in R_n^m(a_1, \dots, a_{n-1})$,
\[ f_R(a_1, \dots, a_n, R_t^m(a_1, \dots, a_n)) = R_t^{m'}(f_{R_1}(a_1), \dots, f_{R_n}(a_1, \dots, a_n)). \]
In fact if this condition holds for all $F(x_1, \dots, x_n)$, $F$ an operator symbol, then it will hold for all $t$. The only requirement of a homomorphism that might appear unusual is the requirement that whenever $\frac{x_1 \in \Delta_1, \dots, x_n \in \Delta_n}{\Delta = \Delta'}$ is a derived rule of $U$ then $f_R = f_{R'}$ where $R$ is the rule $\frac{x_1 \in \Delta_1, \dots, x_n \in \Delta_n}{\Delta \text{ is a type}}$ and where $R'$ is the rule $\frac{x_1 \in \Delta_1, \dots, x_n \in \Delta_n}{\Delta' \text{ is a type}}$.

In the definition of homomorphism we just ask for a family $\lambda L \in \{\text{sort symbols of } U\} . f_L$ of operators of the correct status and then define the $f_R$ by induction (in this notation $f_{\text{introductory rule of } L} = f_L$). And we require just that the two conditions mentioned above are satisfied by the $f_R$.

With homomorphism so defined the notion reduces to the usual model theoretic notion in the special case of the universal conditions theories expressed as generalised algebraic as in \S 1.3.

\section{A Short list of theories}

We list some generalised algebraic theories and pay particular attention to the sort structures.

Firstly, extending the theory of categories by operator symbols and axioms alone (thus there are no additional sort symbols) are the theories of categories with finite products, cartesian closed categories, categories with finite coproducts, monoidal categories, closed categories, additive categories, $U$-categories for a given algebraic theory $U$, groupoids, preorders, partial orders, lattices and monads. Extending the theory of categories by just the equality predicate for morphisms, new operator symbols and new axioms are the theories of categories with equalisers of pairs, categories with finite limits, abelian categories and topoi.

Extending the theory of functors by just operator symbols and axioms are the theories of natural transformations, adjoint pairs and equivalences. In more diverse sort structures we have the theories of $n$-categories for fixed $n$, multicategories, category valued presheaves on an arbitrary category, category valued presheaves on a given category. The theory of hyperdoctrines extends the theory of category valued presheaves by operator symbols and axioms alone. Later we shall come across the theory of contextual categories, a theory which extends the theory of trees.

\section{Interpretations}

The notion of an interpretation of one generalised algebraic theory in another is defined in such a way that there is a category of generalised algebraic theories and interpretations. It is worth noting of the definition that the induction that occurs is on the expressions of a language rather than on the derived rules and that as such the construction is very simple.

The alphabet in which a theory $U$ is written we denote $A_U$.

We assume throughout some fixed enumeration $v_1, v_2, v_3, \dots$ of the set $V$ of variables. We do this because symbols $L \in A_U$ are going to be interpreted by expressions $I(L)$ of a theory $U'$. We will wish to know which free variables in the expression $I(L)$ correspond to which argument places associated with $L$. The simplest way this can be done is to assume the enumeration of $V$ and then chose $I(L)$ such that $v_1$ corresponds to the first argument place of $L$, $v_2$ to the second and so on.

We first define the notion of a preinterpretation and then go on to eventually define an interpretation to be a well formed preinterpretation.

\underline{Definitions}

A \underline{Preinterpretation} $I$ of the theory $U$ in the theory $U'$ consists just of a function $I : A_U \to \text{Expressions of } U'$.

If $I$ is a preinterpretation of $U$ in $U'$ then define the function $\dot{I} : \text{Expressions of } U \to \text{Expressions of } U'$ by induction (see the inductive definition of expression in \S 1.6) with the following clauses:
1. If $x \in V$ then $\dot{I}(x) = x$.
2. If $L \in A_U$ then $\dot{I}(L) = I(L)$.
3. If $L \in A_U$ and $e_1, \dots, e_n$ are expressions then $\dot{I}(L(e_1, \dots, e_n)) = I(L)[\dot{I}(e_1)|v_1, \dots, \dot{I}(e_n)|v_n]$.

If $I$ is a preinterpretation of $U$ in $U'$ then define a function $\hat{I} : \text{Rules of } U \to \text{Rules of } U'$ by:

1. $\hat{I}\left(\frac{x_1 \in \Delta_1, \dots, x_n \in \Delta_n}{\Delta \text{ is a type}}\right) = \frac{x_1 \in \dot{I}(\Delta_1), \dots, x_n \in \dot{I}(\Delta_n)}{\dot{I}(\Delta) \text{ is a type}}$

2. $\hat{I}\left(\frac{x_1 \in \Delta_1, \dots, x_n \in \Delta_n}{t \in \Delta}\right) = \frac{x_1 \in \dot{I}(\Delta_1), \dots, x_n \in \dot{I}(\Delta_n)}{\dot{I}(t) \in \dot{I}(\Delta)}$

3. $\hat{I}\left(\frac{x_1 \in \Delta_1, \dots, x_n \in \Delta_n}{\Delta = \Delta'}\right) = \frac{x_1 \in \dot{I}(\Delta_1), \dots, x_n \in \dot{I}(\Delta_n)}{\dot{I}(\Delta) = \dot{I}(\Delta')}$

4. $\hat{I}\left(\frac{x_1 \in \Delta_1, \dots, x_n \in \Delta_n}{t = t' \in \Delta}\right) = \frac{x_1 \in \dot{I}(\Delta_1), \dots, x_n \in \dot{I}(\Delta_n)}{\dot{I}(t) = \dot{I}(t') \in \dot{I}(\Delta)}$

An \underline{interpretation} $I$ of $U$ in $U'$ is a preinterpretation $I$ of $U$ in $U'$ such that for all introductory rules and axioms $r$ of $U$, $\hat{I}(r)$ is a derived rule of $U'$.

If $I$ is an interpretation of $U$ in $U'$ then for any derived rule $r$ of $U$, $\hat{I}(r)$ is a derived rule of $U'$. We prove this after proving a preliminary lemma.

\begin{lemma}
If $I$ is a preinterpretation of $U$ in $U'$ and if $e$ and $d_1, \dots, d_m$ are expressions of $A_U$ then
\[ \dot{I}(e[d_1|y_1, \dots, d_m|y_m]) = \dot{I}(e)[\dot{I}(d_1)|y_1, \dots, \dot{I}(d_m)|y_m]. \]
\end{lemma}

\begin{proof}
By induction on the length of the expression $e$.
1. If $e = x \in V$ then, if $x = y_i$ for some $i$, $1 \le i \le m$, then L.H.S = $\dot{I}(d_i)$ = R.H.S., otherwise L.H.S. = $x$ = R.H.S.
2. If $e = L \in A_U$ then L.H.S. = $I(L)$ = R.H.S.
3. If $e = L(e_1, \dots, e_n)$ for some $L \in A_U$ and for some expressions $e_1, \dots, e_n$ of $A_U$ then $\dot{I}(e[d_1|y_1, \dots, d_m|y_m]) = \dot{I}(L(e_1[d_1|y_1, \dots, d_m|y_m], \dots, e_n[d_1|y_1, \dots, d_m|y_m])) = I(L)[\dot{I}(e_1[d_1|y_1, \dots, d_m|y_m])|v_1, \dots, \dot{I}(e_n[d_1|y_1, \dots, d_m|y_m])|v_n]$.
By the inductive hypothesis $\dot{I}(e_i[d_1|y_1, \dots, d_m|y_m]) = \dot{I}(e_i)[\dot{I}(d_1)|y_1, \dots, \dot{I}(d_m)|y_m]$. Hence
$\dot{I}(e[d_1|y_1, \dots, d_m|y_m]) = I(L)[\dot{I}(e_1)[\dot{I}(d_1)|y_1, \dots, \dot{I}(d_m)|y_m]|v_1, \dots, \dot{I}(e_n)[\dot{I}(d_1)|y_1, \dots, \dot{I}(d_m)|y_m]|v_n] = I(L)[\dot{I}(e_1)|v_1, \dots, \dot{I}(e_n)|v_n][\dot{I}(d_1)|y_1, \dots, \dot{I}(d_m)|y_m] = \dot{I}(L(e_1, \dots, e_n))[\dot{I}(d_1)|y_1, \dots, \dot{I}(d_m)|y_m] = \dot{I}(e)[\dot{I}(d_1)|y_1, \dots, \dot{I}(d_m)|y_m]$. As required.
\end{proof}

\begin{lemma}
If $I$ is an interpretation of $U$ in $U'$ then for every derived rule $r$ of $U$, $\hat{I}(r)$ is a derived rule of $U'$.
\end{lemma}

\begin{proof}
We wish to show that all derived rules $r$ of $U$ have the property that $\hat{I}(r)$ is derivable. We are given that the axioms have this property so it suffices to show that the principles of derivation transmit the property. That is we should check that each principle of derivation when applied to rules with the property yields a rule with the property. Principles LI1-7, T1, and CF1 are incredibly easy to check. The principle CF2(b) is similar to the principle CF2(a) and also principle SI1 is similar to principle SI2. In view of this we just check CF2(a) and SI2.

\underline{CF2(a).} Suppose that $A$ is a sort symbol of $U$ introduced by the rule $\frac{x_1 \in \Delta_1, \dots, x_n \in \Delta_n}{A(x_1, \dots, x_n) \text{ is a type}}$, and that for each $i$, $1 \le i \le n$,
\[ \frac{y_1 \in \Omega_1, \dots, y_m \in \Omega_m}{t_i \in \Delta_i[t_1|x_1, \dots, t_{i-1}|x_{i-1}]} \]
is a derived rule of $U$. Also suppose that
\[ \hat{I}\left(\frac{y_1 \in \Omega_1, \dots, y_m \in \Omega_m}{t_i \in \Delta_i[t_1|x_1, \dots, t_{i-1}|x_{i-1}]}\right) \]
is a derived rule of $U'$. We wish to show that
\[ \hat{I}\left(\frac{y_1 \in \Omega_1, \dots, y_m \in \Omega_m}{A(t_1, \dots, t_n) \text{ is a type}}\right) \]
is a derived rule of $U'$.

Since $I$ is an interpretation, $\hat{I}$ (introductory rule for $A$) is a derived rule of $U'$. That is
\[ \frac{x_1 \in \dot{I}(\Delta_1), \dots, x_n \in \dot{I}(\Delta_n)}{\dot{I}(A(x_1, \dots, x_n)) \text{ is a type}} \]
is a derived rule of $U'$.

By Lemma 1,
\[ \hat{I}\left(\frac{y_1 \in \Omega_1, \dots, y_m \in \Omega_m}{t_i \in \Delta_i[t_1|x_1, \dots, t_{i-1}|x_{i-1}]}\right) = \frac{y_1 \in \dot{I}(\Omega_1), \dots, y_m \in \dot{I}(\Omega_m)}{\dot{I}(t_i) \in \dot{I}(\Delta_i)[\dot{I}(t_1)|x_1, \dots, \dot{I}(t_{i-1})|x_{i-1}]} \]
This rule is a derived rule for each $i$, $1 \le i \le n$. Hence by the substitution lemma the rule
\[ \frac{y_1 \in \dot{I}(\Omega_1), \dots, y_m \in \dot{I}(\Omega_m)}{\dot{I}(A(x_1, \dots, x_n))[\dot{I}(t_1)|x_1, \dots, \dot{I}(t_n)|x_n]} \]
is a derived rule of $U$.

But $\dot{I}(A(x_1, \dots, x_n)) = I(A)[x_1|v_1, \dots, x_n|v_n]$, hence $\dot{I}(A(x_1, \dots, x_n))[\dot{I}(t_1)|x_1, \dots, \dot{I}(t_n)|x_n] = I(A)[\dot{I}(t_1)|v_1, \dots, \dot{I}(t_n)|v_n] = \dot{I}(A(t_1, \dots, t_n))$. So we have shown that
\[ \hat{I}\left(\frac{y_1 \in \Omega_1, \dots, y_m \in \Omega_m}{A(t_1, \dots, t_n) \text{ is a type}}\right) \]
is a derived rule of $U'$, as $\hat{I}\left(\frac{y_1 \in \Omega_1, \dots, y_m \in \Omega_m}{A(t_1, \dots, t_n) \text{ is a type}}\right)$ is by definition
\[ \frac{y_1 \in \dot{I}(\Omega_1), \dots, y_m \in \dot{I}(\Omega_m)}{\dot{I}(A(t_1, \dots, t_n)) \text{ is a type}}. \]

\underline{SI2.} Suppose that $\frac{x_1 \in \Delta_1, \dots, x_n \in \Delta_n}{t = t' \in \Delta}$ is a derived rule of $U$ and that for each $i, 1 \le i \le n$, $\frac{y_1 \in \Omega_1, \dots, y_m \in \Omega_m}{t_i = t'_i \in \Delta_i[t_1|x_1, \dots, t_{i-1}|x_{i-1}]}$ is a derived rule of $U$. Suppose that $\hat{I}$ applied to each of the rules yields a derivable rule of $U'$.

By definition $\hat{I}\left(\frac{x_1 \in \Delta_1, \dots, x_n \in \Delta_n}{t = t' \in \Delta}\right) = \frac{x_1 \in \dot{I}(\Delta_1), \dots, x_n \in \dot{I}(\Delta_n)}{\dot{I}(t) = \dot{I}(t') \in \dot{I}(\Delta)}$, this latter rule then is a derived rule of $U'$.

By definition and Lemma 1,
\[ \hat{I}\left(\frac{y_1 \in \Omega_1, \dots, y_m \in \Omega_m}{t_i = t'_i \in \Delta_i[t_1|x_1, \dots, t_{i-1}|x_{i-1}]}\right) = \frac{y_1 \in \dot{I}(\Omega_1), \dots, y_m \in \dot{I}(\Omega_m)}{\dot{I}(t_i) = \dot{I}(t'_i) \in \dot{I}(\Delta_i)[\dot{I}(t_1)|x_1, \dots, \dot{I}(t_{i-1})|x_{i-1}]}. \]
This latter rule is derivable for each $i, 1 \le i \le n$.

Hence using principle SI2 (wrt $U'$), the rule
\[ \frac{y_1 \in \dot{I}(\Omega_1), \dots, y_m \in \dot{I}(\Omega_m)}{\dot{I}(t)[\dot{I}(t_1)|x_1, \dots, \dot{I}(t_n)|x_n] = \dot{I}(t')[\dot{I}(t'_1)|x_1, \dots, \dot{I}(t'_n)|x_n] \in \dot{I}(\Delta)[\dot{I}(t'_1)|x_1, \dots, \dot{I}(t'_n)|x_n]} \]
is a derived rule of $U'$. By definition and lemma 1, this last rule is just
\[ \hat{I}\left(\frac{y_1 \in \Omega_1, \dots, y_m \in \Omega_m}{t[t_1|x_1, \dots, t_n|x_n] = t'[t'_1|x_1, \dots, t'_n|x_n] \in \Delta[t'_1|x_1, \dots, t'_n|x_n]}\right). \]
As required, this rule we have shown to be derivable in $U'$.
\end{proof}

If $U, U'$ and $U''$ are theories and $I$ is a preinterpretation of $U$ in $U'$ and $I'$ is a preinterpretation of $U'$ in $U''$ then define $I' \circ I$ to be the preinterpretation of $U$ in $U''$ given by $(I' \circ I)(L) = \dot{I'}(I(L))$, for all $L \in A_U$. On the way to showing that there is a category of generalised algebraic theories and interpretations we need the following lemma and corollaries.

\begin{lemma}
If $I$ and $I'$ are as above then for any expression $e$ of $A_U$, $\dot{(I' \circ I)}(e) = \dot{I'}(\dot{I}(e))$.
\end{lemma}

\begin{proof}
By induction on the length of $e$.
1. If $e = x \in V$ then $\dot{(I' \circ I)}(e) = x = \dot{I'}(\dot{I}(e))$.
2. If $e = L \in A_U$ then $\dot{(I' \circ I)}(L) = (I' \circ I)(L) = \dot{I'}(I(L)) = \dot{I'}(\dot{I}(L))$.
3. If $e = L(e_1, \dots, e_n)$ then $\dot{(I' \circ I)}(e) = (I' \circ I)(L)[\dot{(I' \circ I)}(e_1)|v_1, \dots, \dot{(I' \circ I)}(e_n)|v_n] = \dot{I'}(I(L))[\dot{I'}(\dot{I}(e_1))|v_1, \dots, \dot{I'}(\dot{I}(e_n))|v_n]$ by the inductive hypothesis, $= \dot{I'}(I(L)[\dot{I}(e_1)|v_1, \dots, \dot{I}(e_n)|v_n])$ by lemma 1, $= \dot{I'}(\dot{I}(L(e_1, \dots, e_n))) = \dot{I'}(\dot{I}(e))$, as required.
\end{proof}

\begin{corollary}
If $I$ and $I'$ are as above then for any derived rule $r$ of $U$, $\widehat{(I' \circ I)}(r) = \hat{I'}(\hat{I}(r))$.
\end{corollary}

\begin{corollary}
If $I$ is an interpretation of $U$ in $U'$ and $I'$ is an interpretation of $U'$ in $U''$ then $I' \circ I$ is an interpretation of $U$ in $U''$.
\end{corollary}

We need some identity morphisms. If $U$ is a theory define a preinterpretation $id_U$ of $U$ in $U$ by:
1. If $A$ is a sort symbol of $U$ introduced by $\frac{x_1 \in \Delta_1, \dots, x_n \in \Delta_n}{A(x_1, \dots, x_n) \text{ is a type}}$, then define $id_U(A) = A(v_1, \dots, v_n)$.
2. If $f$ is an operator symbol of $U$ introduced by $\frac{x_1 \in \Delta_1, \dots, x_n \in \Delta_n}{f(x_1, \dots, x_n) \in \Delta}$, then define $id_U(f) = f(v_1, \dots, v_n)$.

$id_U$ is a preinterpretation of $U$ in $U$ which is quickly seen to have the property that for all expressions $e$ of $A_U$, $\dot{id_U}(e) = e$. Thus it has the property that for all rules $r$ of $U$, $\widehat{id_U}(r) = r$. Hence $id_U$ is an interpretation.

It is now clear that there is a category of generalised algebraic theories and interpretations.

Any interpretation $I : U \to U'$ induces a functor between the categories of algebras, denoted $I\text{-alg} : U'\text{-alg} \to U\text{-alg}$. Any functor equivalent to $I\text{-alg}$, for some $I$, is said to be generalised algebraic.

\textbf{Examples}
1. If the theory $U$ is included in the theory $U'$, that is if every introductory rule and axiom of $U$ is a derived rule of the theory $U'$, then there is a canonical interpretation of $U$ in $U'$. The corresponding algebraic functor for $U'\text{-alg}$ to $U\text{-alg}$ is usually called a forgetful functor.

2. If $\mathcal{C}$ and $\mathcal{D}$ are categories and $F : \mathcal{C} \to \mathcal{D}$ is a functor then there is an interpretation $I_F$ of the theory of category valued presheaves on $\mathcal{C}$ into the theory of category valued presheaves on $\mathcal{D}$. The induced generalised algebraic functor is the functor $Cat^F : Cat^{\mathcal{D}^{op}} \to Cat^{\mathcal{C}^{op}}$.

3. The functor which takes an adjoint pair to the monad induced by that adjoint pair is generalised algebraic. It is induced by an interpretation of the theory of monads in the theory of adjoint pairs.

4. The functor which takes a category valued presheaf on an arbitrary category to the total category of its fibration is generalised algebraic. However it is not induced by an interpretation of the theory of categories into the theory of category valued presheaves, as such. Rather it is induced by an interpretation into the theory of category valued presheaves extended by symbols for disjoint unions (one for objects, one for morphisms). The extension of a theory by symbols for disjoint unions is discussed in \S 1.2.

\section{Contexts and Realisations}

\begin{definition}
If $\langle x_1 \in \Delta_1, \dots, x_n \in \Delta_n \rangle$ and $\langle y_1 \in \Omega_1, \dots, y_m \in \Omega_m \rangle$ are contexts of a theory $U$ then a \underline{realisation} of $\langle y_1 \in \Omega_1, \dots, y_m \in \Omega_m \rangle$ wrt $\langle x_1 \in \Delta_1, \dots, x_n \in \Delta_n \rangle$ is an $m$-tuple $\langle t_1, \dots, t_m \rangle$ such that for each $j, 1 \le j \le m$,
\[ \frac{x_1 \in \Delta_1, \dots, x_n \in \Delta_n}{t_j \in \Omega_j[t_1|y_1, \dots, t_{j-1}|y_{j-1}]} \]
is a derived rule of $U$.
\end{definition}

If $U$ is a generalised algebraic theory then there is a category $\mathbb{R}(U)$ of contexts and realisations of $U$. In $\mathbb{R}(U)$, $\langle t_1, \dots, t_m \rangle : \langle x_1 \in \Delta_1, \dots, x_n \in \Delta_n \rangle \to \langle y_1 \in \Omega_1, \dots, y_m \in \Omega_m \rangle$ iff $\langle t_1, \dots, t_m \rangle$ is a realisation of $\langle y_1 \in \Omega_1, \dots, y_m \in \Omega_m \rangle$ wrt $\langle x_1 \in \Delta_1, \dots, x_n \in \Delta_n \rangle$. The identity morphism on an object $\langle x_1 \in \Delta_1, \dots, x_n \in \Delta_n \rangle$ of $\mathbb{R}(U)$ is the realisation $\langle x_1, \dots, x_n \rangle$ of $\langle x_1 \in \Delta_1, \dots, x_n \in \Delta_n \rangle$ wrt $\langle x_1 \in \Delta_1, \dots, x_n \in \Delta_n \rangle$. Composition in $\mathbb{R}(U)$ is defined by $\langle t_1, \dots, t_m \rangle \circ \langle s_1, \dots, s_p \rangle = \langle s_1[t_1|y_1, \dots, t_m|y_m], \dots, s_p[t_1|y_1, \dots, t_m|y_m] \rangle$, whenever $\langle t_1, \dots, t_m \rangle : \langle x_1 \in \Delta_1, \dots, x_n \in \Delta_n \rangle \to \langle y_1 \in \Omega_1, \dots, y_m \in \Omega_m \rangle$ and $\langle s_1, \dots, s_p \rangle : \langle y_1 \in \Omega_1, \dots, y_m \in \Omega_m \rangle \to \langle z_1 \in \Lambda_1, \dots, z_p \in \Lambda_p \rangle$ in $\mathbb{R}(U)$.

The set of objects of $\mathbb{R}(U)$, i.e. the set of contexts of $U$, is structured as a tree with least element $\langle \rangle$ (the empty context) and with $\langle x_1 \in \Delta_1, \dots, x_n \in \Delta_n \rangle$, $n > 0$, preceded by $\langle x_1 \in \Delta_1, \dots, x_{n-1} \in \Delta_{n-1} \rangle$. For any $n,m \ge 0$ and for any context $\langle x_1 \in \Delta_1, \dots, x_n \in \Delta_n, x_{n+1} \in \Delta_{n+1}, \dots, x_{n+m} \in \Delta_{n+m} \rangle$ of $U$, $\langle x_1, \dots, x_n \rangle$ is a realisation of $\langle x_1 \in \Delta_1, \dots, x_n \in \Delta_n \rangle$ wrt $\langle x_1 \in \Delta_1, \dots, x_{n+m} \in \Delta_{n+m} \rangle$. This map, $\langle x_1, \dots, x_n \rangle : \langle x_1 \in \Delta_1, \dots, x_{n+m} \in \Delta_{n+m} \rangle \to \langle x_1 \in \Delta_1, \dots, x_n \in \Delta_n \rangle$ in $\mathbb{R}(U)$, is denoted $p(\langle x_1 \in \Delta_1, \dots, x_{n+m} \in \Delta_{n+m} \rangle, \langle x_1 \in \Delta_1, \dots, x_n \in \Delta_n \rangle)$. Thus for any $A, B \in \mathbb{R}(U)$ such that $A \triangleleft B$, the morphism $p(B,A)$ is defined and $p(B,A) : B \to A$ in $\mathbb{R}(U)$.

If $A = \langle x_1 \in \Delta_1, \dots, x_n \in \Delta_n \rangle$, $A' = \langle y_1 \in \Omega_1, \dots, y_m \in \Omega_m \rangle$, if $B = \langle y_1 \in \Omega_1, \dots, y_{m+q} \in \Omega_{m+q} \rangle$ and if $f = \langle t_1, \dots, t_m \rangle$ is a realisation of $\langle y_1 \in \Omega_1, \dots, y_m \in \Omega_m \rangle$ wrt $\langle x_1 \in \Delta_1, \dots, x_n \in \Delta_n \rangle$, (in which case
\begin{center}
\begin{tikzcd}
B \arrow[d, "p(B{,}A')"] \\
A'
\end{tikzcd}
\begin{tikzcd}
A \arrow[r, "f"] & A'
\end{tikzcd}
\end{center}
in $\mathbb{R}(U)$) then
\begin{center}
\begin{tikzcd}
f^*B \arrow[r, "q(f{,}B)"] \arrow[d, "p(f^*B{,}A)"] & B \arrow[d, "p(B{,}A')"] \\
A \arrow[r, "f"] & A'
\end{tikzcd}
\end{center}
is a pullback diagram in $\mathbb{R}(U)$.

Where $f^*B = \langle x_1 \in \Delta_1, \dots, x_n \in \Delta_n, y_{m+1} \in \Omega_{m+1}[t_1|y_1, \dots, t_m|y_m], \dots, y_{m+q} \in \Omega_{m+q}[t_1|y_1, \dots, t_m|y_m] \rangle$ and $q(f,B) = \langle t_1, \dots, t_m, y_{m+1}, \dots, y_{m+q} \rangle$.

\section{Intended identity of denotation}

We define an equivalence relation on the derived T and $\in$-rules of a theory $U$ which we call the equivalence relation of intended identity of denotation. The idea is that if $r \equiv r'$ by this relation then any kind of model of $U$ should interpret $r$ and $r'$ as identical objects. Semantically, we should not distinguish between the rule $\frac{x_1 \in \Delta_1, \dots, x_n \in \Delta_n}{\Delta \text{ is a type}}$ and the rule $\frac{x'_1 \in \Delta_1, \dots, x'_n \in \Delta_n[x'_1|x_1, \dots, x'_{n-1}|x_{n-1}]}{\Delta[x'_1|x_1, \dots, x'_n|x_n] \text{ is a type}}$ because the two rules only differ by the choice of variables. Neither should we distinguish between $\frac{x_1 \in \Delta_1, \dots, x_n \in \Delta_n}{\Delta \text{ is a type}}$ and $\frac{x_1 \in \Delta'_1, \dots, x_n \in \Delta'_n}{\Delta' \text{ is a type}}$ if for each $i, 1 \le i \le n$, $\frac{x_1 \in \Delta_1, \dots, x_{i-1} \in \Delta_{i-1}}{\Delta_i = \Delta'_i}$ is a derived rule and $\frac{x_1 \in \Delta_1, \dots, x_n \in \Delta_n}{\Delta = \Delta'}$ is a derived rule.

The relation $\equiv$, of intended identity of denotation, is defined between T-rules of a theory $U$ by $\frac{x_1 \in \Delta_1, \dots, x_n \in \Delta_n}{\Delta_{n+1} \text{ is a type}} \equiv \frac{y_1 \in \Omega_1, \dots, y_m \in \Omega_m}{\Omega_{m+1} \text{ is a type}}$ iff $n=m$ and for each $i, 1 \le i \le n+1$, $\frac{x_1 \in \Delta_1, \dots, x_{i-1} \in \Delta_{i-1}}{\Delta_i = \Omega_i[x_1|y_1, \dots, x_{i-1}|y_{i-1}]}$ is a derived rule of $U$.

\begin{lemma}
$\equiv$ is an equivalence relation.
\end{lemma}

\begin{proof}
1. Reflexiveness follows from the principle of derivation LI1.

2. Symmetry. Suppose that $\frac{x_1 \in \Delta_1, \dots, x_n \in \Delta_n}{\Delta_{n+1} \text{ is a type}} \equiv \frac{y_1 \in \Omega_1, \dots, y_n \in \Omega_n}{\Omega_{n+1} \text{ is a type}}$ we prove by induction on $i$ that for all $i, 1 \le i \le n+1$,
\[ \frac{y_1 \in \Omega_1, \dots, y_{i-1} \in \Omega_{i-1}}{\Omega_i = \Delta_i[y_1|x_1, \dots, y_{i-1}|x_{i-1}]} \]
is derivable.

If $i=1$ then certainly $\Omega_1 = \Delta_1$ is derivable since $\Delta_1 = \Omega_1$ is.
Now if $i>1$ and if for all $j, 1 \le j < i$,
\[ \frac{y_1 \in \Omega_1, \dots, y_{j-1} \in \Omega_{j-1}}{\Omega_j = \Delta_j[y_1|x_1, \dots, y_{j-1}|x_{j-1}]} \]
is derivable; then for all such $j$, $\frac{y_1 \in \Omega_1, \dots, y_{j-1} \in \Omega_{j-1}, y_j \in \Omega_j}{y_j \in \Delta_j[y_1|x_1, \dots, y_{j-1}|x_{j-1}]}$ is derivable. Hence for all such $j$, $\frac{y_1 \in \Omega_1, \dots, y_{i-1} \in \Omega_{i-1}}{y_j \in \Delta_j[y_1|x_1, \dots, y_{j-1}|x_{j-1}]}$ is derivable. Since $\frac{x_1 \in \Delta_1, \dots, x_{i-1} \in \Delta_{i-1}}{\Delta_i = \Omega_i[x_1|y_1, \dots, x_{i-1}|y_{i-1}]}$ is derivable, by the substitution lemma so is
\[ \frac{y_1 \in \Omega_1, \dots, y_{i-1} \in \Omega_{i-1}}{\Delta_i[y_1|x_1, \dots, y_{i-1}|x_{i-1}] = \Omega_i} \]
(because $\Omega_i[x_1|y_1, \dots, x_{i-1}|y_{i-1}][y_1|x_1, \dots, y_{i-1}|x_{i-1}] = \Omega_i$).
Hence $\frac{y_1 \in \Omega_1, \dots, y_{i-1} \in \Omega_{i-1}}{\Omega_i = \Delta_i[y_1|x_1, \dots, y_{i-1}|x_{i-1}]}$ is derivable as required. It follows by induction that $\frac{y_1 \in \Omega_1, \dots, y_n \in \Omega_n}{\Omega_{n+1} \text{ is a type}} \equiv \frac{x_1 \in \Delta_1, \dots, x_n \in \Delta_n}{\Delta_{n+1} \text{ is a type}}$.

3. Transitivity. Assume that $\frac{x_1 \in \Delta_1, \dots, x_n \in \Delta_n}{\Delta_{n+1} \text{ is a type}} \equiv \frac{y_1 \in \Omega_1, \dots, y_m \in \Omega_m}{\Omega_{m+1} \text{ is a type}}$ and that $\frac{y_1 \in \Omega_1, \dots, y_m \in \Omega_m}{\Omega_{m+1} \text{ is a type}} \equiv \frac{z_1 \in \Lambda_1, \dots, z_q \in \Lambda_q}{\Lambda_{q+1} \text{ is a type}}$.
Suppose $1 \le i \le n+1$. We have seen above that for each $j, 1 \le j < i$, $\frac{x_1 \in \Delta_1, \dots, x_{i-1} \in \Delta_{i-1}}{x_j \in \Omega_j[x_1|y_1, \dots, x_{j-1}|y_{j-1}]}$ is a derived rule of $U$. From the substitution lemma and $\frac{y_1 \in \Omega_1, \dots, y_{i-1} \in \Omega_{i-1}}{\Omega_i = \Lambda_i[y_1|z_1, \dots, y_{i-1}|z_{i-1}]}$ it follows that $\frac{x_1 \in \Delta_1, \dots, x_{i-1} \in \Delta_{i-1}}{\Omega_i[x_1|y_1, \dots, x_{i-1}|y_{i-1}] = \Lambda_i[y_1|z_1, \dots, y_{i-1}|z_{i-1}][x_1|y_1, \dots, x_{i-1}|y_{i-1}]}$ is a derived rule. That is $\frac{x_1 \in \Delta_1, \dots, x_{i-1} \in \Delta_{i-1}}{\Omega_i[x_1|y_1, \dots, x_{i-1}|y_{i-1}] = \Lambda_i[x_1|z_1, \dots, x_{i-1}|z_{i-1}]}$ is a derived rule. Since $\frac{x_1 \in \Delta_1, \dots, x_{i-1} \in \Delta_{i-1}}{\Delta_i = \Omega_i[x_1|y_1, \dots, x_{i-1}|y_{i-1}]}$ is a derived rule it follows that $\frac{x_1 \in \Delta_1, \dots, x_{i-1} \in \Delta_{i-1}}{\Delta_i = \Lambda_i[x_1|z_1, \dots, x_{i-1}|z_{i-1}]}$ is a derived rule. Since this is the case for each $i, 1 \le i \le n+1$, it follows that $\frac{x_1 \in \Delta_1, \dots, x_n \in \Delta_n}{\Delta_{n+1} \text{ is a type}} \equiv \frac{z_1 \in \Lambda_1, \dots, z_q \in \Lambda_q}{\Lambda_{q+1} \text{ is a type}}$. Which completes the proof that $\equiv$ is an equivalence relation.
\end{proof}

The equivalence relation $\equiv$ on T-rules can be used to define an equivalence relation on contexts simply by $\langle x_1 \in \Delta_1, \dots, x_n \in \Delta_n \rangle \equiv \langle y_1 \in \Omega_1, \dots, y_m \in \Omega_m \rangle$ iff $\frac{x_1 \in \Delta_1, \dots, x_{n-1} \in \Delta_{n-1}}{\Delta_n \text{ is a type}} \equiv \frac{y_1 \in \Omega_1, \dots, y_{m-1} \in \Omega_{m-1}}{\Omega_m \text{ is a type}}$. We define the equivalence relation $\equiv$ of intended identity of denotation on $\in$-rules of a theory $U$ by $\frac{x_1 \in \Delta_1, \dots, x_n \in \Delta_n}{t \in \Delta} \equiv \frac{y_1 \in \Omega_1, \dots, y_m \in \Omega_m}{s \in \Omega}$ iff $\frac{x_1 \in \Delta_1, \dots, x_n \in \Delta_n}{\Delta \text{ is a type}} \equiv \frac{y_1 \in \Omega_1, \dots, y_m \in \Omega_m}{\Omega \text{ is a type}}$ and $\frac{x_1 \in \Delta_1, \dots, x_n \in \Delta_n}{t = s[x_1|y_1, \dots, x_n|y_n]}$ is a derived rule of $U$. That $\equiv$ is an equivalence relation on $\in$-rules of $U$ is a consequence of the following:

\begin{lemma}
If $\langle x_1 \in \Delta_1, \dots, x_n \in \Delta_n \rangle$ and $\langle y_1 \in \Omega_1, \dots, y_m \in \Omega_m \rangle$ are contexts of $U$ and $\langle x_1 \in \Delta_1, \dots, x_n \in \Delta_n \rangle \equiv \langle y_1 \in \Omega_1, \dots, y_m \in \Omega_m \rangle$ then for all derived rules of $U$ of the form $\frac{y_1 \in \Omega_1, \dots, y_m \in \Omega_m}{\text{Conclusion}}$, the rule
\[ \frac{x_1 \in \Delta_1, \dots, x_n \in \Delta_n}{\text{Conclusion}[x_1|y_1, \dots, x_n|y_n]} \]
is a derived rule of $U$.
\end{lemma}

\begin{proof}
For each $i, 1 \le i \le n$, $\frac{x_1 \in \Delta_1, \dots, x_{i-1} \in \Delta_{i-1}}{\Delta_i = \Omega_i[x_1|y_1, \dots, x_{i-1}|y_{i-1}]}$ is a derived rule of $U$ thus, by the substitution lemma, so is $\frac{x_1 \in \Delta_1, \dots, x_n \in \Delta_n}{\Delta_i = \Omega_i[x_1|y_1, \dots, x_{i-1}|y_{i-1}]}$. Since for each $i, 1 \le i \le n$, $\frac{x_1 \in \Delta_1, \dots, x_n \in \Delta_n}{x_i \in \Delta_i}$ is a derived rule of $U$ then so is $\frac{x_1 \in \Delta_1, \dots, x_n \in \Delta_n}{x_i \in \Omega_i[x_1|y_1, \dots, x_{i-1}|y_{i-1}]}$. Now by the substitution lemma, since $\frac{y_1 \in \Omega_1, \dots, y_m \in \Omega_m}{\text{Conclusion}}$ is a derived rule then so is $\frac{x_1 \in \Delta_1, \dots, x_n \in \Delta_n}{\text{Conclusion}[x_1|y_1, \dots, x_n|y_n]}$.
\end{proof}

\begin{corollary}
$\equiv$ is an equivalence relation on the derived $\in$-rules of $U$.
\end{corollary}

\begin{proof}
1. Reflexiveness follows from principle of derivation LI2.
2. Symmetry. Suppose that $\frac{x_1 \in \Delta_1, \dots, x_n \in \Delta_n}{t \in \Delta} \equiv \frac{y_1 \in \Omega_1, \dots, y_n \in \Omega_n}{s \in \Omega}$.
Then $\langle x_1 \in \Delta_1, \dots, x_n \in \Delta_n \rangle \equiv \langle y_1 \in \Omega_1, \dots, y_n \in \Omega_n \rangle$ so by lemma 2,
$\frac{y_1 \in \Omega_1, \dots, y_n \in \Omega_n}{t[y_1|x_1, \dots, y_n|x_n] = s \in \Delta[y_1|x_1, \dots, y_n|x_n]}$ is a derived rule of $U$.
By wellformedness, since $\frac{y_1 \in \Omega_1, \dots, y_n \in \Omega_n}{s \in \Omega}$ and $\frac{y_1 \in \Omega_1, \dots, y_n \in \Omega_n}{s \in \Delta[y_1|x_1, \dots, y_n|x_n]}$ are derived rules of $U$ so is $\frac{y_1 \in \Omega_1, \dots, y_n \in \Omega_n}{\Omega = \Delta[y_1|x_1, \dots, y_n|x_n]}$. Thus $\frac{y_1 \in \Omega_1, \dots, y_n \in \Omega_n}{s = t[y_1|x_1, \dots, y_n|x_n] \in \Omega}$ is a derived rule of $U$ and $\frac{y_1 \in \Omega_1, \dots, y_n \in \Omega_n}{s \in \Omega} \equiv \frac{x_1 \in \Delta_1, \dots, x_n \in \Delta_n}{t \in \Delta}$.

3. Transitivity. Suppose that $\frac{x_1 \in \Delta_1, \dots, x_n \in \Delta_n}{t \in \Delta} \equiv \frac{y_1 \in \Omega_1, \dots, y_n \in \Omega_n}{s \in \Omega}$ and $\frac{y_1 \in \Omega_1, \dots, y_n \in \Omega_n}{s \in \Omega} \equiv \frac{z_1 \in \Lambda_1, \dots, z_n \in \Lambda_n}{r \in \Lambda}$. Then $\langle x_1 \in \Delta_1, \dots, x_n \in \Delta_n \rangle \equiv \langle y_1 \in \Omega_1, \dots, y_n \in \Omega_n \rangle$ and $\frac{y_1 \in \Omega_1, \dots, y_n \in \Omega_n}{s = r[y_1|z_1, \dots, y_n|z_n] \in \Omega}$ is a derived rule of $U$ thus by lemma 2,
$\frac{x_1 \in \Delta_1, \dots, x_n \in \Delta_n}{s[x_1|y_1, \dots, x_n|y_n] = r[x_1|z_1, \dots, x_n|z_n] \in \Omega[x_1|y_1, \dots, x_n|y_n]}$ is a derived rule of $U$. Since we also have that $\frac{x_1 \in \Delta_1, \dots, x_n \in \Delta_n}{t = s[x_1|y_1, \dots, x_n|y_n] \in \Delta}$ is a derived rule of $U$ we can conclude that so too is $\frac{x_1 \in \Delta_1, \dots, x_n \in \Delta_n}{t = r[x_1|z_1, \dots, x_n|z_n] \in \Delta}$. Thus $\frac{x_1 \in \Delta_1, \dots, x_n \in \Delta_n}{t \in \Delta} \equiv \frac{z_1 \in \Lambda_1, \dots, z_n \in \Lambda_n}{r \in \Lambda}$.
\end{proof}

We define an equivalence relation $\equiv$ on realisations of $U$ as follows. If $\langle t_1, \dots, t_m \rangle$ is a realisation of $\langle y_1 \in \Omega_1, \dots, y_m \in \Omega_m \rangle$ wrt $\langle x_1 \in \Delta_1, \dots, x_n \in \Delta_n \rangle$ and if $\langle t'_1, \dots, t'_m \rangle$ is a realisation of $\langle y'_1 \in \Omega'_1, \dots, y'_m \in \Omega'_m \rangle$ wrt $\langle x'_1 \in \Delta'_1, \dots, x'_n \in \Delta'_n \rangle$ then $\langle t_1, \dots, t_m \rangle \equiv \langle t'_1, \dots, t'_m \rangle$ iff $\langle y_1 \in \Omega_1, \dots, y_m \in \Omega_m \rangle \equiv \langle y'_1 \in \Omega'_1, \dots, y'_m \in \Omega'_m \rangle$ and for each $j, 1 \le j \le m$, $\frac{x_1 \in \Delta_1, \dots, x_n \in \Delta_n}{t_j \in \Omega_j[t_1|y_1, \dots, t_{j-1}|y_{j-1}]} \equiv \frac{x'_1 \in \Delta'_1, \dots, x'_n \in \Delta'_n}{t'_j \in \Omega'_j[t'_1|y'_1, \dots, t'_{j-1}|y'_{j-1}]}$.

\begin{lemma}
If $\langle t_1, \dots, t_m \rangle$ is a realisation of $\langle y_1 \in \Omega_1, \dots, y_m \in \Omega_m \rangle$ wrt $\langle x_1 \in \Delta_1, \dots, x_n \in \Delta_n \rangle$ and if $\langle t'_1, \dots, t'_m \rangle$ is a realisation of $\langle y'_1 \in \Omega'_1, \dots, y'_m \in \Omega'_m \rangle$ wrt $\langle x'_1 \in \Delta'_1, \dots, x'_n \in \Delta'_n \rangle$ and if $\langle t_1, \dots, t_m \rangle \equiv \langle t'_1, \dots, t'_m \rangle$ then
(i) if $\frac{y_1 \in \Omega_1, \dots, y_m \in \Omega_m}{\Omega \text{ is a type}}$ and $\frac{y'_1 \in \Omega'_1, \dots, y'_m \in \Omega'_m}{\Omega' \text{ is a type}}$ are derived rules of $U$ s.t. $\langle y_1 \in \Omega_1, \dots, y_m \in \Omega_m, y \in \Omega \rangle \equiv \langle y'_1 \in \Omega'_1, \dots, y'_m \in \Omega'_m, y' \in \Omega' \rangle$ then $\langle x_1 \in \Delta_1, \dots, x_n \in \Delta_n, y \in \Omega[t_1|y_1, \dots, t_m|y_m] \rangle \equiv \langle x'_1 \in \Delta'_1, \dots, x'_n \in \Delta'_n, y' \in \Omega'[t'_1|y'_1, \dots, t'_m|y'_m] \rangle$.
(ii) If $\frac{y_1 \in \Omega_1, \dots, y_m \in \Omega_m}{s \in \Omega}$ and $\frac{y'_1 \in \Omega'_1, \dots, y'_m \in \Omega'_m}{s' \in \Omega'}$ are derived rules of $U$ s.t. $\frac{y_1 \in \Omega_1, \dots, y_m \in \Omega_m}{s = s'[y_1|y'_1, \dots, y_m|y'_m] \in \Omega}$ is a derived rule of $U$ then $\frac{x_1 \in \Delta_1, \dots, x_n \in \Delta_n}{s[t_1|y_1, \dots, t_m|y_m] = s'[t'_1|y'_1, \dots, t'_m|y'_m][x_1|x'_1, \dots, x_n|x'_n] \in \Omega[t_1|y_1, \dots, t_m|y_m]}$ is a derived rule of $U$.
\end{lemma}

\begin{proof}
Since $\langle t_1, \dots, t_m \rangle \equiv \langle t'_1, \dots, t'_m \rangle$, for each $j, 1 \le j \le m$,
\[ \frac{x_1 \in \Delta_1, \dots, x_n \in \Delta_n}{t_j \in \Omega_j[t_1|y_1, \dots, t_{j-1}|y_{j-1}]} \equiv \frac{x'_1 \in \Delta'_1, \dots, x'_n \in \Delta'_n}{t'_j \in \Omega'_j[t'_1|y'_1, \dots, t'_{j-1}|y'_{j-1}]} \]
That is for each $j, 1 \le j \le m$, $x_1 \in \Delta_1, \dots, x_n \in \Delta_n$
\[ t_j = t'_j[x_1|x'_1, \dots, x_n|x'_n] \in \Omega_j[t_1|y_1, \dots, t_{j-1}|y_{j-1}] \]
is a derived rule of $U$. Now (i) follows immediately by use of principle of derivation SI1. (ii) follows from SI2, using the fact that $S'[y_1|y'_1, \dots, y_m|y'_m][t_1[x_1|x'_1, \dots, x_n|x'_n]|y_1, \dots, t_m[x_1|x'_1, \dots, x_n|x'_n]|y_m] = S'[t'_1|y'_1, \dots, t'_m|y'_m][x_1|x'_1, \dots, x_n|x'_n]$.
\end{proof}

\begin{lemma}
(i) If $\langle t_1, \dots, t_m \rangle$ is a realisation of $\langle y_1 \in \Omega_1, \dots, y_m \in \Omega_m \rangle$ wrt $\langle x_1 \in \Delta_1, \dots, x_n \in \Delta_n \rangle$ and if $\langle x_1 \in \Delta_1, \dots, x_n \in \Delta_n \rangle \equiv \langle x'_1 \in \Delta'_1, \dots, x'_n \in \Delta'_n \rangle$ and $\langle y_1 \in \Omega_1, \dots, y_m \in \Omega_m \rangle \equiv \langle y'_1 \in \Omega'_1, \dots, y'_m \in \Omega'_m \rangle$ then there exists a realisation $\langle t'_1, \dots, t'_m \rangle$ of $\langle y'_1 \in \Omega'_1, \dots, y'_m \in \Omega'_m \rangle$ wrt $\langle x'_1 \in \Delta'_1, \dots, x'_n \in \Delta'_n \rangle$ such that $\langle t_1, \dots, t_m \rangle \equiv \langle t'_1, \dots, t'_m \rangle$.

(ii) If $\langle x_1 \in \Delta_1, \dots, x_n \in \Delta_n, x \in \Delta \rangle$ is a context of $U$ and $\langle x_1 \in \Delta_1, \dots, x_n \in \Delta_n \rangle \equiv \langle x'_1 \in \Delta'_1, \dots, x'_n \in \Delta'_n \rangle$ then there exists $\Delta'$ s.t. $\langle x'_1 \in \Delta'_1, \dots, x'_n \in \Delta'_n, x' \in \Delta' \rangle$ is a context and $\langle x_1 \in \Delta_1, \dots, x_n \in \Delta_n, x \in \Delta \rangle \equiv \langle x'_1 \in \Delta'_1, \dots, x'_n \in \Delta'_n, x' \in \Delta' \rangle$.
\end{lemma}

\begin{proof}
For (i) take $t'_j = t_j[x'_1|x_1, \dots, x'_n|x_n]$ and then use lemma 2. Similarly in (ii) take $\Delta' = \Delta[x'_1|x_1, \dots, x'_n|x_n]$ and use lemma 2.
\end{proof}

The following lemma follows directly from the SI2 principle of derivation.

\begin{lemma}
If $\langle t_1, \dots, t_{m+1} \rangle$ is a realisation of $\langle y_1 \in \Omega_1, \dots, y_{m+1} \in \Omega_{m+1} \rangle$ wrt $\langle x_1 \in \Delta_1, \dots, x_n \in \Delta_n \rangle$, if $\langle s_1, \dots, s_m \rangle$ is a realisation of $\langle y_1 \in \Omega_1, \dots, y_m \in \Omega_m \rangle$ wrt $\langle x_1 \in \Delta_1, \dots, x_n \in \Delta_n \rangle$ and if $\langle s_1, \dots, s_m \rangle \equiv \langle t_1, \dots, t_m \rangle$ then $\langle s_1, \dots, s_m, t_{m+1} \rangle$ is a realisation of $\langle y_1 \in \Omega_1, \dots, y_{m+1} \in \Omega_{m+1} \rangle$ wrt $\langle x_1 \in \Delta_1, \dots, x_n \in \Delta_n \rangle$ and $\langle s_1, \dots, s_m, t_{m+1} \rangle \equiv \langle t_1, \dots, t_{m+1} \rangle$.
\end{lemma}

\section{The Category GAT}

We can define an equivalence relation $\equiv$ on the set of interpretations of a theory $U$ in a theory $U'$. If $I$ and $J$ are two such interpretations then define $I \equiv J$ iff for all introductory rules $R$ of $U$, $\hat{I}(R) \equiv \hat{J}(R)$ in $U'$.

\begin{lemma}
If $I$ and $J$ are interpretations of $U$ in $U'$ and $I \equiv J$ then for all derived T and $\in$-rules $R$ of $U$, $\hat{I}(R) \equiv \hat{J}(R)$.
\end{lemma}

\begin{proof}
By induction on derivations in $U$. We check that principles T1, CF1, CF2(a) and (b) preserve the property.

\underline{T1.} Suppose that $\frac{x_1 \in \Delta_1, \dots, x_n \in \Delta_n}{t \in \Delta}$ and $\frac{x_1 \in \Delta_1, \dots, x_n \in \Delta_n}{\Delta = \Delta'}$ are derived rules of $U$ and that $\hat{I}\left(\frac{x_1 \in \Delta_1, \dots, x_n \in \Delta_n}{t \in \Delta}\right) \equiv \hat{J}\left(\frac{x_1 \in \Delta_1, \dots, x_n \in \Delta_n}{t \in \Delta}\right)$.
Then $x_1 \in \dot{I}(\Delta_1), \dots, x_n \in \dot{I}(\Delta_n) : \dot{I}(t) = \dot{J}(t) \in \dot{I}(\Delta)$ is a derived rule of $U'$ and since $I$ is an interpretation, so is $x_1 \in \dot{I}(\Delta_1), \dots, x_n \in \dot{I}(\Delta_n) : \dot{I}(\Delta) = \dot{I}(\Delta')$. Thus $x_1 \in \dot{I}(\Delta_1), \dots, x_n \in \dot{I}(\Delta_n) : \dot{I}(t) = \dot{J}(t) \in \dot{I}(\Delta')$ is a derived rule of $U'$. And as $\langle x_1 \in \dot{I}(\Delta_1), \dots, x_n \in \dot{I}(\Delta_n) \rangle \equiv \langle x_1 \in \dot{J}(\Delta_1), \dots, x_n \in \dot{J}(\Delta_n) \rangle$ is the case so $\hat{I}\left(\frac{x_1 \in \Delta_1, \dots, x_n \in \Delta_n}{t \in \Delta'}\right) \equiv \hat{J}\left(\frac{x_1 \in \Delta_1, \dots, x_n \in \Delta_n}{t \in \Delta'}\right)$.

\underline{CF1} Suppose $\frac{x_1 \in \Delta_1, \dots, x_n \in \Delta_n}{\Delta_{n+1} \text{ is a type}}$ is a derived rule of $U$ and that $\hat{I}\left(\frac{x_1 \in \Delta_1, \dots, x_n \in \Delta_n}{\Delta_{n+1} \text{ is a type}}\right) \equiv \hat{J}\left(\frac{x_1 \in \Delta_1, \dots, x_n \in \Delta_n}{\Delta_{n+1} \text{ is a type}}\right)$. Then $\langle x_1 \in \dot{I}(\Delta_1), \dots, x_{n+1} \in \dot{I}(\Delta_{n+1}) \rangle \equiv \langle x_1 \in \dot{J}(\Delta_1), \dots, x_{n+1} \in \dot{J}(\Delta_{n+1}) \rangle$ and $x_1 \in \dot{I}(\Delta_1), \dots, x_{n+1} \in \dot{I}(\Delta_{n+1}) : x_i = x_i \in \dot{I}(\Delta_i)$ is a derived rule of $U'$, hence $\hat{I}\left(\frac{x_1 \in \Delta_1, \dots, x_{n+1} \in \Delta_{n+1}}{x_i \in \Delta_i}\right) \equiv \hat{J}\left(\frac{x_1 \in \Delta_1, \dots, x_{n+1} \in \Delta_{n+1}}{x_i \in \Delta_i}\right)$.

\underline{CF2(b)} (Principle CF2(a) will be similar). Suppose that $f$ is an operator symbol of $U$ introduced by $\frac{x_1 \in \Delta_1, \dots, x_n \in \Delta_n}{f(x_1, \dots, x_n) \in \Delta}$ and suppose that for each $i, 1 \le i \le n$, $\frac{y_1 \in \Omega_1, \dots, y_m \in \Omega_m}{t_i \in \Delta_i[t_1|x_1, \dots, t_{i-1}|x_{i-1}]}$ is a derived rule of $U$. Suppose that for each $i, 1 \le i \le n$, $\hat{I}\left(\frac{y_1 \in \Omega_1, \dots, y_m \in \Omega_m}{t_i \in \Delta_i[t_1|x_1, \dots, t_{i-1}|x_{i-1}]}\right) \equiv \hat{J}\left(\frac{y_1 \in \Omega_1, \dots, y_m \in \Omega_m}{t_i \in \Delta_i[t_1|x_1, \dots, t_{i-1}|x_{i-1}]}\right)$. Using lemma 1 of \S 1.10 we see that for each $i, 1 \le i \le n$, $\frac{y_1 \in \dot{I}(\Omega_1), \dots, y_m \in \dot{I}(\Omega_m)}{\dot{I}(t_i) \in \dot{I}(\Delta_i)[\dot{I}(t_1)|x_1, \dots, \dot{I}(t_{i-1})|x_{i-1}]} \equiv \frac{y_1 \in \dot{J}(\Omega_1), \dots, y_m \in \dot{J}(\Omega_m)}{\dot{J}(t_i) \in \dot{J}(\Delta_i)[\dot{J}(t_1)|x_1, \dots, \dot{J}(t_{i-1})|x_{i-1}]}$. Also, since $I \equiv J$,
$x_1 \in \dot{I}(\Delta_1), \dots, x_n \in \dot{I}(\Delta_n) : \dot{I}(f(x_1, \dots, x_n)) \in \dot{I}(\Delta) \equiv x_1 \in \dot{J}(\Delta_1), \dots, x_n \in \dot{J}(\Delta_n) : \dot{J}(f(x_1, \dots, x_n)) \in \dot{J}(\Delta)$. In particular $\langle \dot{I}(t_1), \dots, \dot{I}(t_n) \rangle$ is a realisation of $\langle x_1 \in \dot{I}(\Delta_1), \dots, x_n \in \dot{I}(\Delta_n) \rangle$ wrt $\langle y_1 \in \dot{I}(\Omega_1), \dots, y_m \in \dot{I}(\Omega_m) \rangle$, $\langle \dot{J}(t_1), \dots, \dot{J}(t_n) \rangle$ is a realisation of $\langle x_1 \in \dot{J}(\Delta_1), \dots, x_n \in \dot{J}(\Delta_n) \rangle$ wrt $\langle y_1 \in \dot{J}(\Omega_1), \dots, y_m \in \dot{J}(\Omega_m) \rangle$ and $\langle \dot{I}(t_1), \dots, \dot{I}(t_n) \rangle \equiv \langle \dot{J}(t_1), \dots, \dot{J}(t_n) \rangle$. Thus by lemma 3 (i) of \S 1.12, $\dot{I}(f(x_1, \dots, x_n))[\dot{I}(t_1)|x_1, \dots, \dot{I}(t_n)|x_n] = \dot{J}(f(x_1, \dots, x_n))[\dot{J}(t_1)|x_1, \dots, \dot{J}(t_n)|x_n] \in \dot{I}(\Delta)[\dot{J}(t_1)|x_1, \dots, \dot{J}(t_n)|x_n]$ is derived rule of $U'$.
Using lemma 1 of \S 1.10, this rule is just
\[ \frac{y_1 \in \dot{I}(\Omega_1), \dots, y_m \in \dot{I}(\Omega_m)}{\dot{I}(f(t_1, \dots, t_m)) = \dot{J}(f(t_1, \dots, t_m)) \in \dot{I}(\Delta)[\dot{I}(t_1)|x_1, \dots, \dot{I}(t_n)|x_n]}. \]
Thus
\[ \hat{I}\left(\frac{y_1 \in \Omega_1, \dots, y_m \in \Omega_m}{f(t_1, \dots, t_n) \in \Delta[t_1|x_1, \dots, t_n|x_n]}\right) \equiv \hat{J}\left(\frac{y_1 \in \Omega_1, \dots, y_m \in \Omega_m}{f(t_1, \dots, t_n) \in \Delta[t_1|x_1, \dots, t_n|x_n]}\right) \]
\end{proof}

\begin{corollary}
If $I$ and $J$ are interpretations of $U$ in $U'$ then $I \equiv J$ iff for all derived $\in$-rules $R$ of $U$, $\hat{I}(R) \equiv \hat{J}(R)$.
\end{corollary}

\begin{proof}
Use the fact that $\frac{x_1 \in \Delta_1, \dots, x_n \in \Delta_n}{\Delta \text{ is a type}}$ is a derived rule of $U$ iff $\frac{x_1 \in \Delta_1, \dots, x_n \in \Delta_n, x \in \Delta}{x \in \Delta}$ is a derived rule of $U$. From $\hat{I}\left(\frac{x_1 \in \Delta_1, \dots, x_n \in \Delta_n, x \in \Delta}{x \in \Delta}\right) \equiv \hat{J}\left(\frac{x_1 \in \Delta_1, \dots, x_n \in \Delta_n, x \in \Delta}{x \in \Delta}\right)$ we can deduce that $\hat{I}\left(\frac{x_1 \in \Delta_1, \dots, x_n \in \Delta_n}{\Delta \text{ is a type}}\right) \equiv \hat{J}\left(\frac{x_1 \in \Delta_1, \dots, x_n \in \Delta_n}{\Delta \text{ is a type}}\right)$.
\end{proof}

\begin{corollary}
If $I$ and $J$ are interpretations of $U$ in $U'$ and $I'$ and $J'$ are interpretations of $U'$ in $U''$ then $I \equiv J$ and $I' \equiv J'$ implies $I' \circ I \equiv J' \circ J$.
\end{corollary}

We denote the category of generalised algebraic theories and equivalence classes of interpretations by \underline{GAT}. Composition is defined by $[I'] \circ [I] = [I' \circ I]$. Well defined by corollary 3.

\chapter{CONTEXTUAL CATEGORIES}

\section{Algebraic Semantics}

In this chapter we show that there is a generalised algebraic theory (the theory of contextual categories) whose category of models (the category of contextual categories) is equivalent to the category \underline{GAT} of generalised algebraic theories and equivalence classes of interpretations.

How do we interpret this result? Well, there are many other examples of this very strong kind of relationship holding between an algebraic notion of structure and a syntactic notion of theory. The following list is by no means exhaustive:

\vspace{0.5cm}
\begin{center}
\begin{tabular}{lll}
\textbf{Syntactic Notion} & \textbf{Algebraic Notion} & \textbf{Reference} \\
\hline
Propositional Theory & & \\
of classical logic. & Boolean Algebra. & \\
& & \\
Propositional Theory & & \\
of Intuitionistic Logic. & Heyting Algebra. & \\
& & \\
Single Sorted Algebraic & Lawvere's Notion of & Lawvere [11] \\
or Equational Theory. & an Algebraic Theory. & \\
& & \\
Equational Theory in the language & Cartesian Closed & \\
of the typed $\lambda$-calculus. & Category. & Myers [26] \\
& & \\
Theory of higher order & & \\
Intuitionistic Logic. & Topos. & Fourman [4] \\
& & \\
Coherent Theory. & Grothendieck Site. & Reyes [28] \\
\end{tabular}
\end{center}
\vspace{0.5cm}

In all these cases there is definable the notion of a model of a given theory in a given structure. In each case the category of syntactic theories and equivalence classes of interpretations is equivalent to the category of algebraic structures. This last property is the important characterising property. It can lead to the view that the theories in syntactic form should be dispensed with entirely and the structures be given the title of theories. This seems wasteful. It is to be preferred that we think of the structures as providing a semantics for the theories, in fact, the most general possible semantics. We shall call it the \underline{algebraic semantics}. Thus contextual categories are to provide us with the algebraic semantics of generalised algebraic theories.

In case it should be argued that what we have called the algebraic semantics is really none other than the interpretations of one theory considered as a notion of semantics; well we more or less agree, though perhaps it is only when such are considered as interpretations into algebraic structures that they can be properly said to constitute a notion of semantics. The important point here, though, is that structures do frequently appear quite independently of theories; thus the notion of model is certainly enriched by the isomorphism between theories and structures because theories which arise first as structures (being defined by something like "the theory that corresponds to this here structure") are usually theories which would not otherwise have occurred.

\section{Definition and Examples}

\begin{definition}
A \underline{contextual category} consists of
\begin{enumerate}
    \item A category $\mathbb{C}$ with terminal object $1$.
    \item A tree structure on the objects of $\mathbb{C}$ such that the terminal object $1$ is the unique least element of the tree.
    \item For all $A, B \in |\mathbb{C}|$ such that $A \triangleleft B$ a morphism $p(B): B \to A$ in $\mathbb{C}$. This morphism will also be written just as $B \to A$.
    \item For all $A, A' \in |\mathbb{C}|$, for all $f: A \to A'$ in $\mathbb{C}$, for all $B \in |\mathbb{C}|$ such that $A' \triangleleft B$, an object $f^*B$ of $\mathbb{C}$ and a morphism $q(f,B): f^*B \to B$ such that $A \triangleleft f^*B$ and such that the diagram
    \begin{center}
    \begin{tikzcd}
        f^*B \arrow[r, "q(f{,}B)"] \arrow[d] & B \arrow[d, "p(B)"] \\
        A \arrow[r, "f"] & A'
    \end{tikzcd}
    \end{center}
    is a pullback diagram in $\mathbb{C}$.
\end{enumerate}
Such that:
\begin{enumerate}
    \item[(I)] For all $A, B \in |\mathbb{C}|$ such that $A \triangleleft B$, $id_A^*B = B$ and $q(id_A, B) = id_B$.
    \item[(II)] Whenever
    \begin{center}
    \begin{tikzcd}
        A \arrow[r, "f"] & A' \arrow[r, "f'"] & A''
    \end{tikzcd}
    \end{center}
    in $\mathbb{C}$ and $A'' \triangleleft B$, then $(ff')^*B = f^*(f'^*B)$ and $q(ff', B) = q(f', B) \circ q(f, f'^*B)$.
\end{enumerate}
\end{definition}

We shall see that the objects of a contextual category should be thought of as contexts. Recall that a context is a sequence $\langle x_1 \in \Delta_1, \dots, x_n \in \Delta_n \rangle$ such that the rule
\[ \frac{x_1 \in \Delta_1, \dots, x_{n-1} \in \Delta_{n-1}}{\Delta_n \text{ is a type}} \]
is a derived rule and such that $x_n$ is a variable distinct from each of $x_1, \dots, x_{n-1}$. The tree structure on the set of contexts of a theory is easily seen. For $n \ge 1$, the predecessor of a context $\langle x_1 \in \Delta_1, \dots, x_n \in \Delta_n \rangle$ is the context $\langle x_1 \in \Delta_1, \dots, x_{n-1} \in \Delta_{n-1} \rangle$. The empty context $\langle \rangle$ is the unique least element of the tree.

The morphisms of a contextual category should be thought of as realisations. Recall that a realisation of a context $\langle y_1 \in \Omega_1, \dots, y_m \in \Omega_m \rangle$ with respect to the context $\langle x_1 \in \Delta_1, \dots, x_n \in \Delta_n \rangle$ is just an $m$-tuple $\langle t_1, \dots, t_m \rangle$ such that for each $j$, $1 \le j \le m$, the rule
\[ \frac{x_1 \in \Delta_1, \dots, x_n \in \Delta_n}{t_j \in \Omega_j[t_1|y_1, \dots, t_{j-1}|y_{j-1}]} \]
is a derived rule. Think of the morphism $f: A \to A'$ in a contextual category as being a realisation of the context $A'$ wrt the context $A$.

In \S 1.12 we defined the category $\mathbb{R}(U)$ of contexts and realisations of a theory $U$. We could go on and show that for any theory $U$, the category $\mathbb{R}(U)$ with the pullback structure defined in \S 1.12 (actually we defined more structure than was necessary) is a contextual category. But we do not need this construction. Rather we need the construction of a contextual category $\mathbb{C}(U)$ associated with a theory $U$ as part of the equivalence between contextual categories and generalised algebraic theories. This category $\mathbb{C}(U)$ is a category of equivalence classes of contexts and equivalence classes of realisations of $U$. We shall describe it in some detail.

Recall that in \S 1.13 we defined an equivalence relation $\equiv$ on derived T and $\in$-rules of a theory $U$. This equivalence relation we call the equivalence relation of intended identity of denotation. We used this equivalence relation in defining an equivalence relation $\equiv$ on contexts and realisations:
$\langle x_1 \in \Delta_1, \dots, x_n \in \Delta_n \rangle \equiv \langle y_1 \in \Omega_1, \dots, y_m \in \Omega_m \rangle$ iff
\[ \frac{x_1 \in \Delta_1, \dots, x_{n-1} \in \Delta_{n-1}}{\Delta_n \text{ is a type}} \equiv \frac{y_1 \in \Omega_1, \dots, y_{m-1} \in \Omega_{m-1}}{\Omega_m \text{ is a type}} \]
And if $\langle t_1, \dots, t_m \rangle$ is a realisation of $\langle y_1 \in \Omega_1, \dots, y_m \in \Omega_m \rangle$ wrt $\langle x_1 \in \Delta_1, \dots, x_n \in \Delta_n \rangle$ and if $\langle t'_1, \dots, t'_m \rangle$ is a realisation of $\langle y'_1 \in \Omega'_1, \dots, y'_m \in \Omega'_m \rangle$ wrt $\langle x'_1 \in \Delta'_1, \dots, x'_n \in \Delta'_n \rangle$ then $\langle t_1, \dots, t_m \rangle \equiv \langle t'_1, \dots, t'_m \rangle$ iff for each $j$, $1 \le j \le m$,
\[ \frac{x_1 \in \Delta_1, \dots, x_n \in \Delta_n}{t_j \in \Omega_j[t_1|y_1, \dots, t_{j-1}|y_{j-1}]} \equiv \frac{x'_1 \in \Delta'_1, \dots, x'_n \in \Delta'_n}{t'_j \in \Omega'_j[t'_1|y'_1, \dots, t'_{j-1}|y'_{j-1}]} \]

The category $\mathbb{C}(U)$ is defined to have as objects the equivalence classes of contexts of $U$ and to have as morphisms the equivalence classes of realisations of $U$. More precisely if $\langle x_1 \in \Delta_1, \dots, x_n \in \Delta_n \rangle$ and $\langle y_1 \in \Omega_1, \dots, y_m \in \Omega_m \rangle$ are contexts of $U$ then define
\[ Hom_{\mathbb{C}(U)} ([\langle x_1 \in \Delta_1, \dots, x_n \in \Delta_n \rangle], [\langle y_1 \in \Omega_1, \dots, y_m \in \Omega_m \rangle]) = \]
\[ \{ [\langle t_1, \dots, t_m \rangle] \mid \langle t_1, \dots, t_m \rangle \text{ is a realisation of } \langle y_1 \in \Omega_1, \dots, y_m \in \Omega_m \rangle \text{ wrt } \langle x_1 \in \Delta_1, \dots, x_n \in \Delta_n \rangle \}. \]
Hom is well defined just by lemma 4(i) of \S 1.13.

Whenever $\langle t_1, \dots, t_m \rangle$ is a realisation of $\langle y_1 \in \Omega_1, \dots, y_m \in \Omega_m \rangle$ and whenever $\langle s_1, \dots, s_p \rangle$ is a realisation wrt $\langle y_1 \in \Omega_1, \dots, y_m \in \Omega_m \rangle$ then the composition in $\mathbb{C}(U)$ of $[\langle t_1, \dots, t_m \rangle]$ with $[\langle s_1, \dots, s_p \rangle]$ is defined by
\[ [\langle t_1, \dots, t_m \rangle] \circ [\langle s_1, \dots, s_p \rangle] = [\langle s_1[t_1|y_1, \dots, t_m|y_m], \dots, s_p[t_1|y_1, \dots, t_m|y_m] \rangle]. \]
Composition is well defined, this follows from lemma 3(ii) of \S 1.13. The identity morphisms in $\mathbb{C}(U)$ are given by $id_{[\langle x_1 \in \Delta_1, \dots, x_n \in \Delta_n \rangle]} = [\langle x_1, \dots, x_n \rangle]$. Well definedness is trivial.

The objects of $\mathbb{C}(U)$ are structured as a tree by taking the predecessor of $[\langle x_1 \in \Delta_1, \dots, x_n \in \Delta_n \rangle]$ to be $[\langle x_1 \in \Delta_1, \dots, x_{n-1} \in \Delta_{n-1} \rangle]$. The tree structure is well defined because by definition if $\langle x_1 \in \Delta_1, \dots, x_n \in \Delta_n \rangle \equiv \langle x'_1 \in \Delta'_1, \dots, x'_n \in \Delta'_n \rangle$ then $\langle x_1 \in \Delta_1, \dots, x_{n-1} \in \Delta_{n-1} \rangle \equiv \langle x'_1 \in \Delta'_1, \dots, x'_{n-1} \in \Delta'_{n-1} \rangle$. $[\langle \rangle]$ is the least element of the tree.

$[\langle \rangle]$ is a terminal object of $\mathbb{C}(U)$ because by definition of $Hom_{\mathbb{C}(U)}$, $Hom_{\mathbb{C}(U)}([\langle x_1 \in \Delta_1, \dots, x_n \in \Delta_n \rangle], [\langle \rangle]) = \{ [\langle \rangle] \mid \langle \rangle \text{ is a realisation of } \langle \rangle \text{ wrt } \langle x_1 \in \Delta_1, \dots, x_n \in \Delta_n \rangle \}$ and because by the definition of realisation, $\langle \rangle$ is a realisation of the context $\langle \rangle$ wrt the context $\langle x_1 \in \Delta_1, \dots, x_n \in \Delta_n \rangle$.

If $A \triangleleft B$ in $\mathbb{C}(U)$, say $A = [\langle x_1 \in \Delta_1, \dots, x_n \in \Delta_n \rangle]$ and $B = [\langle x_1 \in \Delta_1, \dots, x_n \in \Delta_n, x \in \Delta \rangle]$, then define $p(B): B \to A$ by $p(B) = [\langle x_1, \dots, x_n \rangle]$.

If
\begin{center}
\begin{tikzcd}
& B \\
A \arrow[r, "f"] & A' \arrow[u]
\end{tikzcd}
\end{center}
in $\mathbb{C}(U)$, say $A = [\langle x_1 \in \Delta_1, \dots, x_n \in \Delta_n \rangle]$, $A' = [\langle y_1 \in \Omega_1, \dots, y_m \in \Omega_m \rangle]$, $B = [\langle y_1 \in \Omega_1, \dots, y_m \in \Omega_m, y \in \Omega \rangle]$ and $f = [\langle t_1, \dots, t_m \rangle]$, then define $f^*B = [\langle x_1 \in \Delta_1, \dots, x_n \in \Delta_n, y \in \Omega[t_1|y_1, \dots, t_m|y_m] \rangle]$ and $q(f,B) = [\langle t_1, \dots, t_m, y \rangle]$. $f^*B$ is well defined, by lemma 3(i) of \S 1.13.

\begin{lemma}
$\mathbb{C}(U)$ is a contextual category.
\end{lemma}

\begin{proof}
Firstly we must show that whenever
\begin{center}
\begin{tikzcd}
f^*B \arrow[r, "q(f{,}B)"] \arrow[d] & B \arrow[d] \\
A \arrow[r, "f"] & A'
\end{tikzcd}
\end{center}
in $\mathbb{C}(U)$ then the diagram is a pullback diagram in $\mathbb{C}(U)$.

So suppose that $A = [\langle x_1 \in \Delta_1, \dots, x_n \in \Delta_n \rangle]$, $B = [\langle y_1 \in \Omega_1, \dots, y_m \in \Omega_m, y \in \Omega \rangle]$ and that $f = [\langle t_1, \dots, t_m \rangle]$. Suppose also that $C$ is an object of $\mathbb{C}(U)$ and that $g: C \to A$ and $g': C \to B$ in $\mathbb{C}(U)$ such that the diagram
\begin{center}
\begin{tikzcd}
C \arrow[r, "g'"] \arrow[d, "g"] & B \arrow[d] \\
A \arrow[r, "f"] & A'
\end{tikzcd}
\end{center}
commutes. Call this diagram (I). We can suppose that $C = [\langle z_1 \in \Lambda_1, \dots, z_p \in \Lambda_p \rangle]$, $g = [\langle r_1, \dots, r_n \rangle]$, $g' = [\langle s_1, \dots, s_m, s \rangle]$, where $\langle r_1, \dots, r_n \rangle$ is some realisation of $\langle x_1 \in \Delta_1, \dots, x_n \in \Delta_n \rangle$ wrt $\langle z_1 \in \Lambda_1, \dots, z_p \in \Lambda_p \rangle$ and $\langle s_1, \dots, s_m, s \rangle$ is a realisation of $\langle y_1 \in \Omega_1, \dots, y_m \in \Omega_m, y \in \Omega \rangle$ wrt $\langle z_1 \in \Lambda_1, \dots, z_p \in \Lambda_p \rangle$.

We must show that there exists a unique $h: C \to f^*B$ in $\mathbb{C}(U)$ such that diagrams (II) and (III) both commute.

\begin{center}
\begin{minipage}{0.4\textwidth}
\begin{tikzcd}
C \arrow[r, "h"] \arrow[d, "g"'] & f^*B \arrow[dl] \\
A & 
\end{tikzcd}
(II)
\end{minipage}
\begin{minipage}{0.4\textwidth}
\begin{tikzcd}
C \arrow[r, "g{'}"] \arrow[d, "h"'] & B \\
f^*B \arrow[ur, "q(f{,}B)"'] & 
\end{tikzcd}
(III)
\end{minipage}
\end{center}

I claim that $[\langle r_1, \dots, r_n, s \rangle]$ is such an $h$. Since diagram (I) commutes, $[\langle t_1[r_1|x_1, \dots, r_n|x_n], \dots, t_m[r_1|x_1, \dots, r_n|x_n] \rangle] = [\langle s_1, \dots, s_m \rangle]$. Hence for all $j, 1 \le j \le m$,
\[ \frac{z_1 \in \Lambda_1, \dots, z_p \in \Lambda_p}{s_j = t_j[r_1|x_1, \dots, r_n|x_n] \in \Omega_j[t_1|y_1, \dots, t_{j-1}|y_{j-1}][r_1|x_1, \dots, r_n|x_n]} \]
is a derived rule of $U$. Thus as $\langle s_1, \dots, s_m, s \rangle$ is a realisation of $\langle y_1 \in \Omega_1, \dots, y_m \in \Omega_m, y \in \Omega \rangle$, the rule
\[ \frac{z_1 \in \Lambda_1, \dots, z_p \in \Lambda_p}{s \in \Omega[t_1[r_1|x_1, \dots, r_n|x_n]|y_1, \dots, t_m[r_1|x_1, \dots, r_n|x_n]|y_m]} \]
is a derived rule of $U$. Hence $\langle r_1, \dots, r_n, s \rangle$ is a realisation of $\langle x_1 \in \Delta_1, \dots, x_n \in \Delta_n, y \in \Omega[t_1|y_1, \dots, t_m|y_m] \rangle$ wrt $\langle z_1 \in \Lambda_1, \dots, z_p \in \Lambda_p \rangle$ and thus $[\langle r_1, \dots, r_n, s \rangle] : C \to f^*B$ in $\mathbb{C}(U)$.
Setting $h = [\langle r_1, \dots, r_n, s \rangle]$ then (II) commutes because $[\langle r_1, \dots, r_n, s \rangle] \circ [\langle x_1, \dots, x_n \rangle] = [\langle r_1, \dots, r_n \rangle]$ and (III) commutes because $[\langle r_1, \dots, r_n, s \rangle] \circ [\langle t_1, \dots, t_m, y \rangle] = [\langle t_1[r_1|x_1, \dots, r_n|x_n], \dots, t_m[r_1|x_1, \dots, r_n|x_n], s \rangle] = [\langle s_1, \dots, s_m, s \rangle]$. So $[\langle r_1, \dots, r_n, s \rangle]$ is certainly such an $h$. To show that it is the unique such $h$ suppose now that $k$ is an arbitrary morphism $k: C \to f^*B$ in $\mathbb{C}(U)$ such that the diagrams (II) and (III) commute say $k = [\langle r'_1, \dots, r'_n, s' \rangle]$. Since (II) commutes, for each $i, 1 \le i \le n$, the rule $\frac{z_1 \in \Lambda_1, \dots, z_p \in \Lambda_p}{r_i = r'_i \in \Delta_i[r_1|x_1, \dots, r_{i-1}|x_{i-1}]}$ is a derived rule of $U$. Since (III) commutes, the rule
\[ \frac{z_1 \in \Lambda_1, \dots, z_p \in \Lambda_p}{s = s' \in \Omega[s_1|y_1, \dots, s_m|y_m]} \]
is a derived rule of $U$. Hence $k = [\langle r'_1, \dots, r'_n, s' \rangle] = [\langle r_1, \dots, r_n, s \rangle]$. Which completes the proof that
\begin{center}
\begin{tikzcd}
f^*B \arrow[r, "q(f{,}B)"] \arrow[d] & B \arrow[d] \\
A \arrow[r, "f"] & A'
\end{tikzcd}
\end{center}
is a pullback diagram in $\mathbb{C}(U)$.

It remains to show that the axioms (I) and (II) of the definition of contextual category hold of the structure $\mathbb{C}(U)$. Well, it is easy to see that (I) must hold because if $B = [\langle x_1 \in \Delta_1, \dots, x_n \in \Delta_n, x \in \Delta \rangle]$ then $\Delta[x_1|x_1, \dots, x_n|x_n] = \Delta$. Similarly (II) holds because $\Lambda[s_1|z_1, \dots, s_q|z_q][t_1|y_1, \dots, t_m|y_m] = \Lambda[s_1[t_1|y_1, \dots, t_m|y_m]|z_1, \dots, s_q[t_1|y_1, \dots, t_m|y_m]|z_q]$, whenever $\langle z_1 \in \Lambda_1, \dots, z_q \in \Lambda_q, z \in \Lambda \rangle$ is a context and $\langle s_1, \dots, s_q \rangle$ is a realisation of $\langle z_1 \in \Lambda_1, \dots, z_q \in \Lambda_q \rangle$ wrt $\langle y_1 \in \Omega_1, \dots, y_m \in \Omega_m \rangle$ and $\langle t_1, \dots, t_m \rangle$ is a realisation of $\langle y_1 \in \Omega_1, \dots, y_m \in \Omega_m \rangle$.
\end{proof}

We now turn to the definition of a (large) contextual category, \underline{Fam}, which plays the same role among contextual categories as does the category \underline{Set} among categories. Whereas \underline{Set} is the structured collection of functions so it is that \underline{Fam} is the structured collection of operators. We must refer back to \S 1.9 to the discussion of operators and sets, families of sets, families of families of sets and so on.

The tree of objects of \underline{Fam} is the tree of families introduced in \S 1.9. Thus it is the tree of sets, families of sets, families of families of sets, and so on with a formally adjoined least element 1.

For $n,m \ge 0$, if $1 \triangleleft A_1 \dots \triangleleft A_n$ and $1 \triangleleft B_1 \dots \triangleleft B_m$ in \underline{Fam} then $Hom_{\underline{Fam}}(A_n, B_m) \stackrel{\text{def}}{=} \{ \langle f_1, \dots, f_m \rangle \mid f_1, \dots, f_m \text{ are operators such that the status of the operator } f_j \text{ is given by for } a_1 \in A_1, \text{ for } a_2 \in A_2(a_1) \dots \text{ for } a_n \in A_n(a_1, \dots, a_{n-1}) : f_j(a_1, \dots, a_n) \in B_j(f_1(a_1, \dots, a_n), \dots, f_{j-1}(a_1, \dots, a_n)) \}$

In particular if $n=0$ then we get $Hom(1, B_m) = \{ \langle b_1, \dots, b_m \rangle \mid b_1 \in B_1, b_2 \in B_2(b_1), \dots, \text{ and } b_m \in B_m(b_1, \dots, b_{m-1}) \}$. On the other hand 1 is the terminal object of \underline{Fam} because $Hom(A_n, 1) = \{ \langle \rangle \}$.

Composition in \underline{Fam} is defined as follows. If $1 \triangleleft A_1 \dots \triangleleft A_n$, $1 \triangleleft B_1 \dots \triangleleft B_m$ and $1 \triangleleft C_1 \dots \triangleleft C_q$ in \underline{Fam} and if $\langle f_1, \dots, f_m \rangle : A_n \to B_m$, $\langle g_1, \dots, g_q \rangle : B_m \to C_q$ then the composition is given by $\langle f_1, \dots, f_m \rangle \circ \langle g_1, \dots, g_q \rangle = \langle h_1, \dots, h_q \rangle$, where for each $k$, $1 \le k \le q$, $h_k$ is defined by $h_k(a_1, \dots, a_n) = g_k(f_1(a_1, \dots, a_n), \dots, f_m(a_1, \dots, a_n))$, whenever $a_1 \in A_1, \dots, a_n \in A_n(a_1, \dots, a_{n-1})$.

If $1 \triangleleft A_1 \dots \triangleleft A_n \triangleleft A$ in \underline{Fam} then $p(A) : A \to A_n$ in \underline{Fam} is given by $p(A) = \langle h_1, \dots, h_n \rangle$, where for each $i$, $1 \le i \le n$, $h_i$ is defined by $h_i(a_1, \dots, a_n, a) = a_i$, whenever $a_1 \in A_1, \dots, a_n \in A_n(a_1, \dots, a_{n-1})$ and $a \in A(a_1, \dots, a_n)$.

If $1 \triangleleft A_1 \dots \triangleleft A_n$ and $1 \triangleleft B_1 \dots \triangleleft B_m \triangleleft B$ in \underline{Fam} and if $\langle f_1, \dots, f_m \rangle : A_n \to B_m$ then $\langle f_1, \dots, f_m \rangle^*B$ is defined to be the family $\lambda a_1 \in A_1 . \lambda a_2 \in A_2(a_1) \dots \lambda a_n \in A_n(a_1, \dots, a_{n-1}) . B(f_1(a_1, \dots, a_n), \dots, f_m(a_1, \dots, a_n))$. In this situation $q(\langle f_1, \dots, f_m \rangle, B) = \langle f_1, \dots, f_m, \gamma \rangle$, where $\gamma$ is the operator defined by $\gamma(a_1, \dots, a_n, b) = b$, whenever $a_1 \in A_1, \dots, a_n \in A_n(a_1, \dots, a_{n-1})$ and $b \in B(f_1(a_1, \dots, a_n), \dots, f_m(a_1, \dots, a_n))$.

The proof that \underline{Fam}, so defined, is a contextual category is rather simple. Because the statements asserting the status of operators and families are so similar to the formal rules the proof is similar to the proof that $\mathbb{C}(U)$ is always a contextual category, only easier.

The homomorphisms between contextual categories are called contextual functors. Thus:

\begin{definition}
If $\mathbb{C}$ and $\mathbb{C}'$ are contextual categories then a \underline{contextual functor} $F: \mathbb{C} \to \mathbb{C}'$ is a functor $F: \mathbb{C} \to \mathbb{C}'$ such that:
\begin{enumerate}
    \item $F(1) = 1$ and if $A \triangleleft B$ in $\mathbb{C}$ then $F(A) \triangleleft F(B)$ in $\mathbb{C}'$.
    \item For all objects $A$ of $\mathbb{C}$, $F(p(A)) = p(F(A))$.
    \item For all $f$ and $B$ such that $f^*B$ is defined in $\mathbb{C}$, $F(f^*B) = F(f)^*F(B)$ and $F(q(f,B)) = q(F(f), F(B))$.
\end{enumerate}
\end{definition}

The category of contextual categories and contextual functors is denoted \underline{Con}.

\section{Notation and Basic Lemmas}

If $A \le B$ in the contextual category $\mathbb{C}$ then define the morphism $p(B,A) : B \to A$ in $\mathbb{C}$, also written just as $B \to A$, by $p(B,A) = p(B) \circ p(X_n) \circ \dots \circ p(X_1)$, where $X_1, \dots, X_n$ is the unique sequence of objects of $\mathbb{C}$ such that $A \triangleleft X_1 \dots \triangleleft X_n \triangleleft B$ in $\mathbb{C}$. (in the case $A=B$, $p(B,A)=id_A$).

The contextual category structure supplies us with pullbacks for any map of the form $p(X)$, these given pullbacks can be pieced together to obtain a pullback for any morphism of the form $p(B,A)$ along any morphism with codomain $A$. For we have the following very trivial lemma about pullbacks in any category $\mathcal{C}$.

\begin{lemma}
In any category $\mathcal{C}$,
(a) if $f: A \to B$ in $\mathcal{C}$ then
\begin{center}
\begin{tikzcd}
A \arrow[r, "f"] \arrow[d, "id(A)"] & B \arrow[d, "id(B)"] \\
A \arrow[r, "f"] & B
\end{tikzcd}
\end{center}
is a pullback diagram in $\mathcal{C}$.
(b) If
\begin{center}
\begin{minipage}{0.4\textwidth}
\begin{tikzcd}
A_2 \arrow[r, "f_2"] \arrow[d, "h_1"] & B_2 \arrow[d, "g_1"] \\
A_1 \arrow[r, "f_1"] & B_1
\end{tikzcd}
\end{minipage}
and
\begin{minipage}{0.4\textwidth}
\begin{tikzcd}
A_3 \arrow[r, "f_3"] \arrow[d, "h_2"] & B_3 \arrow[d, "g_2"] \\
A_2 \arrow[r, "f_2"] & B_2
\end{tikzcd}
\end{minipage}
\end{center}
are pullback diagrams in $\mathcal{C}$, then so is
\begin{center}
\begin{tikzcd}
A_3 \arrow[r, "f_3"] \arrow[d, "h_1 h_2"] & B_3 \arrow[d, "g_1 g_2"] \\
A_1 \arrow[r, "f_1"] & B_1
\end{tikzcd}
\end{center}
\end{lemma}

This means, for example, that if $A' \triangleleft X \triangleleft B$ in a contextual category $\mathbb{C}$, if $f: A \to A'$ in $\mathbb{C}$ then the diagram
\begin{center}
\begin{tikzcd}
q(f,X)^*B \arrow[r, "q(q(f{,}X){,}B)"] \arrow[d] & B \arrow[d, "p(B{,}A')"] \\
A \arrow[r, "f"] & A'
\end{tikzcd}
\end{center}
is a pullback diagram, since both of the two diagrams
\begin{center}
\begin{minipage}{0.4\textwidth}
\begin{tikzcd}
f^*X \arrow[r, "q(f{,}X)"] \arrow[d] & X \arrow[d] \\
A \arrow[r, "f"] & A'
\end{tikzcd}
\end{minipage}
and
\begin{minipage}{0.4\textwidth}
\begin{tikzcd}
q(f,X)^*B \arrow[r, "q(q(f{,}X){,}B)"] \arrow[d] & B \arrow[d] \\
f^*X \arrow[r, "q(f{,}X)"] & X
\end{tikzcd}
\end{minipage}
\end{center}
are pullback diagrams. So that gives us a canonical pullback for the morphism $B \to A'$ along any $f: A \to A'$.

In general, it means that whenever $A' \le B$ in $\mathbb{C}$ and $f: A \to A'$ in $\mathbb{C}$ then we have a canonical pullback for the morphism $p(B,A')$ along $f$. For by iterating the method used when $A' \triangleleft X \triangleleft B$ we see that if $A' \triangleleft X_1 \dots \triangleleft X_n \triangleleft B$ in $\mathbb{C}$ and if $f: A \to A'$ then the diagram
\begin{center}
\begin{tikzcd}
q(q(\dots q(f,X_1), X_2) \dots, X_n)^*B \arrow[r, "q(q(\dots q(f{,}X_1){,} \dots{,} X_n){,} B)"] \arrow[d] & B \arrow[d] \\
A \arrow[r, "f"] & A'
\end{tikzcd}
\end{center}
is a pullback diagram in $\mathbb{C}$.

Since these constructed pullbacks form an important part of contextual category structure we would like a simpler notation for them. As no confusion is likely, we extend the $^*$ and $q$ notation to cover these new pullback diagrams.

From now on if $f: A \to A'$ in $\mathbb{C}$ and $A' \le B$ in $\mathbb{C}$ then the diagram
\begin{center}
\begin{tikzcd}
f^*B \arrow[r, "q(f{,}B)"] \arrow[d] & B \arrow[d] \\
A \arrow[r, "f"] & A'
\end{tikzcd}
\end{center}
is the canonical pullback diagram defined above. The following observation which follows from the way the new pullback diagrams are constructed contains all the information we need to remember about that construction:

\begin{lemma}
If $f: A \to A'$ and $A' \triangleleft X \le B$ in the contextual category $\mathbb{C}$ then $f^*B = q(f,X)^*B$ and $q(f,B) = q(q(f,X), B)$.
\begin{center}
\begin{tikzcd}
q(f,X)^*B \arrow[r, "q(q(f{,}X){,}B)"] \arrow[d] & B \arrow[d] \\
f^*X \arrow[r, "q(f{,}X)"] \arrow[d] & X \arrow[d, "p"] \\
A \arrow[r, "f"] & A'
\end{tikzcd}
\end{center}
\end{lemma}

If $U$ is a generalised algebraic theory and if $f: A \to A'$ and $A' \le B$ in the contextual category $\mathbb{C}(U)$ then supposing that $A = [\langle x_1 \in \Delta_1, \dots, x_n \in \Delta_n \rangle]$, $A' = [\langle y_1 \in \Omega_1, \dots, y_m \in \Omega_m \rangle]$, $B = [\langle y_1 \in \Omega_1, \dots, y_m \in \Omega_m, y_{m+1} \in \Omega_{m+1}, \dots, y_{m+q} \in \Omega_{m+q} \rangle]$ and $f = [\langle t_1, \dots, t_m \rangle]$ then $f^*B = [\langle x_1 \in \Delta_1, \dots, x_n \in \Delta_n, y_{m+1} \in \Omega_{m+1}[t_1|y_1, \dots, t_m|y_m], \dots, y_{m+q} \in \Omega_{m+q}[t_1|y_1, \dots, t_m|y_m] \rangle]$ and $q(f,B) = [\langle t_1, \dots, t_m, y_{m+1}, \dots, y_{m+q} \rangle]$.

For $\mathbb{C}$ a contextual category and for $A$ an object of $\mathbb{C}$ we can define a contextual category $\mathbb{C}_A$ whose tree of objects is the tree of those objects of $\mathbb{C}$ which appears above $A$. The construction is similar to the construction of the comma category $\mathbb{C}/A$ of a category $\mathbb{C}$ at an object $A$. The similarity is enhanced by the fact that if $f: A \to A'$ in $\mathbb{C}$ then pulling back along $f$ induces a contextual functor $\mathbb{C}_f : \mathbb{C}_{A'} \to \mathbb{C}_A$.

$\mathbb{C}_A$ is defined by $|\mathbb{C}_A| = \{ B \in |\mathbb{C}| \text{ s.t. } A \le B \}$, $Hom_{\mathbb{C}_A}(B, B') = \{ g: B \to B' \text{ in } \mathbb{C} \mid p(B',A) \circ g = p(B,A) \}$.

We show that $\mathbb{C}_A$ inherits the structure of a contextual category from $\mathbb{C}$. It suffices to show that $\mathbb{C}_A$ is closed under the operations $p$, $^*$ and $q$ of $\mathbb{C}$ and that these operations yield pullback diagrams in $\mathbb{C}_A$.

\begin{lemma}
(a) If $A \le B \triangleleft C$ in $\mathbb{C}$ then $p(C): C \to B$ in $\mathbb{C}_A$.
(b) If $g: B \to B'$ in $\mathbb{C}_A$ and $B' \triangleleft C$ in $\mathbb{C}_A$, then $g^*C \in |\mathbb{C}_A|$ and $q(g,C): g^*C \to C$ in $\mathbb{C}_A$.
(c) If $g: B \to B'$ in $\mathbb{C}_A$ and $B' \triangleleft C$ in $\mathbb{C}_A$, then
\begin{center}
\begin{tikzcd}
g^*C \arrow[r, "q(g{,}C)"] \arrow[d] & C \arrow[d] \\
B \arrow[r, "g"] & B'
\end{tikzcd}
\end{center}
is a pullback diagram in $\mathbb{C}_A$.
\end{lemma}

\begin{proof}
(a) Trivial $p(C) \circ p(B,A) = p(C,A)$.
(b) $g^*C \in |\mathbb{C}_A|$ since $A \le B \le g^*C$. $q(g,C): g^*C \to C$ in $\mathbb{C}_A$ because $q(g,C): g^*C \to C$ in $\mathbb{C}$ and $q(g,C) \circ p(C,A) = q(g,C) \circ p(C) \circ p(B',A) = p(g^*C) \circ g \circ p(B',A) = p(g^*C) \circ p(B,A) = p(g^*C, A)$.
(c) Assume that $D \in |\mathbb{C}_A|$ and that $h_1: D \to B$, $h_2: D \to C$ in $\mathbb{C}_A$ such that $h_1 \circ g = h_2 \circ p(C)$. We must show that there exists a unique $k: D \to g^*C$ in $\mathbb{C}_A$ such that $k \circ p(g^*C) = h_1$ and $k \circ q(g,C) = h_2$.
Since we also have $D \in |\mathbb{C}|$ and $h_1: D \to B$, $h_2: D \to C$ in $\mathbb{C}$ such that $h_1 \circ g = h_2 \circ p(C)$, and since
\begin{center}
\begin{tikzcd}
g^*C \arrow[r, "q(g{,}C)"] \arrow[d] & C \arrow[d] \\
B \arrow[r, "g"] & B'
\end{tikzcd}
\end{center}
is a pullback diagram in $\mathbb{C}$, it follows that there exists a unique $k: D \to g^*C$ in $\mathbb{C}$ such that $k \circ p(g^*C) = h_1$ and $k \circ q(g,C) = h_2$. It suffices to show that $k$ is in $\mathbb{C}_A$, i.e. that $k \circ p(g^*C, A) = p(D,A)$. Since $h_1$ is in $\mathbb{C}_A$ we know that $h_1 \circ p(B,A) = p(D,A)$. Hence $k \circ p(g^*C, A) = k \circ p(g^*C) \circ p(B,A) = h_1 \circ p(B,A) = p(D,A)$.
\end{proof}

So $\mathbb{C}_A$ is a contextual category. Now if we suppose that $f: A \to A'$ in $\mathbb{C}$ then we can define a functor $\mathbb{C}_f: \mathbb{C}_{A'} \to \mathbb{C}_A$ by
\begin{center}
\begin{tikzcd}
B \arrow[r, "g"] & B'
\end{tikzcd}
$\xrightarrow{\mathbb{C}_f}$
\begin{tikzcd}
f^*B \arrow[r, "f^*g"] & f^*B'
\end{tikzcd}
\end{center}
where for any $g: B \to B'$ in $\mathbb{C}_{A'}$, $f^*g$ is the unique morphism from $f^*B$ to $f^*B'$ in $\mathbb{C}_A$ such that $f^*g \circ q(f,B') = q(f,B) \circ g$. Such a morphism exists uniquely in $\mathbb{C}_A$ because
\begin{center}
\begin{tikzcd}
f^*B' \arrow[r, "q(f{,}B')"] \arrow[d] & B' \arrow[d] \\
A \arrow[r, "f"] & A'
\end{tikzcd}
\end{center}
is a pullback diagram in $\mathbb{C}$, because $q(f,B) \circ g \circ p(B', A') = q(f,B) \circ p(B,A') = p(f^*B, A) \circ f$ and because $f^*g$ is in $\mathbb{C}_A$ iff $f^*g \circ p(f^*B', A) = p(f^*B, A)$.

We must check that $\mathbb{C}_f$, so defined, is a functor. So we check that if $g: B \to B'$ and $g': B' \to B''$ in $\mathbb{C}_{A'}$ then $f^*(g \circ g') = f^*g \circ f^*g'$. Well, $f^*(g \circ g')$ is defined as the unique morphism from $f^*B$ to $f^*B''$ in $\mathbb{C}_A$ such that $f^*(g \circ g') \circ q(f,B'') = q(f,B) \circ g \circ g'$. But $f^*g \circ f^*g' : f^*B \to f^*B''$ in $\mathbb{C}_A$ and satisfies $f^*g \circ f^*g' \circ q(f,B'') = f^*g \circ q(f,B') \circ g' = q(f,B) \circ g \circ g'$. So because of the uniqueness of $f^*(g \circ g')$ we must have $f^*(g \circ g') = f^*g \circ f^*g'$.

In fact, $\mathbb{C}_f$ is a contextual functor:

\begin{lemma}
(a) If $A' \le B \triangleleft C$ in $\mathbb{C}$ then $f^*p(C) = p(f^*C)$.
(b) If $g: B \to B'$ in $\mathbb{C}_{A'}$ and $B' \triangleleft C$ then $f^*g^*C = (f^*g)^*(f^*C)$ and $f^*(q(g,C)) = q(f^*g, f^*C)$.
\end{lemma}

\begin{proof}
(a) $f^*(p(C))$ is defined to be the unique map from $f^*C$ to $f^*B$ in $\mathbb{C}_A$ such that $f^*p(C) \circ q(f,B) = q(f,C) \circ p(C)$. So by uniqueness $p(f^*C) = f^*p(C)$ since $p(f^*C) \circ q(f,B) = q(f,C) \circ p(C)$ follows from lemma 1.

(b) $f^*(g^*C) = q(f,B) \circ g^*C$, by lemma 1.
$= (q(f,B) \circ g)^*C$, by axiom (II).
$= (f^*g \circ q(f,B'))^*C$, by def. of $f^*g$.
$= (f^*g)^* (q(f,B')^*C)$, by axiom (II).
$= (f^*g)^* (f^*C)$, by lemma 1. As required for the first part.

Now $f^*(q(g,C))$ is the unique morphism from $f^*g^*C$ to $f^*C$ in $\mathbb{C}_A$ such that $f^*(q(g,C)) \circ q(f,C) = q(f,g^*C) \circ q(g,C)$. That $q(f^*g, f^*C) = f^*(q(g,C))$ follows from uniqueness because
$q(f^*g, f^*C) \circ q(f,C) = q(f^*g, q(f,B')^*C) \circ q(q(f,B'), C)$ by lemma 1.
$= q(f^*g \circ q(f,B'), C)$, by axiom (II).
$= q(q(f,B) \circ g, C)$, by definition of $f^*g$.
$= q(q(f,B), g^*C) \circ q(g,C)$, by axiom (II).
$= q(f, g^*C) \circ q(g,C)$, by lemma 1.
\end{proof}

Axiom (I) of the theory of contextual categories ensures that if $A$ is an object of the contextual category $\mathbb{C}$ then $\mathbb{C}_{id_A} = id_{\mathbb{C}_A}$. Axiom (II) ensures that whenever $f: A \to A'$ in $\mathbb{C}$ and $f': A' \to A''$ in $\mathbb{C}$ then $\mathbb{C}_{f'} \circ \mathbb{C}_f = \mathbb{C}_{f'f}$. Thus $\mathbb{C}_{-} : \mathbb{C}^{op} \to \underline{Con}$ is a functor.

\begin{definition}
If $F: \mathbb{C} \to \mathbb{C}'$ is a contextual functor then if $A$ is an object of $\mathbb{C}$ let $F_A: \mathbb{C}_A \to \mathbb{C}'_{F(A)}$ be the restriction of $F$ to $\mathbb{C}_A$. $F_A$ is, in fact, a contextual functor.
\end{definition}

It is amusing to note that whenever $F: \mathbb{C} \to \mathbb{C}'$ is a contextual functor then $F_-: \mathbb{C}_- \to \mathbb{C}'_{F(-)}$ is a natural transformation. Thus whenever $F: \mathbb{C} \to \mathbb{C}'$ is a contextual functor then we have the following diagram in the 2-category of categories.
\begin{center}
\begin{tikzcd}
\mathbb{C}^{op} \arrow[r, "\mathbb{C}_-"] \arrow[d, "F"] & \underline{Con} \\
\mathbb{C}'^{op} \arrow[ur, "\mathbb{C}'_-"']
\end{tikzcd}
\end{center}

The remainder of this section is ground work for the proof of the next section of the equivalence between theories and contextual categories. We begin by giving two definitions. With reference to the first of these two definitions we must apologise for the notation for we are not asking that the morphism 'f' be thought of in any way as a quotation of f, the notation is merely convenient.

\begin{definition}
If $A, B \in |\mathbb{C}|$ and $f: A \to B$ then 'f' is the unique morphism from $A$ to $(f \circ p(B))^*B$ such that $\text{'f'} \circ p((f \circ p(B))^*B) = id_A$ and $\text{'f'} \circ q(f \circ p(B), B) = f$.
\end{definition}

\begin{center}
\begin{tikzcd}
& (f \circ p(B))^*B \arrow[r, "q(f \circ p(B){,} B)"] \arrow[d, "p((f \circ p(B))^*B)"] & B \arrow[d, "p(B)"] \\
A \arrow[ur, dashed, "{'f'}"] \arrow[r, "id_A"'] & A \arrow[r, "f \circ p(B)"] & A
\end{tikzcd}
\end{center}

\begin{definition}
If $A \triangleleft B$ in $\mathbb{C}$ then $Arr_{\mathbb{C}}(B) = \{ f: A \to B \mid f \circ p(B) = id_A \}$.
\end{definition}

Note that for all $A, B \in |\mathbb{C}|$, for all $f: A \to B$, $\text{'f'} \in Arr_{\mathbb{C}}((f \circ p(B))^*B)$.

\begin{lemma}
If $U$ is a generalised algebraic theory and if $1 \triangleleft A_1 \dots \triangleleft A_n$, $1 \triangleleft B_1 \dots \triangleleft B_m$ in $\mathbb{C}(U)$ are given by $A_n = [\langle x_1 \in \Delta_1, \dots, x_n \in \Delta_n \rangle]$ and $B_m = [\langle y_1 \in \Omega_1, \dots, y_m \in \Omega_m \rangle]$ then
\begin{enumerate}
    \item If $f: A_n \to B_m$ is given by $f = [\langle t_1, \dots, t_m \rangle]$ then $(f \circ p(B_m))^*B_m = [\langle x_1 \in \Delta_1, \dots, x_n \in \Delta_n, z \in \Omega_m[t_1|y_1, \dots, t_{m-1}|y_{m-1}] \rangle]$ and $\text{'f'} = [\langle x_1, \dots, x_n, t_m \rangle]$.
    \item If $A_n \triangleleft A$ in $\mathbb{C}(U)$ and $A = [\langle x_1 \in \Delta_1, \dots, x_n \in \Delta_n, x \in \Delta \rangle]$ then for any morphism $g$ of $\mathbb{C}(U)$, $g \in Arr_{\mathbb{C}(U)}(A)$ iff $g$ is of the form $[\langle x_1, \dots, x_n, t \rangle]$ for some $t$ such that
    \[ \frac{x_1 \in \Delta_1, \dots, x_n \in \Delta_n}{t \in \Delta} \]
    is a derived rule of $U$.
    \item For any $i$, $1 \le i \le n$, $p(A_n, A_i) = [\langle x_1, \dots, x_i \rangle]$.
\end{enumerate}
\end{lemma}

\begin{proof}
(i) and (ii) follow directly from the definition of $\mathbb{C}(U)$.
(iii) The proof is by induction. Certainly $p(A_n, A_n) = [\langle x_1, \dots, x_n \rangle]$ by definition of $\mathbb{C}(U)$. If we assume that the result holds for $i+1$, that is if we assume that in $\mathbb{C}(U)$, $p(A_n, A_{i+1}) = [\langle x_1, \dots, x_{i+1} \rangle]$, then the result follows for $i$, since $p(A_n, A_i) = p(A_n, A_{i+1}) \circ p(A_{i+1}, A_i) = [\langle x_1, \dots, x_{i+1} \rangle] \circ [\langle x_1, \dots, x_i \rangle] = [\langle x_1, \dots, x_i \rangle]$.
Hence the result holds for all $i$, $1 \le i \le n$.
\end{proof}

\begin{lemma}
(i) If
\begin{center}
\begin{tikzcd}
A \arrow[r, "f_2"] \arrow[dr, "f_1"'] & B_2 \arrow[d] \\
& B_1
\end{tikzcd}
\end{center}
is a commutative diagram in $\mathbb{C}$ then for all $x: X \to X'$ in $\mathbb{C}_{B_2}$, $f_2^*x = \text{'f}_2\text{'}^* f_1^*x$.

(ii) If $A \in |\mathbb{C}|$ and $1 \triangleleft B_1 \dots \triangleleft B_m$ in $\mathbb{C}$ and for each $j$, $1 \le j \le m$, $g_j: A \to B_j$ such that each triangle in the diagram commutes,
\begin{center}
\begin{tikzcd}
& B_m \arrow[d] \\
A \arrow[ur, "g_m"] \arrow[r, "g_{m-1}"] \arrow[dr, "g_1"'] & B_{m-1} \\
& \vdots \arrow[d] \\
& B_1
\end{tikzcd}
\end{center}
then for all $x: X \to X'$ in $\mathbb{C}_{B_m}$, $\text{'g}_m\text{'}^* \dots \text{'g}_1\text{'}^* p(A,1)^*x = g_m^*x$.

(iii) If $A \in |\mathbb{C}|$ and $1 \triangleleft B_1 \dots \triangleleft B_m$ in $\mathbb{C}$ and for each $j$, $1 \le j \le m$, $\alpha_j \in Arr_{\mathbb{C}}(\alpha_{j-1}^* \dots \alpha_1^* p(A,1)^* B_j)$ then there exists a unique sequence of morphisms $g_1, \dots, g_m$ of $\mathbb{C}$ such that for each $j$, $1 \le j \le m$, $g_j: A \to B_j$ such that the diagram
\begin{center}
\begin{tikzcd}
& B_m \arrow[d] \\
A \arrow[ur, "g_m"] \arrow[r, "g_{m-1}"] \arrow[dr, "g_1"'] & B_{m-1} \\
& \vdots \arrow[d] \\
& B_1
\end{tikzcd}
\end{center}
commutes and such that for all $j$, $1 \le j \le m$, $\text{'g}_j\text{'} = \alpha_j$.
\end{lemma}

\begin{proof}
(i) $f_2^*x = (\text{'f}_2\text{'}^* q(f_1, B_2))^*x$, since $f_2 = \text{'f}_2\text{'}^* q(f_1, B_2)$.
$= \text{'f}_2\text{'}^* (q(f_1, B_2)^*x)$, by axiom (II).
$= \text{'f}_2\text{'}^* f_1^*x$, by lemma 1.

(ii) The proof is by induction on $m$.
If $m=1$ then we have
\begin{center}
\begin{tikzcd}
& B_1 \arrow[d] \\
A \arrow[ur, "g_1"] \arrow[r, "p(A{,}1)"'] & 1
\end{tikzcd}
\end{center}
in $\mathbb{C}$, thus by part (i) $g_1^*x = \text{'g}_1\text{'}^* p(A,1)^*x$.
If $m > 1$, if we assume that for all $j$, $1 \le j < m$, $g_j^*x = \text{'g}_j\text{'}^* \dots \text{'g}_1\text{'}^* p(A,1)^*x$, for all $x$ in $\mathbb{C}_{B_j}$. Then since
\begin{center}
\begin{tikzcd}
& B_m \arrow[d] \\
A \arrow[ur, "g_m"] \arrow[r, "g_{m-1}"] & B_{m-1}
\end{tikzcd}
\end{center}
in $\mathbb{C}$ thus by part (i) $g_m^*x = \text{'g}_m\text{'}^* (g_{m-1}^*x) = \text{'g}_m\text{'}^* \text{'g}_{m-1}\text{'}^* \dots \text{'g}_1\text{'}^* p(A,1)^*x$, by inductive hypothesis.

(iii) The proof is by induction on $m$.
If $m=1$ then since $\text{'g}_1\text{'}$ is required to be $\alpha_1$, so we must choose $g_1$ such that $g_1 = \alpha_1 \circ q(p(A,1), B_1)$.
If $m>1$, and if we assume that $g_1, \dots, g_{m-1}$ are such that $g_j \circ p(B_j) = g_{j-1}$, for all $j$, $1 < j \le m-1$ and such that $\text{'g}_j\text{'} = \alpha_j$, for each $j, 1 \le j < m$. Then by part (ii) $\alpha_m: A \to g_{m-1}^*B_m$. So we can and must choose $g_m = \alpha_m \circ q(g_{m-1}, B_m)$.
\end{proof}

\begin{lemma}
(i) If $A \xrightarrow{f} B \xrightarrow{g} C$ in $\mathbb{C}$ then $\text{'f} \circ \text{g'} = \text{'f} \circ \text{g'}$.
(ii) If $A \in |\mathbb{C}|$ and $1 \triangleleft B_1 \dots \triangleleft B_m$ in $\mathbb{C}$ and for each $j$, $1 \le j \le m$, $\beta_j \in Arr_{\mathbb{C}}(\beta_{j-1}^* \dots \beta_1^* p(A,1)^* B_j)$ then for each $j$, $1 \le j \le m$, $\beta_m^* \dots \beta_1^* p(A,1)^* p(B_m, B_j)' = \beta_j$.
(iii) If $A \in |\mathbb{C}|$, $1 \triangleleft B_1 \dots \triangleleft B_m$ and $1 \triangleleft C_1 \dots \triangleleft C_p$ in $\mathbb{C}$, if for all $j$, $1 \le j \le m$, $\beta_j \in Arr_{\mathbb{C}}(\beta_{j-1}^* \dots \beta_1^* p(A,1)^* B_j)$ and for all $k$, $1 \le k \le p$, $\gamma_k \in Arr_{\mathbb{C}}(\gamma_{k-1}^* \dots \gamma_1^* p(B_m, 1)^* C_k)$, then for all $x: X \to X'$ in $\mathbb{C}$,
$(\beta_m^* \dots \beta_1^* p(A,1)^* \gamma_p)^* \dots (\beta_m^* \dots \beta_1^* p(A,1)^* \gamma_1)^* (\beta_m^* \dots \beta_1^* p(A,1)^* p(B_m, 1))^* x = \beta_m^* \dots \beta_1^* p(A,1)^* \gamma_p^* \dots \gamma_1^* p(B_m, 1)^* x$.
\end{lemma}

\begin{proof}
(i) $\text{'f} \circ \text{g'}$ is the unique morphism from $A$ to $(f \circ g \circ p(C))^*C$ such that $\text{'f} \circ \text{g'} \circ p((f \circ g \circ p(C))^*C) = id_A$ and $\text{'f} \circ \text{g'} \circ q(f \circ g \circ p(C), C) = f \circ g$. It thus suffices to show that $f''g'$ is such a morphism.

By definition of $\text{'g'}$, $\text{'g'}: B \to (g \circ p(C))^*C$ such that $\text{'g'} \circ p((g \circ p(C))^*C) = id_B$ and $\text{'g'} \circ q(g \circ p(C), C) = g$. Thus $f''g' : A \to f^*((g \circ p(C))^*C)$, that is to say, $f''g' : A \to (f \circ g \circ p(C))^*C$. Also since $\text{'g'} \circ p((g \circ p(C))^*C) = id_B$, $f''g' \circ f^*p((g \circ p(C))^*C) = id_A$. But $f^*p((g \circ p(C))^*C) = p(f^*((g \circ p(C))^*C)) = p((f \circ g \circ p(C))^*C)$, hence $f''g' \circ p((f \circ g \circ p(C))^*C) = id_A$. Which is one property of $\text{'f} \circ \text{g'}$.

As for the other, from $\text{'g'} \circ q(g \circ p(C), C) = g$ deduce that $f^* \text{'g'} \circ q(g \circ p(C), C) = f \circ g$. But $f^* \text{'g'} = f''g' \circ q(f, (g \circ p(C))^*C)$ and $q(f, (g \circ p(C))^*C) \circ q(g \circ p(C), C) = q(f \circ g \circ p(C), C)$, hence we have $f''g' \circ q(f \circ g \circ p(C), C) = f \circ g$. As required. From the uniqueness of $\text{'f} \circ \text{g'}$ we conclude that $f''g' = \text{'f} \circ \text{g'}$.

(ii) By lemma 3(iii), there corresponds to $\beta_1, \dots, \beta_m$ a sequence of morphisms $g_1, \dots, g_m$ such that
\begin{center}
\begin{tikzcd}
& B_m \arrow[d] \\
A \arrow[ur, "g_m"] \arrow[r, "g_{m-1}"] \arrow[dr, "g_1"'] & B_{m-1} \\
& \vdots \arrow[d] \\
& B_1
\end{tikzcd}
\end{center}
commutes and such that $\text{'g}_j\text{'} = \beta_j$.

By lemma 3(ii) $\beta_m^* \dots \beta_1^* p(A,1)^* \text{'p}(B_m, B_j)\text{'} = g_m^* \text{'p}(B_m, B_j)\text{'}$. By part (i) of this lemma $g_m^* \text{'p}(B_m, B_j)\text{'} = \text{'g}_m \circ p(B_m, B_j)\text{'} = \beta_j$ as required.

(iii) Again we use lemma 3(iii). Corresponding to $(\beta_j)_{1 \le j \le m}$ we have $(g_j)_{1 \le j \le m}$ such that $\text{'g}_j\text{'} = \beta_j$. Corresponding to $(\gamma_k)_{1 \le k \le p}$ we have $(h_k)_{1 \le k \le p}$ such that $\text{'h}_k\text{'} = \gamma_k$.
$(\beta_m^* \dots \beta_1^* p(A,1)^* \gamma_p)^* \dots (\beta_m^* \dots \beta_1^* p(A,1)^* \gamma_1)^* (\beta_m^* \dots \beta_1^* p(A,1)^* p(B_m, 1))^* x$
$= (g_m^* \gamma_p)^* \dots (g_m^* \gamma_1)^* p(A,1)^* x$, by lemma 3(ii).
$= \text{'g}_m \circ h_p\text{'}^* \dots \text{'g}_m \circ h_1\text{'}^* p(A,1)^* x$, by this lemma part (i).
$= g_m^* h_p^* x$, by lemma 3(ii).
$= g_m^* h_p^* x$, by axiom (II).
$= \text{'g}_m\text{'}^* \dots \text{'g}_1\text{'}^* p(A,1)^* \text{'h}_p\text{'}^* \dots \text{'h}_1\text{'}^* p(B_m, 1)^* x$, by lemma 3(ii).
$= \beta_m^* \dots \beta_1^* p(A,1)^* \gamma_p^* \dots \gamma_1^* p(B_m, 1)^* x$. As required.
\end{proof}

\begin{lemma}
If $f: A \to A'$ and $A' \triangleleft B$ in $\mathbb{C}$ then $\text{'q}(f,B)\text{'} = id_{f^*B}$.
\end{lemma}

\begin{proof}
By definition of $\text{'id}_{f^*B}\text{'}$, $\text{'id}_{f^*B}\text{'} \circ q(p(f^*B), f^*B) = id_{f^*B}$.
Hence $\text{'id}_{f^*B}\text{'} \circ q(p(f^*B), f^*B) \circ q(f,B) = q(f,B)$, that is to say
$\text{'id}_{f^*B}\text{'} \circ q(p(f^*B) \circ f, B) = q(f,B)$, that is
$\text{'id}_{f^*B}\text{'} \circ q(q(f,B) \circ p(B), B) = q(f,B)$, that is
$\text{'q}(f,B)\text{'}$.
\end{proof}


\section{Contextual Categories = Generalised Algebraic Theories}

In this section we establish the equivalence between the category \underline{Con} of contextual categories and the category \underline{GAT} of generalised algebraic theories. We split the section into subsections as follows:
1. define a functor $\mathbb{C} : \underline{GAT} \to \underline{Con}$,
2. define a functor $U : \underline{Con} \to \underline{GAT}$,
3. prove that $U \circ \mathbb{C} \cong id_{\underline{GAT}}$,
4. prove that $\mathbb{C} \circ U \cong id_{\underline{Con}}$.

\subsection{Definition of $\mathbb{C} : \underline{GAT} \to \underline{Con}$}

$\mathbb{C}$ has been defined on objects in \S 2.2, if $U$ is a theory then $\mathbb{C}(U)$ is a contextual category.

If $U$ and $U'$ are theories and $[I] : U \to U'$ in \underline{GAT} then define $\mathbb{C}([I]) : \mathbb{C}(U) \to \mathbb{C}(U')$ by
\begin{center}
\begin{tikzcd}
{[\langle x_1 \in \Delta_1{,} \dots{,} x_n \in \Delta_n \rangle]} \arrow[d, "{[\langle t_1{,} \dots{,} t_m \rangle]}"] \arrow[r, "\mathbb{C}({[}I{]})"] & {[\langle x_1 \in \dot{I}(\Delta_1){,} \dots{,} x_n \in \dot{I}(\Delta_n) \rangle]} \arrow[d, "{[\langle \dot{I}(t_1){,} \dots{,} \dot{I}(t_m) \rangle]}"] \\
{[\langle y_1 \in \Omega_1{,} \dots{,} y_m \in \Omega_m \rangle]} \arrow[r] & {[\langle y_1 \in \dot{I}(\Omega_1){,} \dots{,} y_m \in \dot{I}(\Omega_m) \rangle]}
\end{tikzcd}
\end{center}

Then $\mathbb{C}$ is well defined on morphisms because by lemma 1 of \S 1.13 if $I$ and $J$ are interpretations of $U$ in $U'$ and if $I \equiv J$ then for all derived T and $\in$-rules $R$ of $U$, $\hat{I}(R) \equiv \hat{J}(R)$.

To see that $\mathbb{C}([I])$ is well defined and takes objects and morphisms of $\mathbb{C}(U)$ to objects respectively morphisms of $\mathbb{C}(U')$ we require the following:

\begin{lemma}
If $U$ and $U'$ are generalised algebraic theories and if $I$ is an interpretation of $U$ in $U'$ then
\begin{enumerate}
    \item[(i)] If $\langle x_1 \in \Delta_1, \dots, x_n \in \Delta_n \rangle$ is a $U$-context then $\langle x_1 \in \dot{I}(\Delta_1), \dots, x_n \in \dot{I}(\Delta_n) \rangle$ is a $U'$-context.
    \item[(ii)] If $\langle x_1 \in \Delta_1, \dots, x_n \in \Delta_n \rangle$ and $\langle x'_1 \in \Delta'_1, \dots, x'_n \in \Delta'_n \rangle$ are $U$-contexts and $\langle x_1 \in \Delta_1, \dots, x_n \in \Delta_n \rangle \equiv \langle x'_1 \in \Delta'_1, \dots, x'_n \in \Delta'_n \rangle$ then $\langle x_1 \in \dot{I}(\Delta_1), \dots, x_n \in \dot{I}(\Delta_n) \rangle \equiv \langle x'_1 \in \dot{I}(\Delta'_1), \dots, x'_n \in \dot{I}(\Delta'_n) \rangle$.
    \item[(iii)] If $\langle t_1, \dots, t_m \rangle$ is a $U$-realisation of the context $\langle y_1 \in \Omega_1, \dots, y_m \in \Omega_m \rangle$ wrt the context $\langle x_1 \in \Delta_1, \dots, x_n \in \Delta_n \rangle$ then $\langle \dot{I}(t_1), \dots, \dot{I}(t_m) \rangle$ is a $U'$-realisation of $\langle y_1 \in \dot{I}(\Omega_1), \dots, y_m \in \dot{I}(\Omega_m) \rangle$ wrt $\langle x_1 \in \dot{I}(\Delta_1), \dots, x_n \in \dot{I}(\Delta_n) \rangle$.
    \item[(iv)] If $\langle t_1, \dots, t_m \rangle$ is a $U$-realisation of $\langle y_1 \in \Omega_1, \dots, y_m \in \Omega_m \rangle$ wrt $\langle x_1 \in \Delta_1, \dots, x_n \in \Delta_n \rangle$ and if $\langle t'_1, \dots, t'_m \rangle$ is a $U$-realisation of $\langle y'_1 \in \Omega'_1, \dots, y'_m \in \Omega'_m \rangle$ wrt $\langle x'_1 \in \Delta'_1, \dots, x'_n \in \Delta'_n \rangle$ and if $\langle t_1, \dots, t_m \rangle \equiv \langle t'_1, \dots, t'_m \rangle$ then $\langle \dot{I}(t_1), \dots, \dot{I}(t_m) \rangle \equiv \langle \dot{I}(t'_1), \dots, \dot{I}(t'_m) \rangle$.
\end{enumerate}
\end{lemma}

\begin{proof}
(i) Follows immediately from lemma 2 of \S 1.11.

(ii) For each $i$, $1 \le i \le n$, $\frac{x_1 \in \Delta_1, \dots, x_{i-1} \in \Delta_{i-1}}{\Delta_i = \Delta'_i[x_1|x'_1, \dots, x_{i-1}|x'_{i-1}]}$ is a derived rule of $U$, hence by lemma 2 of \S 1.11,
$\frac{x_1 \in \dot{I}(\Delta_1), \dots, x_{i-1} \in \dot{I}(\Delta_{i-1})}{\dot{I}(\Delta_i) = \dot{I}(\Delta'_i[x_1|x'_1, \dots, x_{i-1}|x'_{i-1}])}$ is a derived rule of $U'$. But by lemma 1 of \S 1.11, $\dot{I}(\Delta'_i[x_1|x'_1, \dots, x_{i-1}|x'_{i-1}]) \equiv \dot{I}(\Delta'_i)[x_1|x'_1, \dots, x_{i-1}|x'_{i-1}]$. Thus for each $i$, $1 \le i \le n$,
$\frac{x_1 \in \dot{I}(\Delta_1), \dots, x_{i-1} \in \dot{I}(\Delta_{i-1})}{\dot{I}(\Delta_i) = \dot{I}(\Delta'_i)[x_1|x'_1, \dots, x_{i-1}|x'_{i-1}]}$ is a derived rule of $U'$. That is $\langle x_1 \in \dot{I}(\Delta_1), \dots, x_n \in \dot{I}(\Delta_n) \rangle \equiv \langle x'_1 \in \dot{I}(\Delta'_1), \dots, x'_n \in \dot{I}(\Delta'_n) \rangle$.

(iii) For each $j$, $1 \le j \le m$, $\frac{x_1 \in \Delta_1, \dots, x_n \in \Delta_n}{t_j \in \Omega_j[t_1|y_1, \dots, t_{j-1}|y_{j-1}]}$ is a derived rule of $U$, thus by lemmas 1 and 2 of \S 1.11,
$\frac{x_1 \in \dot{I}(\Delta_1), \dots, x_n \in \dot{I}(\Delta_n)}{\dot{I}(t_j) \in \dot{I}(\Omega_j)[\dot{I}(t_1)|y_1, \dots, \dot{I}(t_{j-1})|y_{j-1}]}$ is a derived rule of $U'$.
Thus $\langle \dot{I}(t_1), \dots, \dot{I}(t_m) \rangle$ is a realisation of $\langle y_1 \in \dot{I}(\Omega_1), \dots, y_m \in \dot{I}(\Omega_m) \rangle$ wrt $\langle x_1 \in \dot{I}(\Delta_1), \dots, x_n \in \dot{I}(\Delta_n) \rangle$.

(iv) For each $j$, $1 \le j \le m$, $\frac{x_1 \in \Delta_1, \dots, x_n \in \Delta_n}{t_j = t'_j[x_1|x'_1, \dots, x_n|x'_n] \in \Omega_j[t_1|y_1, \dots, t_{j-1}|y_{j-1}]}$ is a derived rule of $U$, thus by lemmas 1 and 2 of \S 1.11,
$\frac{x_1 \in \dot{I}(\Delta_1), \dots, x_n \in \dot{I}(\Delta_n)}{\dot{I}(t_j) = \dot{I}(t'_j)[x_1|x'_1, \dots, x_n|x'_n] \in \dot{I}(\Omega_j)[\dot{I}(t_1)|y_1, \dots, \dot{I}(t_{j-1})|y_{j-1}]}$ is a derived rule of $U'$. That is $\langle \dot{I}(t_1), \dots, \dot{I}(t_m) \rangle \equiv \langle \dot{I}(t'_1), \dots, \dot{I}(t'_m) \rangle$.
\end{proof}

So $\mathbb{C}([I])$ is well defined. It remains to show that $\mathbb{C}([I])$ is a contextual functor:

\begin{lemma}
(i) If $A \in |\mathbb{C}(U)|$ then $\mathbb{C}([I])(id_A) = id_{\mathbb{C}([I])(A)}$.
(ii) If $A \xrightarrow{f} B \xrightarrow{g} C$ in $\mathbb{C}(U)$ then $\mathbb{C}([I])(f \circ g) = \mathbb{C}([I])(f) \circ \mathbb{C}([I])(g)$.
(iii) If $A \in |\mathbb{C}|$ and $1 \triangleleft A$ then $\mathbb{C}([I])(p(A)) = p(\mathbb{C}([I])(A))$.
(iv) If
\begin{center}
\begin{tikzcd}
& B \arrow[d] \\
A \arrow[r, "f"] & A'
\end{tikzcd}
\end{center}
in $\mathbb{C}(U)$ then $\mathbb{C}([I])(f^*B) = \mathbb{C}([I])(f)^* \mathbb{C}([I])(B)$ and $\mathbb{C}([I])(q(f,B)) = q(\mathbb{C}([I])(f), \mathbb{C}([I])(B))$.
\end{lemma}

\begin{proof}
(i) and (iii) are both remarkably trivial.

(ii) Suppose that $A = [\langle x_1 \in \Delta_1, \dots, x_n \in \Delta_n \rangle]$, $B = [\langle y_1 \in \Omega_1, \dots, y_m \in \Omega_m \rangle]$ and $C = [\langle z_1 \in \Lambda_1, \dots, z_p \in \Lambda_p \rangle]$, $f = [\langle t_1, \dots, t_m \rangle]$ and $g = [\langle s_1, \dots, s_p \rangle]$, where $\langle t_1, \dots, t_m \rangle$ is a realisation of $\langle y_1 \in \Omega_1, \dots, y_m \in \Omega_m \rangle$ wrt $\langle x_1 \in \Delta_1, \dots, x_n \in \Delta_n \rangle$ and $\langle s_1, \dots, s_p \rangle$ is a realisation of $\langle z_1 \in \Lambda_1, \dots, z_p \in \Lambda_p \rangle$ wrt $\langle y_1 \in \Omega_1, \dots, y_m \in \Omega_m \rangle$. If we use the definition of $\mathbb{C}([I])$, the definition of composition in $\mathbb{C}(U)$ and $\mathbb{C}(U')$, and lemma 1 of \S 1.11 then:
$\mathbb{C}([I])(f \circ g) = \mathbb{C}([I])([\langle s_1[t_1|y_1, \dots, t_m|y_m], \dots, s_p[t_1|y_1, \dots, t_m|y_m] \rangle])$
$= [\langle \dot{I}(s_1[t_1|y_1, \dots, t_m|y_m]), \dots, \dot{I}(s_p[t_1|y_1, \dots, t_m|y_m]) \rangle]$
$= [\langle \dot{I}(s_1)[\dot{I}(t_1)|y_1, \dots, \dot{I}(t_m)|y_m], \dots, \dot{I}(s_p)[\dot{I}(t_1)|y_1, \dots, \dot{I}(t_m)|y_m] \rangle]$
$= [\langle \dot{I}(t_1), \dots, \dot{I}(t_m) \rangle] \circ [\langle \dot{I}(s_1), \dots, \dot{I}(s_p) \rangle] = \mathbb{C}([I])(f) \circ \mathbb{C}([I])(g)$.

(iv) Suppose that $A = [\langle x_1 \in \Delta_1, \dots, x_n \in \Delta_n \rangle]$, $B = [\langle y_1 \in \Omega_1, \dots, y_m \in \Omega_m, y \in \Omega \rangle]$ and $f = [\langle t_1, \dots, t_m \rangle]$, where $\langle t_1, \dots, t_m \rangle$ is a realisation of $\langle y_1 \in \Omega_1, \dots, y_m \in \Omega_m \rangle$ wrt $\langle x_1 \in \Delta_1, \dots, x_n \in \Delta_n \rangle$.

$\mathbb{C}([I])(f^*B) = \mathbb{C}([I])([\langle x_1 \in \Delta_1, \dots, x_n \in \Delta_n, y \in \Omega[t_1|y_1, \dots, t_m|y_m] \rangle])$
$= [\langle x_1 \in \dot{I}(\Delta_1), \dots, x_n \in \dot{I}(\Delta_n), y \in \dot{I}(\Omega)[\dot{I}(t_1)|y_1, \dots, \dot{I}(t_m)|y_m] \rangle]$
$= [\langle \dot{I}(t_1), \dots, \dot{I}(t_m) \rangle]^* [\langle y_1 \in \dot{I}(\Omega_1), \dots, y_m \in \dot{I}(\Omega_m), y \in \dot{I}(\Omega) \rangle] = \mathbb{C}([I])(f)^* \mathbb{C}([I])(B)$.

$\mathbb{C}([I])(q(f,B)) = \mathbb{C}([I])([\langle t_1, \dots, t_m, y \rangle]) = [\langle \dot{I}(t_1), \dots, \dot{I}(t_m), y \rangle] = q(\mathbb{C}([I])(f), \mathbb{C}([I])(B))$.
\end{proof}

Supposing that $I$ is an interpretation of $U$ in $U'$ and that $I'$ is an interpretation of $U'$ in $U''$, then by lemma 3 of \S 1.11 for any expression $e$ of $U$, $\dot{(I' \circ I)}(e) = \dot{I'}(\dot{I}(e))$. Thus it follows that in this section $\mathbb{C}([I']) \circ \mathbb{C}([I]) = \mathbb{C}([I' \circ I]) = \mathbb{C}([I'] \circ [I])$. So for sure $\mathbb{C} : \underline{GAT} \to \underline{Con}$ is a functor.

\subsection{The definition of $U : \underline{Con} \to \underline{GAT}$}

It is convenient to say that an object $A$ of a contextual category $\mathbb{C}$ is trivial just in case $A$ is the least element $1$ of $\mathbb{C}$. Similarly we say that a morphism of $\mathbb{C}$ is trivial just in case its codomain is the trivial object.

We begin by describing the functor $U : \underline{Con} \to \underline{GAT}$ on objects of \underline{Con}. If $\mathbb{C}$ is a contextual category then $U(\mathbb{C})$ is the generalised algebraic theory described as follows: $U(\mathbb{C})$ has a sort symbol $\overline{A}$ for every non-trivial object $A$ of $\mathbb{C}$. $U(\mathbb{C})$ has an operator symbol $\overline{f}$ for every non-trivial morphism $f$ of $\mathbb{C}$. If $1 \triangleleft A_1 \dots \triangleleft A_n \triangleleft A$ in $\mathbb{C}$ then the introductory rule for $\overline{A}$ in $U(\mathbb{C})$ is
\[ \frac{x_1 \in \overline{A_1}, \dots, x_n \in \overline{A_n}(x_1, \dots, x_{n-1})}{\overline{A}(x_1, \dots, x_n) \text{ is a type}} \]
If $1 \triangleleft A_1 \dots \triangleleft A_n$ in $\mathbb{C}$ and $f: A_n \to B$ where $1 < B$ then the introductory rule for $\overline{f}$ in $U(\mathbb{C})$ is
\[ \frac{x_1 \in \overline{A_1}, \dots, x_n \in \overline{A_n}(x_1, \dots, x_{n-1})}{\overline{f}(x_1, \dots, x_n) \in \overline{(f \circ p(B))^*B}(x_1, \dots, x_n)} \]

The axioms of $U(\mathbb{C})$ arise from three different situations, $U(\mathbb{C})$ has just the following axioms:

(i) For $n \ge 0$, $m \ge 1$ and $l \ge 0$, if $1 \triangleleft A_1 \dots \triangleleft A_n$, $1 \triangleleft B_1 \dots \triangleleft B_m$ and $1 \triangleleft C_1 \dots \triangleleft C_l$ in $\mathbb{C}$ and if $f: A_n \to B_m$ and $g: B_m \to C_l$ in $\mathbb{C}$ then $U(\mathbb{C})$ has the axiom
\[ \frac{x_1 \in \overline{A_1}, \dots, x_n \in \overline{A_n}(x_1, \dots, x_{n-1})}{\overline{f \circ g}(x_1, \dots, x_n) = \overline{g}(\overline{f \circ p(B_m, B_1)}(x_1, \dots, x_n), \dots, \overline{f \circ p(B_m, B_{m-1})}(x_1, \dots, x_n), \overline{f}(x_1, \dots, x_n)) \in \overline{(f \circ g \circ p(C_l))^*C_l}(x_1, \dots, x_n)} \]

(ii) For $n \ge 0$, if $1 \triangleleft A_1 \dots \triangleleft A_n$ in $\mathbb{C}$ then for each $i$, $1 \le i \le n$, $U(\mathbb{C})$ has the axiom
\[ \frac{x_1 \in \overline{A_1}, \dots, x_n \in \overline{A_n}(x_1, \dots, x_{n-1})}{\overline{p(A_n, A_i)}(x_1, \dots, x_n) = x_i \in \overline{A_i}(x_1, \dots, x_{i-1})} \]

(iii) For $n \ge 0$, $m \ge 1$, if $1 \triangleleft A_1 \dots \triangleleft A_n$ and $1 \triangleleft B_1 \dots \triangleleft B_m \triangleleft B$ in $\mathbb{C}$ and if $f: A_n \to B_m$ then $U(\mathbb{C})$ has the axioms
\[ \frac{x_1 \in \overline{A_1}, \dots, x_n \in \overline{A_n}(x_1, \dots, x_{n-1})}{\overline{f^*B}(x_1, \dots, x_n) = \overline{B}(\overline{f \circ p(B_m, B_1)}(x_1, \dots, x_n), \dots, \overline{f \circ p(B_m, B_{m-1})}(x_1, \dots, x_n), \overline{f}(x_1, \dots, x_n))} \]
and
\[ \frac{x_1 \in \overline{A_1}, \dots, x_n \in \overline{A_n}(x_1, \dots, x_{n-1}), y \in \overline{f^*B}(x_1, \dots, x_n)}{\overline{q(f,B)}(x_1, \dots, x_n, y) = y \in \overline{f^*B}(x_1, \dots, x_n)} \]
This completes the definition of $U(\mathbb{C})$.

As for the action of the functor $U$ on morphisms, if $F: \mathbb{C} \to \mathbb{C}'$ is a contextual functor then define a preinterpretation $U(F)$ of $U(\mathbb{C})$ in $U(\mathbb{C}')$ as follows: If $1 \triangleleft A_1 \dots \triangleleft A_n \triangleleft A$ in $\mathbb{C}$ then define $U(F)(\overline{A}) = \overline{F(A)}(v_1, \dots, v_n)$, if $1 \triangleleft A_1 \dots \triangleleft A_n$ in $\mathbb{C}$ and $f: A_n \to B$ then define $U(F)(\overline{f}) = \overline{F(f)}(v_1, \dots, v_n)$. $v_1, v_2, \dots$ is supposed to be the standard enumeration of the set $V$ of variables, see \S 1.11.

\begin{lemma}
$U(F)$ is an interpretation of $U(\mathbb{C})$ in $U(\mathbb{C}')$.
\end{lemma}

\begin{proof}
We have to check that for every introductory rule or axiom $R$ of $U(\mathbb{C})$, the rule $\widehat{U(F)}(R)$ is a derived rule of $U(\mathbb{C}')$. But it happens that in each case $\widehat{U(F)}(R)$ is actually an introductory rule or an axiom of $U(\mathbb{C}')$ and thus a derived rule. Thus there is little work to be done.

For example, if $1 \triangleleft A_1 \dots \triangleleft A_n \triangleleft A$ in $\mathbb{C}$, so that $U(\mathbb{C})$ has the introductory rule $\frac{x_1 \in \overline{A_1}, \dots, x_n \in \overline{A_n}(x_1, \dots, x_{n-1})}{\overline{A}(x_1, \dots, x_n) \text{ is a type}}$ then
\[ \widehat{I(F)}\left(\frac{x_1 \in \overline{A_1}, \dots, x_n \in \overline{A_n}(x_1, \dots, x_{n-1})}{\overline{A}(x_1, \dots, x_n) \text{ is a type}}\right) = \]
\[ \frac{x_1 \in \dot{U(F)}(\overline{A_1}), \dots, x_n \in \dot{U(F)}(\overline{A_n}(x_1, \dots, x_{n-1}))}{\dot{U(F)}(\overline{A}(x_1, \dots, x_n)) \text{ is a type}} = \]
\[ \frac{x_1 \in U(F)(\overline{A_1}), \dots, x_n \in U(F)(\overline{A_n})[x_1|v_1, \dots, x_{n-1}|v_{n-1}]}{U(F)(\overline{A})[x_1|v_1, \dots, x_n|v_n] \text{ is a type}} = \]
\[ \frac{x_1 \in \overline{F(A_1)}, \dots, x_n \in \overline{F(A_n)}(x_1, \dots, x_{n-1})}{\overline{F(A)}(x_1, \dots, x_n) \text{ is a type}} \]
which is of course the introductory rule for $\overline{F(A)}$ in $U(\mathbb{C}')$.
\end{proof}

If $F: \mathbb{C} \to \mathbb{C}'$ and $F': \mathbb{C}' \to \mathbb{C}''$ in \underline{Con} then $U(F' \circ F) = U(F') \circ U(F)$. This is because for any symbol $\overline{L}$ of $U(\mathbb{C})$ and for appropriate $n$,
$U(F' \circ F)(\overline{L}) = \overline{F'(F(L))}(v_1, \dots, v_n) = \dot{U(F')}(\overline{F(L)}(v_1, \dots, v_n)) = \dot{U(F')}(U(F)(\overline{L})) = (U(F') \circ U(F))(\overline{L})$. Thus we have defined a functor from the category \underline{Con} to the category of generalised algebraic theories and interpretations. By taking the value of $U$ at $F$ to be $[U(F)]$ we get a functor $U : \underline{Con} \to \underline{GAT}$.


\subsection{The proof that $U \circ \mathbb{C} \cong id_{GAT}$}

For every generalised algebraic theory $U$ we define an interpretation $\phi_U$ of $U$ in $U(\mathbb{C}(U))$. We show that $[\phi_-]$ (i.e. $\lambda U \in |GAT| . [\phi_U]$) is a natural transformation $[\phi_-] : id_{GAT} \to U \circ \mathbb{C}$. For every theory $U$ we define an interpretation $\psi_U$ of $U(\mathbb{C}(U))$ in $U$ and show that $[\phi_U] \circ [\psi_U] = id_{U(\mathbb{C}(U))}$ and that $[\psi_U] \circ [\phi_U] = id_U$.

If $U$ is a theory then the preinterpretation $\phi_U$ of $U$ in $U(\mathbb{C}(U))$ is defined as follows: If $A$ is a sort symbol of $U$ introduced by the rule $\frac{x_1 \in \Delta_1, \dots, x_n \in \Delta_n}{A(x_1, \dots, x_n) \text{ is a type}}$ then define
\[ \phi_U(A) = \overline{[\langle x_1 \in \Delta_1, \dots, x_n \in \Delta_n, x \in A(x_1, \dots, x_n) \rangle]}(v_1, \dots, v_n). \]
If $f$ is an operator symbol of $U$ introduced by the rule $\frac{x_1 \in \Delta_1, \dots, x_n \in \Delta_n}{f(x_1, \dots, x_n) \in \Delta}$ then define
\[ \phi_U(f) = \overline{[\langle x_1, \dots, x_n, f(x_1, \dots, x_n) \rangle]}(v_1, \dots, v_n). \]

When it is unlikely to lead to misunderstanding then we drop the subscript $U$ from $\phi_U$. We wish to show that for any theory $U$, the preinterpretation $\phi$ of $U$ in $U(\mathbb{C}(U))$ is actually an interpretation. This requires a long string of lemmas. We do in fact show that for any derived rule of $U$ of the form $\frac{x_1 \in \Delta_1, \dots, x_n \in \Delta_n}{\Delta \text{ is a type}}$,
\[ \hat{\phi}\left(\frac{x_1 \in \Delta_1, \dots, x_n \in \Delta_n}{\Delta \text{ is a type}}\right) \]
\[ \equiv \frac{v_1 \in \overline{[\langle x_1 \in \Delta_1 \rangle]}, \dots, v_n \in \overline{[\langle x_1 \in \Delta_1, \dots, x_n \in \Delta_n \rangle]}(v_1, \dots, v_{n-1})}{\overline{[\langle x_1 \in \Delta_1, \dots, x_n \in \Delta_n, x \in \Delta \rangle]}(v_1, \dots, v_n) \text{ is a type}} \]
and we show that for any derived rule of the form $\frac{x_1 \in \Delta_1, \dots, x_n \in \Delta_n}{t \in \Delta}$
\[ \hat{\phi}\left(\frac{x_1 \in \Delta_1, \dots, x_n \in \Delta_n}{t \in \Delta}\right) \equiv \]
\[ \frac{v_1 \in \overline{[\langle x_1 \in \Delta_1 \rangle]}, \dots, v_n \in \overline{[\langle x_1 \in \Delta_1, \dots, x_n \in \Delta_n \rangle]}(v_1, \dots, v_{n-1})}{\overline{[\langle x_1, \dots, x_n, t \rangle]}(v_1, \dots, v_n) \in \overline{[\langle x_1 \in \Delta_1, \dots, x_n \in \Delta_n, x \in \Delta \rangle]}(v_1, \dots, v_n)} \]
From which it follows of course that $\phi$ is an interpretation.

\begin{lemma}
If $\mathbb{C}$ is a contextual category and if for some $n \ge 0$ and $m \ge 1$, $1 \triangleleft A_1 \dots \triangleleft A_n$, $1 \triangleleft B_1 \dots \triangleleft B_m$ and $f: A_n \to B_m$ in $\mathbb{C}$ then the rule
\[ \frac{x_1 \in \overline{A_1}, \dots, x_n \in \overline{A_n}(x_1, \dots, x_{n-1})}{\overline{f}(x_1, \dots, x_n) \in \overline{B_m}(\overline{f \circ p(B_m, B_1)}(x_1, \dots, x_n), \dots, \overline{f \circ p(B_m, B_{m-1})}(x_1, \dots, x_n))} \]
is a derived rule of $U(\mathbb{C})$.
\end{lemma}

\begin{proof}
The rule can be derived by the principle T1 from the introductory rule of $\overline{f}$ and the axiom of kind (iii) which is
\[ \frac{x_1 \in \overline{A_1}, \dots, x_n \in \overline{A_n}(x_1, \dots, x_{n-1})}{\overline{f \circ p(B_m)^* B_m}(x_1, \dots, x_n) = \overline{B_m}(\overline{f \circ p(B_m, B_1)}(x_1, \dots, x_n), \dots, \overline{f \circ p(B_m, B_{m-1})}(x_1, \dots, x_n))} \]
\end{proof}

\begin{lemma}
If $\mathbb{C}$ is a contextual category and if for some $n \ge 1, m \ge 1$ and $q \ge 0$, $1 \triangleleft A_1 \dots \triangleleft A_n$, $1 \triangleleft B_1 \dots \triangleleft B_m$, $1 \triangleleft C_1 \dots \triangleleft C_q$, $f: A_n \to B_m$, $g_1: C_q \to A_n$ and $g_2: C_q \to B$ in $\mathbb{C}$ such that the diagram
\begin{center}
\begin{tikzcd}
C_q \arrow[r, "g_2"] \arrow[d, "g_1"'] & B \arrow[d] \\
A_n \arrow[r, "f"] & B_m
\end{tikzcd}
\end{center}
commutes, and if $h: C_q \to f^*B$ is the unique morphism such that
\begin{center}
\begin{tikzcd}
C_q \arrow[dr, "h", dashed] \arrow[ddr, "g_1"'] \arrow[drr, "g_2"] & & \\
& f^*B \arrow[r, "q(f{,}B)"] \arrow[d] & B \arrow[d] \\
& A_n \arrow[r, "f"] & B_m
\end{tikzcd}
\end{center}
commute, then the rule
\[ \frac{z_1 \in \overline{C_1}, \dots, z_q \in \overline{C_q}(z_1, \dots, z_{q-1})}{\overline{h}(z_1, \dots, z_q) = \overline{g_2}(z_1, \dots, z_q) \in \overline{(g_1 \circ f)^*B}(z_1, \dots, z_q)} \]
is a derived rule of $U(\mathbb{C})$.
\end{lemma}

\begin{proof}
From Lemma 1 it follows that for each $i, 1 \le i \le n$ the rule
\[ \frac{z_1 \in \overline{C_1}, \dots, z_q \in \overline{C_q}(z_1, \dots, z_{q-1})}{\overline{h \circ p(f^*B, A_i)}(z_1, \dots, z_q) \in \overline{A_i}(\overline{h \circ p(f^*B, A_1)}(z_1, \dots, z_q), \dots, \overline{h \circ p(f^*B, A_{i-1})}(z_1, \dots, z_q))} \]
is a derived rule of $U(\mathbb{C})$. By the same lemma so is
\[ \frac{z_1 \in \overline{C_1}, \dots, z_q \in \overline{C_q}(z_1, \dots, z_{q-1})}{\overline{h}(z_1, \dots, z_q) \in \overline{f^*B}(\overline{h \circ p(f^*B, A_1)}(z_1, \dots, z_q), \dots, \overline{h \circ p(f^*B, A_n)}(z_1, \dots, z_q))} \]
Hence we can substitute this $n+1$-tuple of terms into the axiom
\[ \frac{x_1 \in \overline{A_1}, \dots, x_n \in \overline{A_n}(x_1, \dots, x_{n-1}), y \in \overline{f^*B}(x_1, \dots, x_n)}{\overline{q(f,B)}(x_1, \dots, x_n, y) \in \overline{f^*B}(x_1, \dots, x_n)} \]
\[ \frac{z_1 \in \overline{C_1}, \dots, z_q \in \overline{C_q}(z_1, \dots, z_{q-1})}{\overline{q(f,B)}(\overline{h \circ p(f^*B, A_1)}(z_1, \dots, z_q), \dots, \overline{h \circ p(f^*B, A_n)}(z_1, \dots, z_q), \overline{h}(z_1, \dots, z_q)) = \overline{h}(z_1, \dots, z_q)} \]
is a derived rule of $U(\mathbb{C})$.

But since $h \circ q(f,B) = g_2$ (and also using $g_2 \circ p(B, B_m) = g_1 \circ f$) the rule
\[ \frac{z_1 \in \overline{C_1}, \dots, z_q \in \overline{C_q}(z_1, \dots, z_{q-1})}{\overline{g_2}(z_1, \dots, z_q) = \overline{q(f,B)}(\overline{h \circ p(f^*B, A_1)}(z_1, \dots, z_q), \dots, \overline{h \circ p(f^*B, A_n)}(z_1, \dots, z_q), \overline{h}(z_1, \dots, z_q))} \]
\[ \in \overline{(g_1 \circ f)^*B}(z_1, \dots, z_q) \]
is an axiom of $U(\mathbb{C})$. Thus by transitivity of $=$,
\[ \frac{z_1 \in \overline{C_1}, \dots, z_q \in \overline{C_q}(z_1, \dots, z_{q-1})}{\overline{h}(z_1, \dots, z_q) = \overline{g_2}(z_1, \dots, z_q) \in \overline{(g_1 \circ f)^*B}(z_1, \dots, z_q)} \]
is a derived rule of $U(\mathbb{C})$.
\end{proof}

As a special case of lemma 2 we have:

\begin{corollary}
If $\mathbb{C}$ is a contextual category and if for some $n \ge 0$, $1 \triangleleft A_1 \dots \triangleleft A_n$ and $f: A_n \to B$ in $\mathbb{C}$ then the rule
\[ \frac{x_1 \in \overline{A_1}, \dots, x_n \in \overline{A_n}(x_1, \dots, x_{n-1})}{\overline{'f'}(x_1, \dots, x_n) = \overline{f}(x_1, \dots, x_n) \in \overline{(f \circ p(B))^*B}(x_1, \dots, x_n)} \]
is a derived rule of $U(\mathbb{C})$.
\end{corollary}

If we use the description lemma 2 of \S 2.3 of the contextual category $\mathbb{C}(U)$ associated with a theory $U$ then corollary 3 can be rewritten as follows:

\begin{corollary}
If $\langle x_1 \in \Delta_1, \dots, x_n \in \Delta_n \rangle$ and $\langle y_1 \in \Omega_1, \dots, y_m \in \Omega_m, y \in \Omega \rangle$ are contexts of a theory $U$ and if $\langle t_1, \dots, t_m, t \rangle$ is a realisation of $\langle y_1 \in \Omega_1, \dots, y_m \in \Omega_m, y \in \Omega \rangle$ wrt $\langle x_1 \in \Delta_1, \dots, x_n \in \Delta_n \rangle$ then the rule
\[ \frac{v_1 \in \overline{A_1}, \dots, v_n \in \overline{A_n}(v_1, \dots, v_{n-1})}{\overline{f}(v_1, \dots, v_n) = \overline{g}(v_1, \dots, v_n) \in \overline{C}(v_1, \dots, v_n)} \]
is a derived rule of $U(\mathbb{C}(U))$, where $A_i = [\langle x_1 \in \Delta_1, \dots, x_i \in \Delta_i \rangle]$, $f = [\langle t_1, \dots, t_m, t \rangle]$, $g = [\langle x_1, \dots, x_n, t \rangle]$ and $C = [\langle x_1 \in \Delta_1, \dots, x_n \in \Delta_n, y \in \Omega[t_1|y_1, \dots, t_m|y_m] \rangle]$.
\end{corollary}

\begin{lemma}
If $\mathbb{C}$ is a contextual category and for some $n \ge 0$ and $m \ge 1$, $1 \triangleleft A_1 \dots \triangleleft A_n$, $1 \triangleleft B_1 \dots \triangleleft B_m \triangleleft B$ and $f: A_n \to B_m$ in $\mathbb{C}$, then (i) the rule
\[ \frac{x_1 \in \overline{A_1}, \dots, x_n \in \overline{A_n}(x_1, \dots, x_{n-1})}{\overline{f^*B}(x_1, \dots, x_n) = \overline{B}(\overline{'f \circ p(B_m, B_1)'}(x_1, \dots, x_n), \dots, \overline{'f \circ p(B_m, B_{m-1})'}(x_1, \dots, x_n), \overline{'f'}(x_1, \dots, x_n))} \]
is a derived rule of $U(\mathbb{C})$. (ii) If also $g \in Arr_{\mathbb{C}}(B)$ then the rule
\[ \frac{x_1 \in \overline{A_1}, \dots, x_n \in \overline{A_n}(x_1, \dots, x_{n-1})}{\overline{'f \circ g'}(x_1, \dots, x_n) = \overline{'g'}(\overline{'f \circ p(B_m, B_1)'}(x_1, \dots, x_n), \dots, \overline{'f \circ p(B_m, B_{m-1})'}(x_1, \dots, x_n), \overline{'f'}(x_1, \dots, x_n)) \in \overline{f^*B}(x_1, \dots, x_n)} \]
is a derived rule of $U(\mathbb{C})$.
\end{lemma}

\begin{proof}
Both (i) and (ii) follow from corollary 3. For example (ii) follows from the axiom
\[ \frac{x_1 \in \overline{A_1}, \dots, x_n \in \overline{A_n}(x_1, \dots, x_{n-1})}{\overline{f \circ g}(x_1, \dots, x_n) = \overline{g}(\overline{f \circ p(B_m, B_1)}(x_1, \dots, x_n), \dots, \overline{f \circ p(B_m, B_{m-1})}(x_1, \dots, x_n), \overline{f}(x_1, \dots, x_n)) \in \overline{(f \circ g \circ p(C))^* C}(x_1, \dots, x_n)} \]
since by corollary 3 we have as derived rules of $U(\mathbb{C})$ the rule
\[ \frac{x_1 \in \overline{A_1}, \dots, x_n \in \overline{A_n}(x_1, \dots, x_{n-1})}{\overline{'f \circ g'}(x_1, \dots, x_n) = \overline{f \circ g}(x_1, \dots, x_n) \in \overline{f^*B}(x_1, \dots, x_n)} \]
(since $f \circ g \circ p(B) = f \circ id_B = f$) and for each $j$, $1 \le j \le m$, the rule
\[ \frac{x_1 \in \overline{A_1}, \dots, x_n \in \overline{A_n}(x_1, \dots, x_{n-1})}{\overline{'f \circ p(B_m, B_j)'}(x_1, \dots, x_n) = \overline{f \circ p(B_m, B_j)}(x_1, \dots, x_n) \in \overline{(f \circ p(B_m, B_{j-1}))^* B_j}(x_1, \dots, x_n)} \]
\end{proof}

In particular if $\mathbb{C} = \mathbb{C}(U)$ then we get the following:

\begin{corollary}
If $U$ is a theory, if for some $n \ge 0, m \ge 1$, $\langle x_1 \in \Delta_1, \dots, x_n \in \Delta_n \rangle$, $\langle y_1 \in \Omega_1, \dots, y_m \in \Omega_m, y \in \Omega \rangle$ are contexts of $U$ and $\langle t_1, \dots, t_m \rangle$ is a realisation of $\langle y_1 \in \Omega_1, \dots, y_m \in \Omega_m \rangle$ wrt $\langle x_1 \in \Delta_1, \dots, x_n \in \Delta_n \rangle$ then (i) the rule
\[ \frac{v_1 \in \overline{A_1}, \dots, v_n \in \overline{A_n}(v_1, \dots, v_{n-1})}{\overline{C}(v_1, \dots, v_n) = \overline{B}(\overline{g_1}(v_1, \dots, v_n), \dots, \overline{g_m}(v_1, \dots, v_n))} \]
is a derived rule of $U(\mathbb{C}(U))$, where $A_i = [\langle x_1 \in \Delta_1, \dots, x_i \in \Delta_i \rangle]$, $C = [\langle x_1 \in \Delta_1, \dots, x_n \in \Delta_n, y \in \Omega[t_1|y_1, \dots, t_m|y_m] \rangle]$, $B = [\langle y_1 \in \Omega_1, \dots, y_m \in \Omega_m, y \in \Omega \rangle]$ and $g_j = [\langle x_1, \dots, x_n, t_j \rangle]$.
(ii). If $\frac{y_1 \in \Omega_1, \dots, y_m \in \Omega_m}{t \in \Omega}$ is a derived rule of $U$ the rule
\[ \frac{v_1 \in \overline{A_1}, \dots, v_n \in \overline{A_n}(v_1, \dots, v_{n-1})}{\overline{f}(v_1, \dots, v_n) = \overline{h}(\overline{g_1}(v_1, \dots, v_n), \dots, \overline{g_m}(v_1, \dots, v_n)) \in \overline{C}(v_1, \dots, v_n)} \]
is a derived rule of $U(\mathbb{C}(U))$, where $f = [\langle x_1, \dots, x_n, t[t_1|y_1, \dots, t_m|y_m] \rangle]$ and $h = [\langle y_1, \dots, y_m, t \rangle]$.
\end{corollary}

\begin{lemma}
If $U$ is a theory then (i) for every derived T-rule $\frac{x_1 \in \Delta_1, \dots, x_n \in \Delta_n}{\Delta \text{ is a type}}$ of $U$ the rule
\[ \frac{x_1 \in \overline{A_1}, \dots, x_n \in \overline{A_n}(x_1, \dots, x_{n-1})}{\overline{A}(x_1, \dots, x_n) = \hat{\phi}(\Delta)} \]
is a derived rule of $U(\mathbb{C}(U))$, where $A_i = [\langle x_1 \in \Delta_1, \dots, x_i \in \Delta_i \rangle]$ and $A = [\langle x_1 \in \Delta_1, \dots, x_n \in \Delta_n, x \in \Delta \rangle]$.
(ii) For every derived $\in$-rule $\frac{x_1 \in \Delta_1, \dots, x_n \in \Delta_n}{t \in \Delta}$ of $U$ the rule
\[ \frac{x_1 \in \overline{A_1}, \dots, x_n \in \overline{A_n}(x_1, \dots, x_{n-1})}{\overline{[\langle x_1, \dots, x_n, t \rangle]}(x_1, \dots, x_n) = \hat{\phi}(t) \in \overline{A}(x_1, \dots, x_n)} \]
is a derived rule of $U(\mathbb{C}(U))$.
\end{lemma}

\begin{proof}
The proof is by induction on derivations in $U$. We wish to show that all the derived T and $\in$-rules of $U$ have a certain property so we just show that any rule that is derived from rules that have the property must itself have the property. We must check the principles of derivation T1, CF1 and CF2.

\underline{T1.} Suppose that we derive the rule $\frac{x_1 \in \Delta_1, \dots, x_n \in \Delta_n}{t \in \Delta'}$ from the rules $\frac{x_1 \in \Delta_1, \dots, x_n \in \Delta_n}{t \in \Delta}$ and $\frac{x_1 \in \Delta_1, \dots, x_n \in \Delta_n}{\Delta = \Delta'}$, and suppose also that $\frac{x_1 \in \Delta_1, \dots, x_n \in \Delta_n}{t \in \Delta}$ has the property, which is to say suppose that
\[ \frac{x_1 \in \overline{A_1}, \dots, x_n \in \overline{A_n}(x_1, \dots, x_{n-1})}{\overline{[\langle x_1, \dots, x_n, t \rangle]}(x_1, \dots, x_n) = \hat{\phi}(t) \in \overline{A}(x_1, \dots, x_n)} \]
is a derived rule of $U(\mathbb{C}(U))$. We wish to show that $\frac{x_1 \in \Delta_1, \dots, x_n \in \Delta_n}{t \in \Delta'}$ has the property i.e. that
\[ \frac{x_1 \in \overline{A_1}, \dots, x_n \in \overline{A_n}(x_1, \dots, x_{n-1})}{\overline{[\langle x_1, \dots, x_n, t \rangle]}(x_1, \dots, x_n) = \hat{\phi}(t) \in \overline{A'}(x_1, \dots, x_n)} \]
is a derived rule of $U(\mathbb{C}(U))$, where $A' = [\langle x_1 \in \Delta_1, \dots, x_n \in \Delta_n, x \in \Delta' \rangle]$. But of course it is, because $\frac{x_1 \in \Delta_1, \dots, x_n \in \Delta_n}{\Delta = \Delta'}$ is a derived rule implies $A=A'$.

\underline{CF1.} Suppose that $\frac{x_1 \in \Delta_1, \dots, x_n \in \Delta_n}{\Delta_{n+1} \text{ is a type}}$ is a derived rule of $U$ such that
\[ \frac{x_1 \in \overline{A_1}, \dots, x_n \in \overline{A_n}(x_1, \dots, x_{n-1})}{\overline{A_{n+1}}(x_1, \dots, x_n) = \hat{\phi}(\Delta_{n+1})} \]
is a derived rule of $U(\mathbb{C}(U))$. We must show that for each $i$, $1 \le i \le n+1$, the rule
\[ \frac{x_1 \in \overline{A_1}, \dots, x_{n+1} \in \overline{A_{n+1}}(x_1, \dots, x_n)}{\overline{[\langle x_1, \dots, x_{n+1}, x_i \rangle]}(x_1, \dots, x_{n+1}) = x_i \in \overline{C_i}(x_1, \dots, x_{n+1})} \]
is a derived rule of $U(\mathbb{C}(U))$, where $C_i = [\langle x_1 \in \Delta_1, \dots, x_{n+1} \in \Delta_{n+1}, y \in \Delta_i \rangle]$.

This follows because $\langle x_1 \in \Delta_1, \dots, x_{n+1} \in \Delta_{n+1} \rangle$ and $\langle x_1 \in \Delta_1, \dots, x_i \in \Delta_i \rangle$ are contexts of $U$ and $\langle x_1, \dots, x_i \rangle$ is a realisation of $\langle x_1 \in \Delta_1, \dots, x_i \in \Delta_i \rangle$ wrt $\langle x_1 \in \Delta_1, \dots, x_{n+1} \in \Delta_{n+1} \rangle$, thus by corollary 4 the rule
\[ \frac{x_1 \in \overline{A_1}, \dots, x_{n+1} \in \overline{A_{n+1}}(x_1, \dots, x_n)}{\overline{[\langle x_1, \dots, x_i \rangle]}(x_1, \dots, x_{n+1}) = [\langle x_1, \dots, x_{n+1}, x_i \rangle](x_1, \dots, x_{n+1}) \in \overline{C_i}(x_1, \dots, x_{n+1})} \]
is a derived rule of $U(\mathbb{C}(U))$; and because, since $p(A_{n+1}, A_i)$ in $\mathbb{C}(U)$ is just $[\langle x_1, \dots, x_i \rangle]$ (lemma 2 of \S 2.3), $U(\mathbb{C}(U))$ has the axiom
\[ \frac{x_1 \in \overline{A_1}, \dots, x_{n+1} \in \overline{A_{n+1}}(x_1, \dots, x_n)}{\overline{[\langle x_1, \dots, x_i \rangle]}(x_1, \dots, x_n) = x_i \in \overline{A_i}(x_1, \dots, x_n)} \]

\underline{CF2(a).} Suppose that $B$ is a sort symbol of $U$ introduced by
\[ \frac{y_1 \in \Omega_1, \dots, y_m \in \Omega_m}{B(y_1, \dots, y_m) \text{ is a type}} \]
Suppose that for each $j, 1 \le j \le m$, the rule $\frac{x_1 \in \Delta_1, \dots, x_n \in \Delta_n}{t_j \in \Omega_j[t_1|y_1, \dots, t_{j-1}|y_{j-1}]}$ is a derived rule of $U$ with the property, i.e. such that the rule
\[ \frac{x_1 \in \overline{A_1}, \dots, x_n \in \overline{A_n}(x_1, \dots, x_{n-1})}{\overline{[\langle x_1, \dots, x_n, t_j \rangle]}(x_1, \dots, x_n) = \hat{\phi}(t_j) \in \overline{Q_j}(x_1, \dots, x_n)} \]
is a derived rule of $U(\mathbb{C}(U))$ where $Q_j = [\langle x_1 \in \Delta_1, \dots, x_n \in \Delta_n, y_j \in \Omega_j[t_1|y_1, \dots, t_{j-1}|y_{j-1}] \rangle]$. We wish to show the rule
\[ \frac{x_1 \in \Delta_1, \dots, x_n \in \Delta_n}{B(t_1, \dots, t_m) \text{ is a type}} \]
has the property, i.e. that the rule
\[ \frac{x_1 \in \overline{A_1}, \dots, x_n \in \overline{A_n}(x_1, \dots, x_{n-1})}{\overline{C}(x_1, \dots, x_n) = \hat{\phi}(B(t_1, \dots, t_m))} \]
is a derived rule of $U(\mathbb{C}(U))$, where $C = [\langle x_1 \in \Delta_1, \dots, x_n \in \Delta_n, z \in B(t_1, \dots, t_m) \rangle]$.
Let $L = [\langle y_1 \in \Omega_1, \dots, y_m \in \Omega_m, y \in B(y_1, \dots, y_m) \rangle]$.
By corollary 6(i), $\frac{x_1 \in \overline{A_1}, \dots, x_n \in \overline{A_n}(x_1, \dots, x_{n-1})}{\overline{C}(x_1, \dots, x_n) = \overline{L}(\overline{g_1}(x_1, \dots, x_n), \dots, \overline{g_m}(x_1, \dots, x_n))}$ is a derived rule of $U(\mathbb{C}(U))$ for each $j, 1 \le j \le m$, where $B_j = [\langle y_1 \in \Omega_1, \dots, y_j \in \Omega_j \rangle]$ and $g_j = [\langle x_1, \dots, x_n, t_j \rangle]$.
Hence for each $j, 1 \le j \le m$, the rule
\[ \frac{x_1 \in \overline{A_1}, \dots, x_n \in \overline{A_n}(x_1, \dots, x_{n-1})}{\overline{g_j}(x_1, \dots, x_n) = \hat{\phi}(t_j) \in \overline{B_j}(\overline{g_1}(x_1, \dots, x_n), \dots, \overline{g_{j-1}}(x_1, \dots, x_n))} \]
is a derived rule of $U(\mathbb{C}(U))$, and from the introductory rule for $\overline{L}$ in $U(\mathbb{C}(U))$, which is
\[ \frac{y_1 \in \overline{B_1}, \dots, y_m \in \overline{B_m}(y_1, \dots, y_{m-1})}{\overline{L}(y_1, \dots, y_m) \text{ is a type}} \]
we get
\[ \frac{x_1 \in \overline{A_1}, \dots, x_n \in \overline{A_n}(x_1, \dots, x_{n-1})}{\overline{L}(\overline{g_1}(x_1, \dots, x_n), \dots, \overline{g_m}(x_1, \dots, x_n)) = \overline{L}(\hat{\phi}(t_1), \dots, \hat{\phi}(t_m))} \]
as a derived rule of $U(\mathbb{C}(U))$. But again by 6(i), the rule
\[ \frac{x_1 \in \overline{A_1}, \dots, x_n \in \overline{A_n}(x_1, \dots, x_{n-1})}{\overline{L}(\overline{g_1}(x_1, \dots, x_n), \dots, \overline{g_m}(x_1, \dots, x_n)) = \overline{C}(x_1, \dots, x_n)} \]
is a derived rule of $U(\mathbb{C}(U))$. Thus the rule
\[ \frac{x_1 \in \overline{A_1}, \dots, x_n \in \overline{A_n}(x_1, \dots, x_{n-1})}{\overline{C}(x_1, \dots, x_n) = \overline{L}(\hat{\phi}(t_1), \dots, \hat{\phi}(t_m))} \]
is a derived rule of $U(\mathbb{C}(U))$, which is just whats wanted since $\hat{\phi}(B(t_1, \dots, t_m)) = \overline{L}(\hat{\phi}(t_1), \dots, \hat{\phi}(t_m))$.

\underline{CF2(b).} Very similar to CF2(a), uses corollary 6(i) and (ii).

\begin{corollary}
For every theory $U$, $\phi_U$ is an interpretation of $U$ in $U(\mathbb{C}(U))$.
\end{corollary}

\begin{proof}
It suffices to show that for any derived rule $R$ of $U$, $\hat{\phi}(R)$ is a derived rule of $U(\mathbb{C}(U))$. We check for each of the four forms separately.

1. The T-rules. If $\frac{x_1 \in \Delta_1, \dots, x_n \in \Delta_n}{\Delta_{n+1} \text{ is a type}}$ is a derived rule of $U$ then by definition $\hat{\phi}\left(\frac{x_1 \in \Delta_1, \dots, x_n \in \Delta_n}{\Delta_{n+1} \text{ is a type}}\right) = \frac{x_1 \in \hat{\phi}(\Delta_1), \dots, x_n \in \hat{\phi}(\Delta_n)}{\hat{\phi}(\Delta_{n+1}) \text{ is a type}}$.
By lemma 7, for each $i, 1 \le i \le n+1$, the rule
\[ \frac{x_1 \in \overline{A_1}, \dots, x_{i-1} \in \overline{A_{i-1}}(x_1, \dots, x_{i-1})}{\overline{A_i}(x_1, \dots, x_{i-1}) = \hat{\phi}(\Delta_i)} \]
is a derived rule of $U(\mathbb{C}(U))$. Hence for each $i, 1 \le i \le n+1$, the rule
\[ \frac{x_1 \in \hat{\phi}(\Delta_1), \dots, x_{i-1} \in \hat{\phi}(\Delta_{i-1})}{\overline{A_i}(x_1, \dots, x_{i-1}) = \hat{\phi}(\Delta_i)} \]
is a derived rule of $U(\mathbb{C}(U))$ (argue by induction). In particular
\[ \frac{x_1 \in \hat{\phi}(\Delta_1), \dots, x_n \in \hat{\phi}(\Delta_n)}{\overline{A_{n+1}}(x_1, \dots, x_n) = \hat{\phi}(\Delta_{n+1})} \]
is a derived rule of $U(\mathbb{C}(U))$. Thus because of wellformedness of derived rules (see \S 1.7) we must have $\frac{x_1 \in \hat{\phi}(\Delta_1), \dots, x_n \in \hat{\phi}(\Delta_n)}{\hat{\phi}(\Delta_{n+1}) \text{ is a type}}$ as a derived rule of $U(\mathbb{C}(U))$.

2. The $\in$-rules. Suppose that $\frac{x_1 \in \Delta_1, \dots, x_n \in \Delta_n}{t \in \Delta}$ is a derived rule of $U$. By wellformedness and by part 1. above,
\[ \frac{x_1 \in \hat{\phi}(\Delta_1), \dots, x_n \in \hat{\phi}(\Delta_n)}{\hat{\phi}(\Delta) = \overline{A}(x_1, \dots, x_n)} \quad \text{and} \quad \frac{x_1 \in \hat{\phi}(\Delta_1), \dots, x_{i-1} \in \hat{\phi}(\Delta_{i-1})}{\hat{\phi}(\Delta_i) = \overline{A_i}(x_1, \dots, x_{i-1})} \]
$1 \le i \le n$, are derived rules of $U(\mathbb{C}(U))$.
By lemma 7, $\frac{x_1 \in \overline{A_1}, \dots, x_n \in \overline{A_n}(x_1, \dots, x_{n-1})}{\overline{[\langle x_1, \dots, x_n, t \rangle]}(x_1, \dots, x_n) = \hat{\phi}(t) \in \overline{A}(x_1, \dots, x_n)}$ is a derived rule of $U(\mathbb{C}(U))$. Thus so is
\[ \frac{x_1 \in \hat{\phi}(\Delta_1), \dots, x_n \in \hat{\phi}(\Delta_n)}{\overline{[\langle x_1, \dots, x_n, t \rangle]}(x_1, \dots, x_n) = \hat{\phi}(t) \in \hat{\phi}(\Delta)} \]
a derived rule of $U(\mathbb{C}(U))$.
Hence by wellformedness the rule $\frac{x_1 \in \hat{\phi}(\Delta_1), \dots, x_n \in \hat{\phi}(\Delta_n)}{\hat{\phi}(t) \in \hat{\phi}(\Delta)}$ is a derived rule of $U(\mathbb{C}(U))$.

3. The T=rules. If $\frac{x_1 \in \Delta_1, \dots, x_n \in \Delta_n}{\Delta = \Delta'}$ is a derived rule of $U$ then $\frac{x_1 \in \hat{\phi}(\Delta_1), \dots, x_n \in \hat{\phi}(\Delta_n)}{\hat{\phi}(\Delta) = \hat{\phi}(\Delta')}$ is a derived rule of $U(\mathbb{C}(U))$ because by lemma 7 and 1. above, the rules $\frac{x_1 \in \hat{\phi}(\Delta_1), \dots, x_n \in \hat{\phi}(\Delta_n)}{\overline{A}(x_1, \dots, x_n) = \hat{\phi}(\Delta)}$ and $\frac{x_1 \in \hat{\phi}(\Delta_1), \dots, x_n \in \hat{\phi}(\Delta_n)}{\overline{A'}(x_1, \dots, x_n) = \hat{\phi}(\Delta')}$ are derived rules of $U(\mathbb{C}(U))$, where of course $A = [\langle x_1 \in \Delta_1, \dots, x_n \in \Delta_n, x \in \Delta \rangle] = [\langle x_1 \in \Delta_1, \dots, x_n \in \Delta_n, x' \in \Delta' \rangle] = A'$.

4. Similarly if $\frac{x_1 \in \Delta_1, \dots, x_n \in \Delta_n}{t = t' \in \Delta}$ is a derived rule of $U$ then
\[ \frac{x_1 \in \hat{\phi}(\Delta_1), \dots, x_n \in \hat{\phi}(\Delta_n)}{\hat{\phi}(t) = \hat{\phi}(t') \in \hat{\phi}(\Delta)} \]
is a derived rule of $U(\mathbb{C}(U))$.
\end{proof}

Recap: We are attempting to show that the functor $id_{\underline{GAT}} : \underline{GAT} \to \underline{GAT}$ is isomorphic to the functor $U \circ \mathbb{C} : \underline{GAT} \to \underline{GAT}$.
So far we have defined an interpretation $\phi_U$ of $U$ in $U(\mathbb{C}(U))$ for every theory $U$. Thus for every $U \in |\underline{GAT}|$, $[\phi_U] : U \to U(\mathbb{C}(U))$ is a morphism of \underline{GAT}. It remains to show that for every $U \in |\underline{GAT}|$, $\phi_U$ is an isomorphism and that $[\phi_-]$ is a natural transformation, $[\phi_-] : id_{\underline{GAT}} \to U \circ \mathbb{C}$. It is understood that we write $[\phi_-]$ for what otherwise might be written as $\lambda U \in |\underline{GAT}| . [\phi_U]$.

\begin{lemma}
$[\phi_-]$ is a natural transformation, $[\phi_-] : id_{\underline{GAT}} \to U \circ \mathbb{C}$.
\end{lemma}

\begin{proof}
We must show that whenever $U$ and $U'$ are theories and $I$ is an interpretation of $U$ in $U'$ then the diagram
\begin{center}
\begin{tikzcd}
U \arrow[r, "{[\phi_U]}"] \arrow[d, "{[I]}"] & U(\mathbb{C}(U)) \arrow[d, "{\mathbb{C}(U([I]))}"] \\
U' \arrow[r, "{[\phi_{U'}]}"] & U(\mathbb{C}(U'))
\end{tikzcd}
\end{center}
commutes in \underline{GAT}.

Suppose that we have such an $I$. By corollary 2 of \S 1.14 it suffices to show that for any derived $\in$-rule $R$ of $U$,
\[ \widehat{U(\mathbb{C}([I]))}(\widehat{\phi_U}(R)) \equiv \widehat{\phi_{U'}}(\widehat{I}(R)). \]

Suppose then that $x_1 \in \Delta_1, \dots, x_n \in \Delta_n : t \in \Delta$ is a derived rule of $U$. By lemma 7,
\[ \hat{\phi}_U\left(\frac{x_1 \in \Delta_1, \dots, x_n \in \Delta_n}{t \in \Delta}\right) \equiv \frac{x_1 \in \overline{A_1}, \dots, x_n \in \overline{A_n}(x_1, \dots, x_{n-1})}{\overline{f}(x_1, \dots, x_n) \in \overline{A}(x_1, \dots, x_n)} \]
where $A_i = [\langle x_1 \in \Delta_1, \dots, x_i \in \Delta_i \rangle]$, $A = [\langle x_1 \in \Delta_1, \dots, x_n \in \Delta_n, x \in \Delta \rangle]$ and $f = [\langle x_1, \dots, x_n, t \rangle]$. Thus by definition of the functor $U$,
\[ \widehat{U(\mathbb{C}([I]))}\left(\hat{\phi}_U\left(\frac{x_1 \in \Delta_1, \dots, x_n \in \Delta_n}{t \in \Delta}\right)\right) \equiv \]
\[ \frac{x_1 \in \overline{\mathbb{C}([I])(A_1)}, \dots, x_n \in \overline{\mathbb{C}([I])(A_n)}(x_1, \dots, x_{n-1})}{\overline{\mathbb{C}([I])(f)}(x_1, \dots, x_n) \in \overline{\mathbb{C}([I])(A)}(x_1, \dots, x_n)} \]

On the other hand, $\hat{I}\left(\frac{x_1 \in \Delta_1, \dots, x_n \in \Delta_n}{t \in \Delta}\right) = \frac{x_1 \in \dot{I}(\Delta_1), \dots, x_n \in \dot{I}(\Delta_n)}{\dot{I}(t) \in \dot{I}(\Delta)}$
Thus by lemma 7, $\hat{\phi}_{U'}\left(\hat{I}\left(\frac{x_1 \in \Delta_1, \dots, x_n \in \Delta_n}{t \in \Delta}\right)\right) \equiv$
\[ \frac{x_1 \in \overline{B_1}, \dots, x_n \in \overline{B_n}(x_1, \dots, x_{n-1})}{\overline{g}(x_1, \dots, x_n) \in \overline{B}(x_1, \dots, x_n)} \]
where $B_i = [\langle x_1 \in \dot{I}(\Delta_1), \dots, x_i \in \dot{I}(\Delta_i) \rangle]$, $B = [\langle x_1 \in \dot{I}(\Delta_1), \dots, x_n \in \dot{I}(\Delta_n), x \in \dot{I}(\Delta) \rangle]$ and $g = [\langle x_1, \dots, x_n, \dot{I}(t) \rangle]$. But by definition of $\mathbb{C}([I])$, $\mathbb{C}([I])(A_i) = B_i$, $\mathbb{C}([I])(A) = B$ and $\mathbb{C}([I])(f) = g$.
Hence $\widehat{U(\mathbb{C}([I]))}(\widehat{\phi_U}(R)) \equiv \widehat{\phi_{U'}}(\widehat{I}(R))$, as required.
\end{proof}

It remains to show that for any generalised algebraic theory $U$, the morphism $[\phi_U]$ of the category \underline{GAT} is an isomorphism. It suffices to define an interpretation $\psi_U$ of $U(\mathbb{C}(U))$ in $U$ and to show that $\psi_U \circ \phi_U = id_U$ and $\phi_U \circ \psi_U = id_{U(\mathbb{C}(U))}$.

Let $U$ be a theory. We define a preinterpretation $\psi_U$ of $U(\mathbb{C}(U))$ in $U$. The preinterpretation $\psi_U$ is defined on sort symbols $\overline{A}$ of $U$ by choosing an element $\langle v_1 \in \Delta_1, \dots, v_n \in \Delta_n, v_{n+1} \in \Delta \rangle$ of the equivalence class $A$ and by defining $\psi_U(\overline{A}) = \Delta$. To simplify matters we can make the choices in such a way that if $1 \triangleleft A_1 \dots \triangleleft A_n \triangleleft A$ in $\mathbb{C}(U)$ and if $\langle v_1 \in \Delta_1, \dots, v_n \in \Delta_n \rangle$ is chosen to represent $A_n$ then a context of the form $\langle v_1 \in \Delta_1, \dots, v_n \in \Delta_n, v_{n+1} \in \Delta \rangle$ is chosen to represent $A$. This is always possible by virtue of corollary 2(b) of \S 2.2.

If $f: A_n \to B$ in $\mathbb{C}(U)$ and $B$ is non-trivial then $\psi_U(\overline{f})$ is defined by choosing an element $\langle t_1, \dots, t_m, t \rangle$ of the equivalence class $f$ and by defining $\psi_U(\overline{f}) = t$. However we choose the representation $\langle t_1, \dots, t_m, t \rangle$ of $f$ in such a way that if $\langle v_1 \in \Delta_1, \dots, v_n \in \Delta_n \rangle$ represents $A_n$ and if $\langle v_1 \in \Omega_1, \dots, v_m \in \Omega_m, v_{m+1} \in \Omega \rangle$ represents $B$ then $\langle t_1, \dots, t_m, t \rangle$ is a realisation of $\langle v_1 \in \Omega_1, \dots, v_m \in \Omega_m, v_{m+1} \in \Omega \rangle$ wrt $\langle v_1 \in \Delta_1, \dots, v_n \in \Delta_n \rangle$. This is possible by lemma 4(i) of \S 1.13.
Moreover to simplify matters the choices are made in such a way that if $f: A_n \to B$ and if $f \circ p(B)$ (assuming it is non-trivial) is represented by $\langle t_1, \dots, t_m \rangle$ then $f$ is represented by $\langle t_1, \dots, t_m, t \rangle$, for some $t$. This is possible by lemma 5 of \S 1.13.

\begin{lemma}
$\psi_U$ is an interpretation of $U(\mathbb{C}(U))$ in $U$.
\end{lemma}

\begin{proof}
We must check that all the introductory rules and axioms of $U(\mathbb{C}(U))$ are mapped by $\psi_U$ to derived rules of $U$. We just check two cases the other cases are just as simple to check.

1. Suppose $\overline{A}$ is a sort symbol of $U(\mathbb{C}(U))$. Say $1 \triangleleft A_1 \dots \triangleleft A_n \triangleleft A$ in $\mathbb{C}(U)$, so that $\overline{A}$ has the introductory rule $\frac{x_1 \in \overline{A_1}, \dots, x_n \in \overline{A_n}(x_1, \dots, x_{n-1})}{\overline{A}(x_1, \dots, x_n) \text{ is a type}}$.
Suppose that $A$ has been represented by $\langle v_1 \in \Delta_1, \dots, v_n \in \Delta_n, v_{n+1} \in \Delta \rangle$, then
\[ \hat{\psi}_U\left(\frac{x_1 \in \overline{A_1}, \dots, x_n \in \overline{A_n}(x_1, \dots, x_{n-1})}{\overline{A}(x_1, \dots, x_n) \text{ is a type}}\right) = \]
\[ \frac{x_1 \in \Delta_1, x_2 \in \Delta_2[x_1|v_1], \dots, x_n \in \Delta_n[x_1|v_1, \dots, x_{n-1}|v_{n-1}]}{\Delta[x_1|v_1, \dots, x_n|v_n] \text{ is a type}} \]
and this rule is a derived rule of $U$ by the change of variable lemma of \S 1.7 because the rule $\frac{v_1 \in \Delta_1, \dots, v_n \in \Delta_n}{\Delta \text{ is a type}}$ is a derived rule of $U$.

2. If $1 \triangleleft A_1 \dots \triangleleft A_n$, $1 \triangleleft B_1 \dots \triangleleft B_m \triangleleft B$ and $f: A_n \to B_m$ in $\mathbb{C}(U)$ so that $U(\mathbb{C}(U))$ has the axiom $R$, where $R=$
\[ \frac{x_1 \in \overline{A_1}, \dots, x_n \in \overline{A_n}(x_1, \dots, x_{n-1})}{\overline{f^*B}(x_1, \dots, x_n) = \overline{B}(\overline{f \circ p(B_m, B_1)}(x_1, \dots, x_n), \dots, \overline{f \circ p(B_m, B_{m-1})}(x_1, \dots, x_n), \overline{f}(x_1, \dots, x_n))} \]

Suppose that $A_n$ has been represented by $\langle v_1 \in \Delta_1, \dots, v_n \in \Delta_n \rangle$, $B$ has been represented by $\langle v_1 \in \Omega_1, \dots, v_m \in \Omega_m, v_{m+1} \in \Omega \rangle$, $f^*B$ has been represented by $\langle v_1 \in \Delta_1, \dots, v_n \in \Delta_n, v_{n+1} \in \Delta \rangle$ and $f$ has been represented by $\langle t_1, \dots, t_m \rangle$. Then $\hat{\psi}(R) =$
\[ \frac{x_1 \in \Delta_1, \dots, x_n \in \Delta_n[x_1|v_1, \dots, x_{n-1}|v_{n-1}]}{\Delta[x_1|v_1, \dots, x_n|v_n] = \Omega[t_1[x_1|v_1, \dots, x_n|v_n]|v_1, \dots, t_m[x_1|v_1, \dots, x_n|v_n]|v_m]} \]
which is a derived rule of $U$ by the change of variable lemma since $\frac{v_1 \in \Delta_1, \dots, v_n \in \Delta_n}{\Delta = \Omega[t_1|v_1, \dots, t_m|v_m]}$ is a derived rule of $U$ because $[\langle v_1 \in \Delta_1, \dots, v_n \in \Delta_n, v_{n+1} \in \Delta \rangle] = f^*B = [\langle v_1 \in \Delta_1, \dots, v_n \in \Delta_n, v_{n+1} \in \Omega[t_1|v_1, \dots, t_m|v_m] \rangle]$.
\end{proof}

\begin{lemma}
$\psi_U \circ \phi_U \equiv id_U$.
\end{lemma}

\begin{proof}
Use corollary 2 of \S 1.14. Suppose that $\frac{x_1 \in \Delta_1, \dots, x_n \in \Delta_n}{t \in \Delta}$ is a derived rule of $U$. Let $A_i = [\langle x_1 \in \Delta_1, \dots, x_i \in \Delta_i \rangle]$ and let $A = [\langle x_1 \in \Delta_1, \dots, x_n \in \Delta_n, x \in \Delta \rangle]$. By lemma 7 of this section,
\[ \hat{\phi}_U\left(\frac{x_1 \in \Delta_1, \dots, x_n \in \Delta_n}{t \in \Delta}\right) \equiv \frac{x_1 \in \overline{A_1}, \dots, x_n \in \overline{A_n}(x_1, \dots, x_{n-1})}{\overline{[\langle x_1, \dots, x_n, t \rangle]}(x_1, \dots, x_n) \in \overline{A}(x_1, \dots, x_n)} \]
Therefore
\[ \widehat{\psi_U}(\widehat{\phi_U}\left(\frac{x_1 \in \Delta_1, \dots, x_n \in \Delta_n}{t \in \Delta}\right)) \equiv \widehat{\psi_U}\left(\frac{x_1 \in \overline{A_1}, \dots, x_n \in \overline{A_n}(x_1, \dots, x_{n-1})}{\overline{[\langle x_1, \dots, x_n, t \rangle]}(x_1, \dots, x_n) \in \overline{A}(x_1, \dots, x_n)}\right) \]
and it follows from the definition of $\psi_U$ that this rule is equivalent to
$\frac{x_1 \in \Delta_1, \dots, x_n \in \Delta_n}{t \in \Delta}$.
\end{proof}

\begin{lemma}
$\phi_U \circ \psi_U \equiv id_{U(\mathbb{C}(U))}$.
\end{lemma}

\begin{proof}
Suppose that $1 \triangleleft A_1 \dots A_n \triangleleft A$ in $\mathbb{C}(U)$, so that $\overline{A}$ is a sort symbol of $U(\mathbb{C}(U))$ introduced by the rule
\[ \frac{v_1 \in \overline{A_1}, \dots, v_n \in \overline{A_n}(v_1, \dots, v_{n-1})}{\overline{A}(v_1, \dots, v_n) \text{ is a type}} \]
If $A$ has been represented by $\langle v_1 \in \Delta_1, \dots, v_n \in \Delta_n, v_{n+1} \in \Delta \rangle$ then
\[ \hat{\phi}_U\left(\hat{\psi}_U\left(\frac{v_1 \in \overline{A_1}, \dots, v_n \in \overline{A_n}(v_1, \dots, v_{n-1})}{\overline{A}(v_1, \dots, v_n) \text{ is a type}}\right)\right) = \hat{\phi}_U\left(\frac{v_1 \in \Delta_1, \dots, v_n \in \Delta_n}{\Delta \text{ is a type}}\right) \equiv \]
\[ \frac{v_1 \in \overline{A_1}, \dots, v_n \in \overline{A_n}(v_1, \dots, v_{n-1})}{\overline{A}(v_1, \dots, v_n) \text{ is a type}}, \text{ by lemma 7.} \]

If also $1 \triangleleft B_1 \dots \triangleleft B_m \triangleleft B$ in $\mathbb{C}(U)$ and $f: A_n \to B_m$ in $\mathbb{C}(U)$. If $B_m$ has been represented by $\langle v_1 \in \Omega_1, \dots, v_m \in \Omega_m \rangle$, if $f$ has been represented by $\langle t_1, \dots, t_m \rangle$ and if $(f \circ p(B_m))^*B$ has been represented by $\langle v_1 \in \Delta_1, \dots, v_n \in \Delta_n, v_{n+1} \in \Omega \rangle$ then
\[ \hat{\phi}_U\left(\hat{\psi}_U\left(\frac{v_1 \in \overline{A_1}, \dots, v_n \in \overline{A_n}(v_1, \dots, v_{n-1})}{\overline{f}(v_1, \dots, v_n) \in \overline{(f \circ p(B_m))^*B}(v_1, \dots, v_n)}\right)\right) = \hat{\phi}_U\left(\frac{v_1 \in \Delta_1, \dots, v_n \in \Delta_n}{t_m \in \Omega}\right) \]
\[ \equiv \frac{v_1 \in \overline{A_1}, \dots, v_n \in \overline{A_n}(v_1, \dots, v_{n-1})}{\overline{[\langle v_1, \dots, v_n, t_m \rangle]}(v_1, \dots, v_n) \in \overline{(f \circ p(B_m))^*B}(v_1, \dots, v_n)} \]
by lemma 7 of this section,
\[ \equiv \frac{v_1 \in \overline{A_1}, \dots, v_n \in \overline{A_n}(v_1, \dots, v_{n-1})}{\overline{[\langle t_1, \dots, t_m \rangle]}(v_1, \dots, v_n) \in \overline{f \circ p(B_m))^*B}(v_1, \dots, v_n)} \]
by corollary 4 of this section.

This completes the proof that $\phi_U \circ \psi_U$ and $id_{U(\mathbb{C}(U))}$ agree up to equivalence on the introductory rules of $U(\mathbb{C}(U))$.
\end{proof}

\subsection{The proof that $\mathbb{C} \circ U \cong id_{Con}$}

We define a natural transformation $\eta: id_{Con} \to \mathbb{C} \circ U$. That is we define for each contextual category $\mathbb{C}$, a contextual functor $\eta_{\mathbb{C}} : \mathbb{C} \to \mathbb{C}(U(\mathbb{C}))$, such that if $F: \mathbb{C} \to \mathbb{C}'$ is a contextual functor then the diagram
\begin{center}
\begin{tikzcd}
\mathbb{C} \arrow[r, "\eta_{\mathbb{C}}"] \arrow[d, "F"] & \mathbb{C}(U(\mathbb{C})) \arrow[d, "\mathbb{C}(U(F))"] \\
\mathbb{C}' \arrow[r, "\eta_{\mathbb{C}'}"] & \mathbb{C}(U(\mathbb{C}'))
\end{tikzcd}
\end{center}
commutes in \underline{Con}. Eventually we show that $\eta$ is an isomorphism, that is that for each contextual category $\mathbb{C}$, $\eta_{\mathbb{C}}$ is an isomorphism. Then $\eta$ is the required isomorphism between $id_{\underline{Con}}$ and $\mathbb{C} \circ U$.

If $\mathbb{C}$ is a contextual category then $\eta_{\mathbb{C}}$ is defined on the trivial objects and on the trivial morphisms of $\mathbb{C}$ in the trivial manner, that is by $\eta_{\mathbb{C}}(1) = 1$ and $\eta_{\mathbb{C}}(p(A,1)) = p(\eta_{\mathbb{C}}(A), 1)$. $\eta_{\mathbb{C}}$ is defined on the non-trivial objects and morphisms of $\mathbb{C}$ as follows:

If $1 \triangleleft A_1 \dots \triangleleft A_n \triangleleft A$ in $\mathbb{C}$ then $\eta_{\mathbb{C}}(A) = [\langle v_1 \in \overline{A_1}, \dots, v_n \in \overline{A_n}(v_1, \dots, v_{n-1}), v_{n+1} \in \overline{A}(v_1, \dots, v_n) \rangle]$.

If $1 \triangleleft A_1 \dots \triangleleft A_n$ and $1 \triangleleft B_1 \dots \triangleleft B_m$ and $f: A_n \to B_m$ in $\mathbb{C}$ then $\eta_{\mathbb{C}}(f) = [\langle \overline{f \circ p(B_m, B_1)}(v_1, \dots, v_n), \dots, \overline{f \circ p(B_m, B_{m-1})}(v_1, \dots, v_n), \overline{f}(v_1, \dots, v_n) \rangle]$.

\begin{lemma}
If $\mathbb{C}$ is a contextual category then $\eta_{\mathbb{C}} : \mathbb{C} \to \mathbb{C}(U(\mathbb{C}))$ is a contextual functor.
\end{lemma}

\begin{proof}
(i). $\eta_{\mathbb{C}}$ preserves identity morphisms because if $1 \triangleleft A_1 \dots \triangleleft A_n$ in $\mathbb{C}$ then $\eta_{\mathbb{C}}(id_{A_n}) =$
$[\langle \overline{p(A_n, A_1)}(v_1, \dots, v_n), \dots, \overline{p(A_n, A_{n-1})}(v_1, \dots, v_n), \overline{id_{A_n}}(v_1, \dots, v_n) \rangle]$, by def. of $\eta$, $= [\langle v_1, \dots, v_n \rangle]$, because of axioms of $U(\mathbb{C})$ to that effect,
$= id_{[\langle v_1 \in \overline{A_1}, \dots, v_n \in \overline{A_n}(v_1, \dots, v_{n-1}) \rangle]}$.

(ii). $\eta_{\mathbb{C}}$ preserves composition because if $1 \triangleleft A_1 \dots \triangleleft A_n$, $1 \triangleleft B_1 \dots \triangleleft B_m$, $1 \triangleleft C_1 \dots \triangleleft C_q$ and $f: A_n \to B_m$, $g: B_m \to C_q$ in $\mathbb{C}$ then $\eta(f \circ g) =$
$[\langle \overline{f \circ g \circ p(C_q, C_1)}(v_1, \dots, v_n), \dots, \overline{f \circ g}(v_1, \dots, v_n) \rangle]$, by def. of $\eta_{\mathbb{C}}$.
$= [\langle \overline{g \circ p(C_q, C_1)}(\overline{f \circ p(B_m, B_1)}(v_1, \dots, v_n), \dots, \overline{f}(v_1, \dots, v_n)), \dots, \overline{g}(\overline{f \circ p(B_m, B_1)}(v_1, \dots, v_n), \dots, \overline{f}(v_1, \dots, v_n)) \rangle]$, because of axioms to that effect in $U(\mathbb{C})$.
$= [\langle \overline{f \circ p(B_m, B_1)}(v_1, \dots, v_n), \dots, \overline{f}(v_1, \dots, v_n) \rangle] \circ [\langle \overline{g \circ p(C_q, C_1)}(v_1, \dots, v_m), \dots, \overline{g}(v_1, \dots, v_m) \rangle] = \eta_{\mathbb{C}}(f) \circ \eta_{\mathbb{C}}(g)$, as required.

(iii). If $1 \triangleleft A_1 \dots \triangleleft A_n$ in $\mathbb{C}$ then $\eta_{\mathbb{C}}(p(A_n)) =$
$[\langle \overline{p(A_n, A_1)}(v_1, \dots, v_n), \dots, \overline{p(A_n, A_{n-1})}(v_1, \dots, v_n) \rangle] = [\langle v_1, \dots, v_{n-1} \rangle] = p(\eta_{\mathbb{C}}(A_n))$.

(iv). If $f: A_n \to B_m$ in $\mathbb{C}$ where $1 \triangleleft A_1 \dots \triangleleft A_n$ and $1 \triangleleft B_1 \dots \triangleleft B_m \triangleleft B$ in $\mathbb{C}$ then $\eta_{\mathbb{C}}(f^*B) = [\langle v_1 \in \overline{A_1}, \dots, v_n \in \overline{A_n}(v_1, \dots, v_{n-1}), v_{n+1} \in \overline{f^*B}(v_1, \dots, v_n) \rangle]$ by def. of $\eta_{\mathbb{C}}$.
$= [\langle v_1 \in \overline{A_1}, \dots, v_n \in \overline{A_n}(v_1, \dots, v_{n-1}), v_{n+1} \in \overline{B}(\overline{f \circ p(B_m, B_1)}(v_1, \dots, v_n), \dots, \overline{f}(v_1, \dots, v_n)) \rangle]$, because there is an axiom to that effect in $U(\mathbb{C})$.
$= [\langle \overline{f \circ p(B_m, B_1)}(v_1, \dots, v_n), \dots, \overline{f}(v_1, \dots, v_n) \rangle]^* [\langle v_1 \in \overline{B_1}, \dots, v_m \in \overline{B_m}(v_1, \dots, v_{m-1}), v_{m+1} \in \overline{B}(v_1, \dots, v_m) \rangle]$, by def. of $\mathbb{C}(U(\mathbb{C}))$.
$= \eta_{\mathbb{C}}(f)^* \eta_{\mathbb{C}}(B)$, as required.

Also in this situation, $\eta_{\mathbb{C}}(q(f,B)) = [\langle \overline{q(f,B) \circ p(B, B_1)}(v_1, \dots, v_{n+1}), \dots, \overline{q(f,B)}(v_1, \dots, v_{n+1}) \rangle] = [\langle \overline{p(f^*B) \circ f \circ p(B_m, B_1)}(v_1, \dots, v_{n+1}), \dots, \overline{p(f^*B) \circ f}(v_1, \dots, v_{n+1}), \overline{q(f,B)}(v_1, \dots, v_{n+1}) \rangle] = [\langle \overline{f \circ p(B_m, B_1)}(v_1, \dots, v_n), \dots, \overline{f}(v_1, \dots, v_n), v_{n+1} \rangle] = q(\eta_{\mathbb{C}}(f), \eta_{\mathbb{C}}(B))$.
\end{proof}

\begin{lemma}
If $F: \mathbb{C} \to \mathbb{C}'$ in \underline{Con} then the diagram
\begin{center}
\begin{tikzcd}
\mathbb{C} \arrow[r, "\eta_{\mathbb{C}}"] \arrow[d, "F"] & \mathbb{C}(U(\mathbb{C})) \arrow[d, "\mathbb{C}(U(F))"] \\
\mathbb{C}' \arrow[r, "\eta_{\mathbb{C}'}"] & \mathbb{C}(U(\mathbb{C}'))
\end{tikzcd}
\end{center}
commutes.
\end{lemma}

\begin{proof}
If $f: A_n \to B_m$, where $1 \triangleleft A_1 \dots \triangleleft A_n, 1 \triangleleft B_1 \dots \triangleleft B_m$ in $\mathbb{C}$, then $\eta_{\mathbb{C}}(f) = [\langle \overline{f \circ p(B_m, B_1)}(v_1, \dots, v_n), \dots, \overline{f}(v_1, \dots, v_n) \rangle]$. Thus $\mathbb{C}(U(F))(\eta_{\mathbb{C}}(f)) = [\langle F(\overline{f \circ p(B_m, B_1)})(v_1, \dots, v_n), \dots, F(\overline{f})(v_1, \dots, v_n) \rangle] = [\langle \overline{F(f) \circ p(F(B_m), F(B_1))}(v_1, \dots, v_n), \dots, \overline{F(f)}(v_1, \dots, v_n) \rangle] = \eta_{\mathbb{C}'}(F(f))$.
\end{proof}


So we have a natural transformation $\eta : id_{\underline{Con}} \to \mathbb{C} \circ U$. We wish to show that for each $\mathbb{C}$, $\eta_{\mathbb{C}}$ is an isomorphism in \underline{Con}. Unfortunately this turns out to be rather tricky. We have to define a contextual functor $\xi_{\mathbb{C}} : \mathbb{C}(U(\mathbb{C})) \to \mathbb{C}$ and show that $\xi_{\mathbb{C}} \circ \eta_{\mathbb{C}} = id_{\mathbb{C}}$ and $\eta_{\mathbb{C}} \circ \xi_{\mathbb{C}} = id_{\mathbb{C}(U(\mathbb{C}))}$. The procedure that we adopt in defining $\xi_{\mathbb{C}}$ is to define a function $J$ from derived T and $\in$-rules of $U(\mathbb{C})$ to objects respectively morphisms of $\mathbb{C}$. We show that $J$ induces an equivalence preserving map from contexts and realisations of $U(\mathbb{C})$ to objects respectively morphisms of $\mathbb{C}$. Thus we get a map from objects and morphisms of $\mathbb{C}(U(\mathbb{C}))$ to objects respectively morphisms of $\mathbb{C}$, we show that this map is a left and right inverse to $\eta_{\mathbb{C}}$.

Initially $J$ is defined just as a partial function from the derived T and $\in$-rules of $U(\mathbb{C})$ to the objects respectively morphisms of $\mathbb{C}$, though eventually we show that $J$ is total.

Consider the forms that the derived T and $\in$-rules of $U(\mathbb{C})$ can take. By the derivation lemma of \S 1.7 every derived T-rule of $U(\mathbb{C})$ is of the form
\[ \frac{x_1 \in \Delta_1, \dots, x_m \in \Delta_m}{\overline{A}(t_1, \dots, t_n) \text{ is a type}} \]
for some object $A$ of $\mathbb{C}$ such that $1 \triangleleft A_1 \dots \triangleleft A_n \triangleleft A$ in $\mathbb{C}$ and such that for each $i, 1 \le i \le n$,
\[ \frac{x_1 \in \Delta_1, \dots, x_m \in \Delta_m}{t_i \in \overline{A_i}(t_1, \dots, t_{i-1})} \]
is a derived rule of $U(\mathbb{C})$. By the same lemma any derived $\in$-rule of $U(\mathbb{C})$ is either of the form
\[ \frac{x_1 \in \Delta_1, \dots, x_m \in \Delta_m}{x_j \in \Lambda} \]
or else is of the form
\[ \frac{x_1 \in \Delta_1, \dots, x_m \in \Delta_m}{\overline{f}(t_1, \dots, t_n) \in \Delta} \]
for some morphism $f: A_n \to B$ where $1 \triangleleft A_1 \dots \triangleleft A_n$ in $\mathbb{C}$, such that for each $i, 1 \le i \le n$, the rule
\[ \frac{x_1 \in \Delta_1, \dots, x_m \in \Delta_m}{t_i \in \overline{A_i}(t_1, \dots, t_{i-1})} \]
is a derived rule of $U(\mathbb{C})$ and such that
\[ \frac{x_1 \in \Delta_1, \dots, x_m \in \Delta_m}{\overline{B}(t_1, \dots, t_n) = \Delta} \]
is a derived rule of $U(\mathbb{C})$.

Bearing this in mind the function $J$ from derived T and $\in$-rules of $U(\mathbb{C})$ to the objects and morphisms of $\mathbb{C}$ is defined inductively as follows:

\[ J\left( \frac{x_1 \in \Delta_1, \dots, x_m \in \Delta_m}{\overline{A}(t_1, \dots, t_n) \text{ is a type}} \right) = J(R_{t_n})^* \dots J(R_{t_1})^* p(J(R_{\Delta_m}), 1)^* A \]

\[ J\left( \frac{x_1 \in \Delta_1, \dots, x_m \in \Delta_m}{\overline{f}(t_1, \dots, t_n) \in \Delta} \right) = J(R_{t_n})^* \dots J(R_{t_1})^* p(J(R_{\Delta_m}), 1)^* \text{'f'} \]

\[ J\left( \frac{x_1 \in \Delta_1, \dots, x_m \in \Delta_m}{x_j \in \Lambda} \right) = \text{'p}(J(R_{\Delta_m}), J(R_{\Delta_j}))\text{'}. \]

Where for each $i, 1 \le i \le n$, $R_{t_i}$ is the derived rule
\[ \frac{x_1 \in \Delta_1, \dots, x_m \in \Delta_m}{t_i \in \overline{A_i}(t_1, \dots, t_{i-1})} \]
and where for each $j, 1 \le j \le m$, $R_{\Delta_j}$ is the derived rule
\[ \frac{x_1 \in \Delta_1, \dots, x_{j-1} \in \Delta_{j-1}}{\Delta_j \text{ is a type}}. \]

If $R$ is a derived T-rule of $U(\mathbb{C})$ and if $J(R)$ is defined then $J(R)$ is an object of $\mathbb{C}$. If $R$ is a derived $\in$-rule of $U(\mathbb{C})$ and if $J(R)$ is defined then $J(R)$ is a morphism of $\mathbb{C}$.

$J\left( \frac{x_1 \in \Delta_1, \dots, x_m \in \Delta_m}{\overline{A}(t_1, \dots, t_n) \text{ is a type}} \right)$ can fail to be defined either because one or more of the $J(R_{t_i})$'s is not defined or else because the composite is not defined in $\mathbb{C}$. For example, $J\left( \frac{x_1 \in \Delta_1, \dots, x_m \in \Delta_m}{\overline{A}(t_1, \dots, t_n) \text{ is a type}} \right)$ will certainly not be defined unless $J(R_{t_1})$ is a morphism of $\mathbb{C}$ whose codomain occurs lower in the tree structure of $\mathbb{C}$ than the object $p(J(R_{\Delta_m}), 1)^* A$, for otherwise $J(R_{t_1})^* p(J(R_{\Delta_m}), 1)^* A$ will not be defined and thus $J(R_{t_n})^* \dots J(R_{t_1})^* p(J(R_{\Delta_m}), 1)^* A$ will not be defined.

We wish to show that $J(R)$ is defined for all derived T and $\in$-rules $R$ of $U(\mathbb{C})$. This is going to require a proof by induction on derivations in $U(\mathbb{C})$. It turns out that the inductive hypothesis that we must use is rather complicated. This is because as we proceed to prove that $J$ is defined on the derived rules $R$ of $U(\mathbb{C})$ we must keep an eye on the behaviour of $J$ on substitution instances of $R$, otherwise the induction does not go through. If we call the inductive hypothesis $H$ then $H$ is a possible property of derived rules of $U(\mathbb{C})$. That is for any derived rule $R$ of $U$ either $H(R)$ is or is not the case. Of course we aim to show that $H(R)$ is always the case.

$H$ is defined inductively. The definition of $H(R)$ depends on which kind of rule $R$ is, thus there are four cases to consider:

\textbf{Case 1. T-rules.} If $R_{\Delta}$ is a derived T-rule of $U(\mathbb{C})$ of the form $\frac{x_1 \in \Delta_1, \dots, x_n \in \Delta_n}{\Delta \text{ is a type}}$ then $H(R_{\Delta})$ is equivalent to 1(a) and 1(b) and 1(c), which are as follows:

1(a). If $n \ge 1$, that is if the premise of $R_{\Delta}$ is not the empty premise, then $H(R_{\Delta_n})$, where $R_{\Delta_n}$ is the rule $\frac{x_1 \in \Delta_1, \dots, x_{n-1} \in \Delta_{n-1}}{\Delta_n \text{ is a type}}$.

1(b). $J(R_{\Delta})$ is defined and if $n \ge 1$ then $J(R_{\Delta_n}) \triangleleft J(R_{\Delta})$ in $\mathbb{C}$. If $n=0$ then $1 \triangleleft J(R_{\Delta})$ in $\mathbb{C}$.

1(c). $J\left( \frac{y_1 \in \Omega_1, \dots, y_m \in \Omega_m}{\Delta[t_1|x_1, \dots, t_n|x_n] \text{ is a type}} \right)$ is defined and is equal to $J(R_{t_n})^* \dots J(R_{t_1})^* p(J(R_{\Omega_m}), 1)^* J(R_{\Delta})$, whenever $\langle y_1 \in \Omega_1, \dots, y_m \in \Omega_m \rangle$ is a context of $U(\mathbb{C})$ and whenever $\langle t_1, \dots, t_n \rangle$ is a realisation of $\langle x_1 \in \Delta_1, \dots, x_n \in \Delta_n \rangle$ wrt $\langle y_1 \in \Omega_1, \dots, y_m \in \Omega_m \rangle$ with the property that for each $i, 1 \le i \le n$,
\[ J\left( \frac{y_1 \in \Omega_1, \dots, y_m \in \Omega_m}{\Delta_i[t_1|x_1, \dots, t_{i-1}|x_{i-1}] \text{ is a type}} \right) \]
is defined and $J(R_{t_i})$ is defined and
\[ J(R_{t_i}) \in Arr_{\mathbb{C}}\left( J\left( \frac{y_1 \in \Omega_1, \dots, y_m \in \Omega_m}{\Delta_i[t_1|x_1, \dots, t_{i-1}|x_{i-1}] \text{ is a type}} \right) \right). \]
Where $R_{t_i}$ is the rule $\frac{y_1 \in \Omega_1, \dots, y_m \in \Omega_m}{t_i \in \Delta_i[t_1|x_1, \dots, t_{i-1}|x_{i-1}]}$ and $R_{\Omega_m}$ is the rule $\frac{y_1 \in \Omega_1, \dots, y_{m-1} \in \Omega_{m-1}}{\Omega_m \text{ is a type}}$. (In future we use the convention that $R_t$ denotes a rule of the form $\frac{-\in-, \dots, -\in-}{t \in -}$. It will then usually be quite clear which rule $R_t$ denotes and we will not have to mention it. Similarly, unless we say otherwise $R_{\Delta_i}$ denotes the rule $\frac{x_1 \in \Delta_1, \dots, x_{i-1} \in \Delta_{i-1}}{\Delta_i \text{ is a type}}$.)

\textbf{Case 2. $\in$-rules.} If $R_t$ is a derived $\in$-rule of $U(\mathbb{C})$ of the form $\frac{x_1 \in \Delta_1, \dots, x_n \in \Delta_n}{t \in \Delta}$ then $H(R_t)$ is equivalent to 2(a) and 2(b) and 2(c) which are as follows:

2(a). $H(R_{\Delta})$. (it is hoped that it is understood that $R_{\Delta}$ is the rule $\frac{x_1 \in \Delta_1, \dots, x_n \in \Delta_n}{\Delta \text{ is a type}}$).

2(b). $J(R_t)$ is defined and $J(R_t) \in Arr_{\mathbb{C}}(J(R_{\Delta}))$.

2(c). $J\left( \frac{y_1 \in \Omega_1, \dots, y_m \in \Omega_m}{t[t_1|x_1, \dots, t_n|x_n] \in \Delta[t_1|x_1, \dots, t_n|x_n]} \right)$ is defined and is equal to $J(R_{t_n})^* \dots J(R_{t_1})^* p(J(R_{\Omega_m}), 1)^* J(R_t)$, whenever $\langle y_1 \in \Omega_1, \dots, y_m \in \Omega_m \rangle$ is a context of $U(\mathbb{C})$ and whenever $\langle t_1, \dots, t_n \rangle$ is a realisation of $\langle x_1 \in \Delta_1, \dots, x_n \in \Delta_n \rangle$ wrt $\langle y_1 \in \Omega_1, \dots, y_m \in \Omega_m \rangle$ with the property that for all $i, 1 \le i \le n$,
\[ J\left( \frac{y_1 \in \Omega_1, \dots, y_m \in \Omega_m}{\Delta_i[t_1|x_1, \dots, t_{i-1}|x_{i-1}] \text{ is a type}} \right) \]
is defined and $J(R_{t_i})$ is defined and
\[ J(R_{t_i}) \in Arr_{\mathbb{C}}\left( J\left( \frac{y_1 \in \Omega_1, \dots, y_m \in \Omega_m}{\Delta_i[t_1|x_1, \dots, t_{i-1}|x_{i-1}] \text{ is a type}} \right) \right). \]

\textbf{Case 3. T=rules.} If $\frac{x_1 \in \Delta_1, \dots, x_n \in \Delta_n}{\Delta = \Delta'}$ is a derived rule of $U(\mathbb{C})$ then $H\left( \frac{x_1 \in \Delta_1, \dots, x_n \in \Delta_n}{\Delta = \Delta'} \right)$ is equivalent to $H(R_{\Delta})$ and $H(R_{\Delta'})$ and $J(R_{\Delta}) = J(R_{\Delta'})$.

\textbf{Case 4. $\in$=rules.} If $\frac{x_1 \in \Delta_1, \dots, x_n \in \Delta_n}{t = t' \in \Delta}$ is a derived rule of $U(\mathbb{C})$ then $H\left( \frac{x_1 \in \Delta_1, \dots, x_n \in \Delta_n}{t = t' \in \Delta} \right)$ is equivalent to $H(R_t)$ and $H(R_{t'})$ and $J(R_t) = J(R_{t'})$.

This completes the definition of the inductive hypothesis $H$. We still need two lemmas before we can proceed with the induction.

\begin{lemma}
If $R$ is a derived rule of $U(\mathbb{C})$ of the form $\frac{x_1 \in \Delta_1, \dots, x_n \in \Delta_n}{\text{Conclusion}}$ such that $H(R)$, If $\langle t_1, \dots, t_n \rangle$ is a realisation of $\langle x_1 \in \Delta_1, \dots, x_n \in \Delta_n \rangle$ wrt $\langle y_1 \in \Omega_1, \dots, y_m \in \Omega_m \rangle$ such that for each $i, 1 \le i \le n$, $H(R_{t_i})$ then also
\[ H\left( \frac{y_1 \in \Omega_1, \dots, y_m \in \Omega_m}{\text{Conclusion}[t_1|x_1, \dots, t_n|x_n]} \right). \]
\end{lemma}

\begin{proof}
By induction on $n$. We suppose that the result holds for all rules $R'$ of the form $\frac{x'_1 \in \Delta'_1, \dots, x'_{n'} \in \Delta'_{n'}}{\text{Conclusion}'}$ for $n' < n$.
We show that the result holds of $R$, we must treat each of the four kinds of rules separately.

Case 1. $R$ is a T-rule, say $R = R_{\Delta} = \frac{x_1 \in \Delta_1, \dots, x_n \in \Delta_n}{\Delta \text{ is a type}}$. We must show that $H$ holds of the rule $\frac{y_1 \in \Omega_1, \dots, y_m \in \Omega_m}{\Delta[t_1|x_1, \dots, t_n|x_n] \text{ is a type}}$, that is we must show that 1(a), 1(b) and 1(c) hold of this rule.

1(a). $H(R_{\Omega_m})$ is the case because $H\left( \frac{y_1 \in \Omega_1, \dots, y_m \in \Omega_m}{t_1 \in \Delta_1} \right)$ is the case.

1(b). Since for each $i, 1 \le i \le n$, $H(R_{t_i})$, so for each $i, 1 \le i \le n$, $J(R_{t_i})$ is defined and
\[ J(R_{t_i}) \in Arr_{\mathbb{C}}\left( J\left( \frac{y_1 \in \Omega_1, \dots, y_m \in \Omega_m}{\Delta_i[t_1|x_1, \dots, t_{i-1}|x_{i-1}] \text{ is a type}} \right) \right). \]
Since $H(R_{\Delta})$, so $R_{\Delta}$ has property 1(c). Thus
\[ J\left( \frac{y_1 \in \Omega_1, \dots, y_m \in \Omega_m}{\Delta[t_1|x_1, \dots, t_n|x_n] \text{ is a type}} \right) \]
is defined and is equal to $J(R_{t_n})^* \dots J(R_{t_1})^* p(J(R_{\Omega_m}), 1)^* J(R_{\Delta})$. We wish to show that $J(R_{\Omega_m}) \triangleleft J\left( \frac{y_1 \in \Omega_1, \dots, y_m \in \Omega_m}{\Delta[t_1|x_1, \dots, t_n|x_n] \text{ is a type}} \right)$, thus we must show that
\[ J(R_{\Omega_m}) \triangleleft J(R_{t_n})^* \dots J(R_{t_1})^* p(J(R_{\Omega_m}), 1)^* J(R_{\Delta}). \]
$H(R_{\Delta})$ implies that for each $i, 1 \le i \le n$, $H(R_{\Delta_i})$. Thus, just as above, for each $i, 1 \le i \le n$,
\[ J\left( \frac{y_1 \in \Omega_1, \dots, y_m \in \Omega_m}{\Delta_i[t_1|x_1, \dots, t_{i-1}|x_{i-1}] \text{ is a type}} \right) \]
is defined and is equal to $J(R_{t_{i-1}})^* \dots J(R_{t_1})^* p(J(R_{\Omega_m}), 1)^* J(R_{\Delta_i})$.
The situation in $\mathbb{C}$ then is that $1 \triangleleft J(R_{\Delta_1}) \dots \triangleleft J(R_{\Delta_n}) \triangleleft J(R_{\Delta})$ and for each $i, 1 \le i \le n$, $J(R_{t_i}) \in Arr_{\mathbb{C}}(J(R_{t_{i-1}})^* \dots J(R_{t_1})^* p(J(R_{\Omega_m}), 1)^* J(R_{\Delta_i}))$. Hence $J(R_{\Omega_m}) \triangleleft J(R_{t_n})^* \dots J(R_{t_1})^* p(J(R_{\Omega_m}), 1)^* J(R_{\Delta})$ in $\mathbb{C}$. The situation is as in lemma 3 of \S 2.3.

1(c). Suppose that $\langle s_1, \dots, s_m \rangle$ is a realisation of $\langle y_1 \in \Omega_1, \dots, y_m \in \Omega_m \rangle$ wrt $\langle z_1 \in \Lambda_1, \dots, z_p \in \Lambda_p \rangle$ and suppose that for each $j, 1 \le j \le m$, $J(R_{s_j})$ is defined and
\[ J(R_{s_j}) \in Arr_{\mathbb{C}}\left( J\left( \frac{z_1 \in \Lambda_1, \dots, z_p \in \Lambda_p}{\Omega_j[s_1|y_1, \dots, s_{j-1}|y_{j-1}] \text{ is a type}} \right) \right). \]
We must show that
\[ J\left( \frac{z_1 \in \Lambda_1, \dots, z_p \in \Lambda_p}{\Delta[t_1|x_1, \dots, t_n|x_n][s_1|y_1, \dots, s_m|y_m] \text{ is a type}} \right) \]
is defined and is equal to $J(R_{s_m})^* \dots J(R_{s_1})^* p(J(R_{\Lambda_p}), 1)^* J\left( \frac{y_1 \in \Omega_1, \dots, y_m \in \Omega_m}{\Delta[t_1|x_1, \dots, t_n|x_n] \text{ is a type}} \right)$.

By the induction we can assume the corresponding result for the rule $\frac{x_1 \in \Delta_1, \dots, x_{i-1} \in \Delta_{i-1}}{\Delta_i \text{ is a type}}$, whenever $i \le n$. That is we can assume that for each $i, 1 \le i \le n$,
\[ J\left( \frac{z_1 \in \Lambda_1, \dots, z_p \in \Lambda_p}{\Delta_i[t_1|x_1, \dots, t_{i-1}|x_{i-1}][s_1|y_1, \dots, s_m|y_m] \text{ is a type}} \right) \]
is defined and is equal to $J(R_{s_m})^* \dots J(R_{s_1})^* p(J(R_{\Lambda_p}), 1)^* J\left( \frac{y_1 \in \Omega_1, \dots, y_m \in \Omega_m}{\Delta_i[t_1|x_1, \dots, t_{i-1}|x_{i-1}] \text{ is a type}} \right)$. Call this the inductive assumption.

Since for each $i, 1 \le i \le n$, $H(R_{t_i})$, so for each $i, 1 \le i \le n$,
\[ J\left( \frac{z_1 \in \Lambda_1, \dots, z_p \in \Lambda_p}{t_i[s_1|y_1, \dots, s_m|y_m] \in \Delta_i[t_1|x_1, \dots, t_{i-1}|x_{i-1}][s_1|y_1, \dots, s_m|y_m]} \right) \]
is defined and is equal to $J(R_{s_m})^* \dots J(R_{s_1})^* p(J(R_{\Lambda_p}), 1)^* J(R_{t_i})$. Since $J(R_{t_i}) \in Arr_{\mathbb{C}}\left( J\left( \frac{y_1 \in \Omega_1, \dots, y_m \in \Omega_m}{\Delta_i[t_1|x_1, \dots, t_{i-1}|x_{i-1}] \text{ is a type}} \right) \right)$, so
\[ J\left( \frac{z_1 \in \Lambda_1, \dots, z_p \in \Lambda_p}{t_i[s_1|y_1, \dots, s_m|y_m] \in \Delta_i[t_1|x_1, \dots, t_{i-1}|x_{i-1}][s_1|y_1, \dots, s_m|y_m]} \right) \]
\[ \in Arr_{\mathbb{C}}\left( J(R_{s_m})^* \dots J(R_{s_1})^* p(J(R_{\Lambda_p}), 1)^* J\left( \frac{y_1 \in \Omega_1, \dots, y_m \in \Omega_m}{\Delta_i[t_1|x_1, \dots, t_{i-1}|x_{i-1}] \text{ is a type}} \right) \right). \]

Now we use the inductive assumption to get, for each $i, 1 \le i \le n$,
\[ J\left( \frac{z_1 \in \Lambda_1, \dots, z_p \in \Lambda_p}{t_i[s_1|y_1, \dots, s_m|y_m] \in \Delta_i[t_1|x_1, \dots, t_{i-1}|x_{i-1}][s_1|y_1, \dots, s_m|y_m]} \right) \]
\[ \in Arr_{\mathbb{C}}\left( J\left( \frac{z_1 \in \Lambda_1, \dots, z_p \in \Lambda_p}{\Delta_i[t_1|x_1, \dots, t_{i-1}|x_{i-1}][s_1|y_1, \dots, s_m|y_m]} \right) \right). \]
Let $\sigma_i = t_i[s_1|y_1, \dots, s_m|y_m]$ for each $i, 1 \le i \le n$. We have shown that $\langle \sigma_1, \dots, \sigma_n \rangle$ is a realisation of $\langle x_1 \in \Delta_1, \dots, x_n \in \Delta_n \rangle$ wrt $\langle z_1 \in \Lambda_1, \dots, z_p \in \Lambda_p \rangle$ such that for each $i, 1 \le i \le n$, $J(R_{\sigma_i})$ is defined and $J(R_{\sigma_i}) \in Arr_{\mathbb{C}}\left( J\left( \frac{z_1 \in \Lambda_1, \dots, z_p \in \Lambda_p}{\Delta_i[\sigma_1|x_1, \dots, \sigma_{i-1}|x_{i-1}] \text{ is a type}} \right) \right)$.
Since $H(R_{\Delta})$ we can conclude that
\[ J\left( \frac{z_1 \in \Lambda_1, \dots, z_p \in \Lambda_p}{\Delta[\sigma_1|x_1, \dots, \sigma_n|x_n] \text{ is a type}} \right) \]
is defined and is equal to $J(R_{\sigma_n})^* \dots J(R_{\sigma_1})^* p(J(R_{\Lambda_p}), 1)^* J(R_{\Delta})$.

Since $H(R_{t_i})$, so $J(R_{\sigma_i}) = J(R_{s_m})^* \dots J(R_{s_1})^* p(J(R_{\Lambda_p}), 1)^* J(R_{t_i})$.
Hence $J\left( \frac{z_1 \in \Lambda_1, \dots, z_p \in \Lambda_p}{\Delta[\sigma_1|x_1, \dots, \sigma_n|x_n] \text{ is a type}} \right) = (J(R_{s_m})^* \dots J(R_{s_1})^* p(J(R_{\Lambda_p}), 1)^* J(R_{t_n}))^* \dots (J(R_{s_m})^* \dots J(R_{s_1})^* p(J(R_{\Lambda_p}), 1)^* J(R_{t_1}))^* p(J(R_{\Lambda_p}), 1)^* J(R_{\Delta}) = J(R_{s_m})^* \dots J(R_{s_1})^* p(J(R_{\Lambda_p}), 1)^* J(R_{t_n})^* \dots J(R_{t_1})^* p(J(R_{\Omega_m}), 1)^* J(R_{\Delta})$ by lemma 4(iii) of \S 2.3.
$= J(R_{s_m})^* \dots J(R_{s_1})^* p(J(R_{\Lambda_p}), 1)^* J\left( \frac{y_1 \in \Omega_1, \dots, y_m \in \Omega_m}{\Delta[t_1|x_1, \dots, t_n|x_n] \text{ is a type}} \right)$, as required, since $H(R_{\Delta})$.

Case 2. $R$ is an $\in$-rule, say $R = R_t = \frac{x_1 \in \Delta_1, \dots, x_n \in \Delta_n}{t \in \Delta}$.
We assume $H(R_t)$ and we must show that
\[ H\left( \frac{y_1 \in \Omega_1, \dots, y_m \in \Omega_m}{t[t_1|x_1, \dots, t_n|x_n] \in \Delta[t_1|x_1, \dots, t_n|x_n]} \right). \]
That is we must show that 2(a), 2(b) and 2(c) hold of the rule.

2(a). It follows from Case 1 that $H\left( \frac{y_1 \in \Omega_1, \dots, y_m \in \Omega_m}{\Delta[t_1|x_1, \dots, t_n|x_n] \text{ is a type}} \right)$.

2(b). Because $H(R_t)$ it follows that
\[ J\left( \frac{y_1 \in \Omega_1, \dots, y_m \in \Omega_m}{t[t_1|x_1, \dots, t_n|x_n] \in \Delta[t_1|x_1, \dots, t_n|x_n]} \right) \]
is defined and is equal to $J(R_{t_n})^* \dots J(R_{t_1})^* p(J(R_{\Omega_m}), 1)^* J(R_t)$. Because $H(R_t)$ it follows that $J(R_t) \in Arr_{\mathbb{C}}(J(R_{\Delta}))$. Hence
\[ J\left( \frac{y_1 \in \Omega_1, \dots, y_m \in \Omega_m}{t[t_1|x_1, \dots, t_n|x_n] \in \Delta[t_1|x_1, \dots, t_n|x_n]} \right) \in Arr_{\mathbb{C}}(J(R_{t_n})^* \dots J(R_{t_1})^* p(J(R_{\Omega_m}), 1)^* J(R_{\Delta})). \]
But since $H(R_{\Delta})$, $J(R_{t_n})^* \dots J(R_{t_1})^* p(J(R_{\Omega_m}), 1)^* J(R_{\Delta}) = J\left( \frac{y_1 \in \Omega_1, \dots, y_m \in \Omega_m}{\Delta[t_1|x_1, \dots, t_n|x_n] \text{ is a type}} \right)$, thus
\[ J\left( \frac{y_1 \in \Omega_1, \dots, y_m \in \Omega_m}{t[t_1|x_1, \dots, t_n|x_n] \in \Delta[t_1|x_1, \dots, t_n|x_n]} \right) \in Arr_{\mathbb{C}}\left( J\left( \frac{y_1 \in \Omega_1, \dots, y_m \in \Omega_m}{\Delta[t_1|x_1, \dots, t_n|x_n] \text{ is a type}} \right) \right) \]
as required.

2(c). Suppose that $\langle s_1, \dots, s_m \rangle$ is a realisation of $\langle y_1 \in \Omega_1, \dots, y_m \in \Omega_m \rangle$ wrt $\langle z_1 \in \Lambda_1, \dots, z_p \in \Lambda_p \rangle$ such that for each $j, 1 \le j \le m$, $J(R_{s_j})$ is defined and $J(R_{s_j}) \in Arr_{\mathbb{C}}\left( J\left( \frac{z_1 \in \Lambda_1, \dots, z_p \in \Lambda_p}{\Omega_j[s_1|y_1, \dots, s_{j-1}|y_{j-1}] \text{ is a type}} \right) \right)$.
If we let $\sigma_i = t_i[s_1|y_1, \dots, s_m|y_m]$ for each $i, 1 \le i \le n$, then $t[t_1|x_1, \dots, t_n|x_n][s_1|y_1, \dots, s_m|y_m] = t[\sigma_1|x_1, \dots, \sigma_n|x_n]$ and $\Delta[t_1|x_1, \dots, t_n|x_n][s_1|y_1, \dots, s_m|y_m] = \Delta[\sigma_1|x_1, \dots, \sigma_n|x_n]$.
Thus we just have to show that
\[ J\left( \frac{z_1 \in \Lambda_1, \dots, z_p \in \Lambda_p}{t[\sigma_1|x_1, \dots, \sigma_n|x_n] \in \Delta[\sigma_1|x_1, \dots, \sigma_n|x_n]} \right) \]
is defined and is equal to $J(R_{s_m})^* \dots J(R_{s_1})^* p(J(R_{\Lambda_p}), 1)^* J\left( \frac{y_1 \in \Omega_1, \dots, y_m \in \Omega_m}{t[t_1|x_1, \dots, t_n|x_n] \in \Delta[t_1|x_1, \dots, t_n|x_n]} \right)$.

As in case 1, $\langle \sigma_1, \dots, \sigma_n \rangle$ is a realisation of $\langle x_1 \in \Delta_1, \dots, x_n \in \Delta_n \rangle$ wrt $\langle z_1 \in \Lambda_1, \dots, z_p \in \Lambda_p \rangle$ such that for each $i, 1 \le i \le n$, $J(R_{\sigma_i})$ is defined and $J(R_{\sigma_i}) \in Arr_{\mathbb{C}}\left( J\left( \frac{z_1 \in \Lambda_1, \dots, z_p \in \Lambda_p}{\Delta_i[\sigma_1|x_1, \dots, \sigma_{i-1}|x_{i-1}] \text{ is a type}} \right) \right)$.
Thus since $H(R_t)$,
\[ J\left( \frac{z_1 \in \Lambda_1, \dots, z_p \in \Lambda_p}{t[\sigma_1|x_1, \dots, \sigma_n|x_n] \in \Delta[\sigma_1|x_1, \dots, \sigma_n|x_n]} \right) \]
is defined and is equal to $J(R_{\sigma_n})^* \dots J(R_{\sigma_1})^* p(J(R_{\Lambda_p}), 1)^* J(R_t)$. But as $H(R_{t_i})$, it follows that $J(R_{\sigma_i}) = J(R_{s_m})^* \dots J(R_{s_1})^* p(J(R_{\Lambda_p}), 1)^* J(R_{t_i})$.
Hence
\[ J\left( \frac{z_1 \in \Lambda_1, \dots, z_p \in \Lambda_p}{t[\sigma_1|x_1, \dots, \sigma_n|x_n] \in \Delta[\sigma_1|x_1, \dots, \sigma_n|x_n]} \right) = \]
$(J(R_{s_m})^* \dots J(R_{s_1})^* p(J(R_{\Lambda_p}), 1)^* J(R_{t_n}))^* \dots (J(R_{s_m})^* \dots J(R_{s_1})^* p(J(R_{\Lambda_p}), 1)^* J(R_{t_1}))^* p(J(R_{\Lambda_p}), 1)^* J(R_t) = J(R_{s_m})^* \dots J(R_{s_1})^* p(J(R_{\Lambda_p}), 1)^* J(R_{t_n})^* \dots J(R_{t_1})^* p(J(R_{\Omega_m}), 1)^* J(R_t)$, by lemma 4(iii) of \S 2.3.
But since $H(R_t)$, $J(R_{t_n})^* \dots J(R_{t_1})^* p(J(R_{\Omega_m}), 1)^* J(R_t) = J\left( \frac{y_1 \in \Omega_1, \dots, y_m \in \Omega_m}{t[t_1|x_1, \dots, t_n|x_n] \in \Delta[t_1|x_1, \dots, t_n|x_n]} \right)$. Thus
\[ J\left( \frac{z_1 \in \Lambda_1, \dots, z_p \in \Lambda_p}{t[\sigma_1|x_1, \dots, \sigma_n|x_n] \in \Delta[\sigma_1|x_1, \dots, \sigma_n|x_n]} \right) = \]
$J(R_{s_m})^* \dots J(R_{s_1})^* p(J(R_{\Lambda_p}), 1)^* J\left( \frac{y_1 \in \Omega_1, \dots, y_m \in \Omega_m}{t[t_1|x_1, \dots, t_n|x_n] \in \Delta[t_1|x_1, \dots, t_n|x_n]} \right)$, as required.

Case 3. $R$ is a T=rule, say $R = \frac{x_1 \in \Delta_1, \dots, x_n \in \Delta_n}{\Delta = \Delta'}$.
We assume $H(R)$ and we must show that $H\left( \frac{y_1 \in \Omega_1, \dots, y_m \in \Omega_m}{\Delta[t_1|x_1, \dots, t_n|x_n] = \Delta'[t_1|x_1, \dots, t_n|x_n]} \right)$.
$H(R)$ implies that $H(R_{\Delta})$ and by case 1 that implies
$H\left( \frac{y_1 \in \Omega_1, \dots, y_m \in \Omega_m}{\Delta[t_1|x_1, \dots, t_n|x_n] \text{ is a type}} \right)$.
Similarly we must have $H\left( \frac{y_1 \in \Omega_1, \dots, y_m \in \Omega_m}{\Delta'[t_1|x_1, \dots, t_n|x_n] \text{ is a type}} \right)$.
$H(R_{\Delta})$ also implies that $J\left( \frac{y_1 \in \Omega_1, \dots, y_m \in \Omega_m}{\Delta[t_1|x_1, \dots, t_n|x_n] \text{ is a type}} \right) = J(R_{t_n})^* \dots J(R_{t_1})^* p(J(R_{\Omega_m}), 1)^* J(R_{\Delta})$.
$H(R_{\Delta'})$ implies that $J\left( \frac{y_1 \in \Omega_1, \dots, y_m \in \Omega_m}{\Delta'[t_1|x_1, \dots, t_n|x_n] \text{ is a type}} \right) = J(R_{t_n})^* \dots J(R_{t_1})^* p(J(R_{\Omega_m}), 1)^* J(R_{\Delta'})$.
But $H(R)$ implies that $J(R_{\Delta}) = J(R_{\Delta'})$, hence
\[ J\left( \frac{y_1 \in \Omega_1, \dots, y_m \in \Omega_m}{\Delta[t_1|x_1, \dots, t_n|x_n] \text{ is a type}} \right) = J\left( \frac{y_1 \in \Omega_1, \dots, y_m \in \Omega_m}{\Delta'[t_1|x_1, \dots, t_n|x_n] \text{ is a type}} \right). \]
This completes the proof that $H\left( \frac{y_1 \in \Omega_1, \dots, y_m \in \Omega_m}{\Delta[t_1|x_1, \dots, t_n|x_n] = \Delta'[t_1|x_1, \dots, t_n|x_n]} \right)$.

Case 4. $R$ is an $\in$=rule. Very similar to case 3.
\end{proof}

\begin{lemma}
(i). For every $n \ge 1$, if $1 \triangleleft A_1 \dots \triangleleft A_n$ in $\mathbb{C}$ then (a)
\[ J\left( \frac{x_1 \in \overline{A_1}, \dots, x_{n-1} \in \overline{A_{n-1}}(x_1, \dots, x_{n-2})}{\overline{A_n}(x_1, \dots, x_{n-1}) \text{ is a type}} \right) \]
is defined and is equal to $A_n$. (b) for any $i, 1 \le i \le n$,
\[ J\left( \frac{x_1 \in \overline{A_1}, \dots, x_n \in \overline{A_n}(x_1, \dots, x_{n-1})}{x_i \in \overline{A_i}(x_1, \dots, x_{i-1})} \right) \]
is defined and is equal to $\text{'p}(A_n, A_i)\text{'}$.

(ii). For every $n \ge 1$, if $1 \triangleleft A_1 \dots \triangleleft A_n$ in $\mathbb{C}$ and if $f: A_n \to B$ is a non-trivial morphism of $\mathbb{C}$ then
\[ J\left( \frac{x_1 \in \overline{A_1}, \dots, x_n \in \overline{A_n}(x_1, \dots, x_{n-1})}{\overline{f}(x_1, \dots, x_n) \in \overline{(f \circ p(B))^*B}(x_1, \dots, x_n)} \right) \]
is defined and is equal to $\text{'f'}$.
\end{lemma}

\begin{proof}
(i). The proof is by induction on $n$. If $n=1$ then (a) $J(\overline{A_1} \text{ is a type})$ is by definition $A_1$. (b) $J\left( \frac{x \in \overline{A_1}}{x \in \overline{A_1}} \right)$ is by definition $\text{'p}(A_1, A_1)\text{'}$. If $n>1$ then (a)
\[ J\left( \frac{x_1 \in \overline{A_1}, \dots, x_{n-1} \in \overline{A_{n-1}}(x_1, \dots, x_{n-2})}{\overline{A_n}(x_1, \dots, x_{n-1}) \text{ is a type}} \right) \]
is defined to be $J(R_{x_{n-1}})^* \dots J(R_{x_1})^* p(J(R_{A_{n-1}}), 1)^* A_n$, where $R_{x_i}$ is the rule $\frac{x_1 \in \overline{A_1}, \dots, x_{n-1} \in \overline{A_{n-1}}(x_1, \dots, x_{n-2})}{x_i \in \overline{A_i}(x_1, \dots, x_{i-1})}$ and $R_{A_{n-1}}$ is the rule
$\frac{x_1 \in \overline{A_1}, \dots, x_{n-2} \in \overline{A_{n-2}}(x_1, \dots, x_{n-2})}{\overline{A_{n-1}}(x_1, \dots, x_{n-2}) \text{ is a type}}$. By the inductive hypothesis
$J(R_{x_i}) = \text{'p}(A_{n-1}, A_i)\text{'}$ and $J(R_{A_{n-1}}) = A_{n-1}$. Thus
$J\left( \frac{x_1 \in \overline{A_1}, \dots, x_{n-1} \in \overline{A_{n-1}}(x_1, \dots, x_{n-2})}{\overline{A_n}(x_1, \dots, x_{n-1}) \text{ is a type}} \right)$ is defined and is equal to $A_n$ by lemma 3(ii) of \S 2.3.
(b) In view of (a),
$J\left( \frac{x_1 \in \overline{A_1}, \dots, x_n \in \overline{A_n}(x_1, \dots, x_{n-1})}{x_i \in \overline{A_i}(x_1, \dots, x_{i-1})} \right)$ is just defined to be $\text{'p}(A_n, A_i)\text{'}$.

(ii). By definition of $J$,
\[ J\left( \frac{x_1 \in \overline{A_1}, \dots, x_n \in \overline{A_n}(x_1, \dots, x_{n-1})}{\overline{f}(x_1, \dots, x_n) \in \overline{(f \circ p(B))^*B}(x_1, \dots, x_n)} \right) = \]
\[ J\left( \frac{x_1 \in \overline{A_1}, \dots, x_n \in \overline{A_n}(x_1, \dots, x_{n-1})}{x_n \in \overline{A_n}(x_1, \dots, x_{n-1})} \right)^* \dots J\left( \frac{x_1 \in \overline{A_1}, \dots, x_n \in \overline{A_n}(x_1, \dots, x_{n-1})}{x_1 \in \overline{A_1}} \right)^* \]
\[ p\left( J\left( \frac{x_1 \in \overline{A_1}, \dots, x_{n-1} \in \overline{A_{n-1}}(x_1, \dots, x_{n-2})}{\overline{A_n}(x_1, \dots, x_{n-1}) \text{ is a type}} \right), 1 \right)^* \text{'f'}. \]
So by part (i) of this lemma, this equals $\text{'p}(A_n, A_n)\text{'}^* \dots \text{'p}(A_n, A_1)\text{'}^* p(A_n, 1)^* \text{'f'}$, which equals $\text{'f'}$ by lemma 3(ii) of \S 2.3.
\end{proof}

\begin{corollary}
(i) For every sort symbol $\overline{A}$ of $U(\mathbb{C})$, if $1 \triangleleft A_1 \dots \triangleleft A_n \triangleleft A$ in $\mathbb{C}$ then
\[ H\left( \frac{x_1 \in \overline{A_1}, \dots, x_n \in \overline{A_n}(x_1, \dots, x_{n-1})}{\overline{A}(x_1, \dots, x_n) \text{ is a type}} \right) \]

(ii) For every sort symbol $\overline{f}$ of $U(\mathbb{C})$, if $1 \triangleleft A_1 \dots \triangleleft A_n$ in $\mathbb{C}$ and $f: A_n \to B$ in $\mathbb{C}$ then
\[ H\left( \frac{x_1 \in \overline{A_1}, \dots, x_n \in \overline{A_n}(x_1, \dots, x_{n-1})}{\overline{f}(x_1, \dots, x_n) \in \overline{(f \circ p(B))^*B}(x_1, \dots, x_n)} \right) \]
\end{corollary}

\begin{proof}
In view of lemma 12 it only remains to show that $J$ behaves correctly on all substitution instances of the given rules. But the definition of $J$ ensures exactly this.
\end{proof}

\begin{lemma}
For every derived rule $R$ of $U(\mathbb{C})$, $H(R)$.
\end{lemma}

\begin{proof}
By induction on the derivations in $U(\mathbb{C})$. We must check that every principle of derivation preserves the property $H$.
The principles LI1-7 preserve property $H$. This can be seen at a glance. We go on to the other principles.

\underline{T1.} Let $R_t$ be a derived rule of $U(\mathbb{C})$ of the form $\frac{x_1 \in \Delta_1, \dots, x_n \in \Delta_n}{t \in \Delta}$ and let $R$ be a derived rule of $U(\mathbb{C})$ of the form $\frac{x_1 \in \Delta_1, \dots, x_n \in \Delta_n}{\Delta = \Delta'}$. Let $R'_t$ be the rule $\frac{x_1 \in \Delta_1, \dots, x_n \in \Delta_n}{t \in \Delta'}$. We must show that $H(R_t)$ and $H(R)$ implies that $H(R'_t)$.

$J\left( \frac{P}{t \in \Delta} \right)$ is always defined independently of $\Delta$ so since $J(R_t)$ is defined and belongs to $Arr_{\mathbb{C}}(J(R_{\Delta}))$ it follows that $J(R'_t)$ is defined and belongs to $Arr_{\mathbb{C}}(J(R_{\Delta}))$. Since $H(R)$ it follows that $J(R_{\Delta}) = J(R_{\Delta'})$, hence $J(R'_t)$ is defined and belongs to $Arr_{\mathbb{C}}(J(R_{\Delta'}))$. That is 2(b) holds of $R'_t$. 2(a) holds of $R'_t$ because $H(R)$ implies $H(R_{\Delta'})$. 2(c) holds of $R'_t$ because 2(c) holds of $R_t$ and because $J\left( \frac{P}{t \in \Delta} \right)$ is defined independently of $\Delta$.

\underline{CF1.} Suppose that $H$ holds of the derived rule $\frac{x_1 \in \Delta_1, \dots, x_n \in \Delta_n}{\Delta_{n+1} \text{ is a type}}$ of $U(\mathbb{C})$. We wish to show that for all $i, 1 \le i \le n+1$,
\[ H\left( \frac{x_1 \in \Delta_1, \dots, x_{n+1} \in \Delta_{n+1}}{x_i \in \Delta_i} \right). \]
The proof is by induction on $i$. Fix an $i, 1 \le i \le n+1$, assume that for all $j, 1 \le j < i$,
\[ H\left( \frac{x_1 \in \Delta_1, \dots, x_{n+1} \in \Delta_{n+1}}{x_j \in \Delta_j} \right) \]
we can now show that
\[ H\left( \frac{x_1 \in \Delta_1, \dots, x_{n+1} \in \Delta_{n+1}}{x_i \in \Delta_i} \right) \]
is the case as follows:
$H(R_{\Delta_{n+1}})$ implies that for all $k, 1 \le k \le n+1, H(R_{\Delta_k})$. In particular $H(R_{\Delta_i})$. Also, $\langle x_1, \dots, x_{i-1} \rangle$ is a realisation of $\langle x_1 \in \Delta_1, \dots, x_{i-1} \in \Delta_{i-1} \rangle$ wrt $\langle x_1 \in \Delta_1, \dots, x_{n+1} \in \Delta_{n+1} \rangle$ such that for each $j, 1 \le j \le i-1$, $H\left( \frac{x_1 \in \Delta_1, \dots, x_{n+1} \in \Delta_{n+1}}{x_j \in \Delta_j} \right)$.
Thus by lemma 11 it follows that
\[ H\left( \frac{x_1 \in \Delta_1, \dots, x_{n+1} \in \Delta_{n+1}}{\Delta_i \text{ is a type}} \right) \]
Which is to say that 2(a) holds of $\frac{x_1 \in \Delta_1, \dots, x_{n+1} \in \Delta_{n+1}}{x_i \in \Delta_i}$.

By definition of $J$,
\[ J\left( \frac{x_1 \in \Delta_1, \dots, x_{n+1} \in \Delta_{n+1}}{x_j \in \Delta_j} \right) = \text{'p}(J(R_{\Delta_{n+1}}), J(R_{\Delta_j}))\text{'}. \]
Thus since $H(R_{\Delta_i})$,
\[ J\left( \frac{x_1 \in \Delta_1, \dots, x_{n+1} \in \Delta_{n+1}}{\Delta_i \text{ is a type}} \right) = \text{'p}(J(R_{\Delta_{n+1}}), J(R_{\Delta_{i-1}}))\text{'}^* \dots \text{'p}(J(R_{\Delta_{n+1}}), J(R_{\Delta_1}))\text{'}^* p(J(R_{\Delta_{n+1}}), 1)^* J(R_{\Delta_i}). \]
Thus, by lemma 3(ii) of \S 2.3,
\[ J\left( \frac{x_1 \in \Delta_1, \dots, x_{n+1} \in \Delta_{n+1}}{\Delta_i \text{ is a type}} \right) = p(J(R_{\Delta_{n+1}}), J(R_{\Delta_{i-1}}))^* J(R_{\Delta_i}). \]
By definition
\[ J\left( \frac{x_1 \in \Delta_1, \dots, x_{n+1} \in \Delta_{n+1}}{x_i \in \Delta_i} \right) = \text{'p}(J(R_{\Delta_{n+1}}), J(R_{\Delta_i}))\text{'}. \]
Hence
\[ J\left( \frac{x_1 \in \Delta_1, \dots, x_{n+1} \in \Delta_{n+1}}{x_i \in \Delta_i} \right) \in Arr_{\mathbb{C}}\left( J\left( \frac{x_1 \in \Delta_1, \dots, x_{n+1} \in \Delta_{n+1}}{\Delta_i \text{ is a type}} \right) \right). \]
Which is to say 2(b) holds of $\frac{x_1 \in \Delta_1, \dots, x_{n+1} \in \Delta_{n+1}}{x_i \in \Delta_i}$.

Now suppose that $\langle t_1, \dots, t_{n+1} \rangle$ is a realisation of $\langle x_1 \in \Delta_1, \dots, x_{n+1} \in \Delta_{n+1} \rangle$ wrt $\langle y_1 \in \Omega_1, \dots, y_m \in \Omega_m \rangle$ and that for each $k, 1 \le k \le n+1$, $J(R_{t_k})$ is defined, belongs to
\[ Arr_{\mathbb{C}}\left( J\left( \frac{y_1 \in \Omega_1, \dots, y_m \in \Omega_m}{\Delta_k[t_1|x_1, \dots, t_{k-1}|x_{k-1}] \text{ is a type}} \right) \right). \]
We must show that $J(R_{t_i}) = J(R_{t_{n+1}})^* \dots J(R_{t_1})^* p(J(R_{\Omega_m}), 1)^* J\left( \frac{x_1 \in \Delta_1, \dots, x_{n+1} \in \Delta_{n+1}}{x_i \in \Delta_i} \right)$.
But that is immediate from lemma 4(i) of \S 2.3, since
\[ J\left( \frac{x_1 \in \Delta_1, \dots, x_{n+1} \in \Delta_{n+1}}{x_i \in \Delta_i} \right) = \text{'p}(J(R_{\Delta_{n+1}}), J(R_{\Delta_i}))\text{'}. \]
Thus 2(c) holds of $\frac{x_1 \in \Delta_1, \dots, x_{n+1} \in \Delta_{n+1}}{x_i \in \Delta_i}$.

\underline{CF2(a).} Suppose that $\overline{A}$ is a sort symbol of $U(\mathbb{C})$ introduced by the rule $\frac{x_1 \in \overline{A_1}, \dots, x_n \in \overline{A_n}(x_1, \dots, x_{n-1})}{\overline{A}(x_1, \dots, x_n) \text{ is a type}}$. Suppose that for each $i, 1 \le i \le n$, $\frac{y_1 \in \Omega_1, \dots, y_m \in \Omega_m}{t_i \in \overline{A_i}(t_1, \dots, t_{i-1})}$ is a derived rule of $U(\mathbb{C})$ of which $H$ holds.
We must show that $H\left( \frac{y_1 \in \Omega_1, \dots, y_m \in \Omega_m}{\overline{A}(t_1, \dots, t_n) \text{ is a type}} \right)$.
This is an immediate consequence of lemma 11 and corollary 13.

\underline{CF2(b).} Similar to CF2(a).

\underline{SI1.} Suppose that $R$ is a derived rule of $U(\mathbb{C})$ of the form $\frac{x_1 \in \Delta_1, \dots, x_n \in \Delta_n}{\Delta = \Delta'}$ and that for each $i, 1 \le i \le n$, $R_i$ is a derived rule of $U(\mathbb{C})$ of the form $\frac{y_1 \in \Omega_1, \dots, y_m \in \Omega_m}{t_i = t'_i \in \Delta_i[t_1|x_1, \dots, t_{i-1}|x_{i-1}]}$.
Suppose also that $H(R)$ and that for each $i, 1 \le i \le n, H(R_i)$. We must show that
\[ H\left( \frac{y_1 \in \Omega_1, \dots, y_m \in \Omega_m}{\Delta[t_1|x_1, \dots, t_n|x_n] = \Delta'[t'_1|x_1, \dots, t'_n|x_n]} \right). \]

From $H(R)$ we deduce $H(R_{\Delta})$ and from each $H(R_i)$ we deduce $H(R_{t_i})$. By lemma 11 it follows that
\[ H\left( \frac{y_1 \in \Omega_1, \dots, y_m \in \Omega_m}{\Delta[t_1|x_1, \dots, t_n|x_n] \text{ is a type}} \right). \]
Similarly it follows that
\[ H\left( \frac{y_1 \in \Omega_1, \dots, y_m \in \Omega_m}{\Delta'[t'_1|x_1, \dots, t'_n|x_n] \text{ is a type}} \right). \]
Finally, $J\left( \frac{y_1 \in \Omega_1, \dots, y_m \in \Omega_m}{\Delta[t_1|x_1, \dots, t_n|x_n] \text{ is a type}} \right) = J(R_{t_n})^* \dots J(R_{t_1})^* p(J(R_{\Omega_m}), 1)^* J(R_{\Delta})$.
$J(R_{\Delta}) = J(R_{\Delta'})$. $J(R_{t_i}) = J(R_{t'_i})$.
Thus $J(R_{t_n})^* \dots J(R_{t_1})^* p(J(R_{\Omega_m}), 1)^* J(R_{\Delta}) = J(R_{t'_n})^* \dots J(R_{t'_1})^* p(J(R_{\Omega_m}), 1)^* J(R_{\Delta'}) = J\left( \frac{y_1 \in \Omega_1, \dots, y_m \in \Omega_m}{\Delta'[t'_1|x_1, \dots, t'_n|x_n] \text{ is a type}} \right)$.
So $H\left( \frac{y_1 \in \Omega_1, \dots, y_m \in \Omega_m}{\Delta[t_1|x_1, \dots, t_n|x_n] = \Delta'[t'_1|x_1, \dots, t'_n|x_n]} \right)$, as required.

\underline{SI2.} Similar to SI1.

\underline{A1 and A2.} We wish to show that these principles preserve property $H$. We must show that whenever $\frac{P}{\Delta = \Delta'}$ is an axiom of $U(\mathbb{C})$ such that $H\left( \frac{P}{\Delta \text{ is a type}} \right)$ and $H\left( \frac{P}{\Delta' \text{ is a type}} \right)$ then $H\left( \frac{P}{\Delta = \Delta'} \right)$.
And we must show that whenever $\frac{P}{t = t' \in \Delta}$ is an axiom of $U(\mathbb{C})$ such that $H\left( \frac{P}{t \in \Delta} \right)$ and $H\left( \frac{P}{t' \in \Delta} \right)$ then $H\left( \frac{P}{t = t' \in \Delta} \right)$.
In the first case this amounts to showing that $J\left( \frac{P}{\Delta \text{ is a type}} \right) = J\left( \frac{P}{\Delta' \text{ is a type}} \right)$, in the second case it amounts to showing that $J\left( \frac{P}{t \in \Delta} \right) = J\left( \frac{P}{t' \in \Delta} \right)$.
So we go through all of the axioms of $U(\mathbb{C})$ checking that one or the other holds as appropriate.

(i) For $n \ge 0, m \ge 1, l \ge 1$, if $1 \triangleleft A_1 \dots \triangleleft A_n, 1 \triangleleft B_1 \dots \triangleleft B_m$ and $1 \triangleleft C_1 \dots \triangleleft C_l$ and $f: A_n \to B_m, g: B_m \to C_l$ in $\mathbb{C}$ then $U(\mathbb{C})$ has the axiom
\[ \frac{x_1 \in \overline{A_1}, \dots, x_n \in \overline{A_n}(x_1, \dots, x_{n-1})}{\overline{f \circ g}(x_1, \dots, x_n) = \overline{g}(\overline{f \circ p(B_m, B_1)}(x_1, \dots, x_n), \dots, \overline{f}(x_1, \dots, x_n)) \in \overline{(f \circ g \circ p(C_l))^* C_l}(x_1, \dots, x_n)} \]

By lemma 12,
\[ J\left( \frac{x_1 \in \overline{A_1}, \dots, x_n \in \overline{A_n}(x_1, \dots, x_{n-1})}{\overline{f \circ g}(x_1, \dots, x_n) \in \overline{(f \circ g \circ p(C_l))^* C_l}(x_1, \dots, x_n)} \right) = \text{'f} \circ \text{g'}. \]
On the other hand, by definition
\[ J\left( \frac{x_1 \in \overline{A_1}, \dots, x_n \in \overline{A_n}(x_1, \dots, x_{n-1})}{\overline{g}(\overline{f \circ p(B_m, B_1)}(x_1, \dots, x_n), \dots, \overline{f}(x_1, \dots, x_n)) \in \overline{(f \circ g \circ p(C_l))^* C_l}(x_1, \dots, x_n)} \right) \]
\[ J\left( \frac{x_1 \in \overline{A_1}, \dots, x_n \in \overline{A_n}(x_1, \dots, x_{n-1})}{\overline{f}(x_1, \dots, x_n) \in \overline{B_m}} \right)^* \dots J\left( \frac{x_1 \in \overline{A_1}, \dots, x_n \in \overline{A_n}(x_1, \dots, x_{n-1})}{\overline{f \circ p(B_m, B_1)}(x_1, \dots, x_n) \in \overline{B_1}} \right)^* \]
\[ p\left( J\left( \frac{x_1 \in \overline{A_1}, \dots, x_n \in \overline{A_n}(x_1, \dots, x_{n-1})}{\overline{(f \circ p(B_m, B_1))^* B_1}(x_1, \dots, x_n) \text{ is a type}} \right), 1 \right)^* \text{'g'} \]
which by lemma 12 equals $\text{'f'}^* \dots \text{'f} \circ p(B_m, B_1)\text{'}^* p(A_n, 1)^* \text{'g'}$. By lemma 3(ii) of \S 2.3 this equals $f''g'$, which by lemma 4(i) of \S 2.3 equals $\text{'f} \circ \text{g'}$. Which is just whats required.

(ii). For $n \ge 1$, if $1 \triangleleft A_1 \dots \triangleleft A_n$ in $\mathbb{C}$ and $1 \le i \le n$ then $U(\mathbb{C})$ has the axiom
\[ \frac{x_1 \in \overline{A_1}, \dots, x_n \in \overline{A_n}(x_1, \dots, x_{n-1})}{\overline{p(A_n, A_i)}(x_1, \dots, x_n) = x_i \in \overline{A_i}(x_1, \dots, x_{i-1})} \]
\[ J\left( \frac{x_1 \in \overline{A_1}, \dots, x_n \in \overline{A_n}(x_1, \dots, x_{n-1})}{\overline{p(A_n, A_i)}(x_1, \dots, x_n) \in \overline{A_i}(x_1, \dots, x_{i-1})} \right) \]
is defined and is equal to $\text{'p}(A_n, A_i)\text{'}$ by lemma 12(ii).
\[ J\left( \frac{x_1 \in \overline{A_1}, \dots, x_n \in \overline{A_n}(x_1, \dots, x_{n-1})}{x_i \in \overline{A_i}(x_1, \dots, x_{i-1})} \right) \]
is defined and equal to $\text{'p}(A_n, A_i)\text{'}$ by virtue of lemma 12(i). Thus
\[ J\left( \frac{x_1 \in \overline{A_1}, \dots, x_n \in \overline{A_n}(x_1, \dots, x_{n-1})}{\overline{p(A_n, A_i)}(x_1, \dots, x_n) \in \overline{A_i}(x_1, \dots, x_{i-1})} \right) = J\left( \frac{x_1 \in \overline{A_1}, \dots, x_n \in \overline{A_n}(x_1, \dots, x_{n-1})}{x_i \in \overline{A_i}(x_1, \dots, x_{i-1})} \right) \]
as required.

(iii). For $n \ge 0, m \ge 1$, if $1 \triangleleft A_1 \dots \triangleleft A_n$, $1 \triangleleft B_1 \dots \triangleleft B_m \triangleleft B$ and $f: A_n \to B_m$ in $\mathbb{C}$ then $U(\mathbb{C})$ has the axioms
\[ \frac{x_1 \in \overline{A_1}, \dots, x_n \in \overline{A_n}(x_1, \dots, x_{n-1})}{\overline{f^*B}(x_1, \dots, x_n) = \overline{B}(\overline{f \circ p(B_m, B_1)}(x_1, \dots, x_n), \dots, \overline{f}(x_1, \dots, x_n))} \]
and
\[ \frac{x_1 \in \overline{A_1}, \dots, x_n \in \overline{A_n}(x_1, \dots, x_{n-1}), y \in \overline{f^*B}(x_1, \dots, x_n)}{\overline{q(f,B)}(x_1, \dots, x_n, y) = y \in \overline{f^*B}(x_1, \dots, x_n)} \]

By virtue of lemma 12,
\[ J\left( \frac{x_1 \in \overline{A_1}, \dots, x_n \in \overline{A_n}(x_1, \dots, x_{n-1})}{\overline{B}(\overline{f \circ p(B_m, B_1)}(x_1, \dots, x_n), \dots, \overline{f}(x_1, \dots, x_n)) \text{ is a type}} \right) \]
is defined to be $\text{'f'}^* \dots \text{'f} \circ p(B_m, B_1)\text{'}^* p(A_n, 1)^* B$, which by lemma 3(i) of \S 2.3 is just $f^*B$, which is
\[ J\left( \frac{x_1 \in \overline{A_1}, \dots, x_n \in \overline{A_n}(x_1, \dots, x_{n-1})}{\overline{f^*B}(x_1, \dots, x_n) \text{ is a type}} \right) \]
as required.

By definition of $J$,
\[ J\left( \frac{x_1 \in \overline{A_1}, \dots, x_n \in \overline{A_n}(x_1, \dots, x_{n-1}), y \in \overline{f^*B}(x_1, \dots, x_n)}{y \in \overline{f^*B}(x_1, \dots, x_n)} \right) = \text{'id}_{f^*B}\text{'}. \]
By lemma 12,
\[ J\left( \frac{x_1 \in \overline{A_1}, \dots, x_n \in \overline{A_n}(x_1, \dots, x_{n-1}), y \in \overline{f^*B}(x_1, \dots, x_n)}{\overline{q(f,B)}(x_1, \dots, x_n, y) \in \overline{f^*B}(x_1, \dots, x_n)} \right) = \text{'q}(f,B)\text{'}. \]
But by lemma 5 of \S 2.3, $\text{'id}_{f^*B}\text{'} = \text{'q}(f,B)\text{'}$.
\end{proof}

We can collect together the information about $J$ that we really want in the following:

\begin{corollary}
(i) If $\langle x_1 \in \Delta_1, \dots, x_n \in \Delta_n \rangle$ is a context of $U(\mathbb{C})$ then $1 \triangleleft J(R_{\Delta_1}) \dots \triangleleft J(R_{\Delta_n})$ in $\mathbb{C}$, where $R_{\Delta_i} = \frac{x_1 \in \Delta_1, \dots, x_{i-1} \in \Delta_{i-1}}{\Delta_i \text{ is a type}}$.

(ii) If $\langle t_1, \dots, t_m \rangle$ is a realisation of $\langle y_1 \in \Omega_1, \dots, y_m \in \Omega_m \rangle$ wrt $\langle x_1 \in \Delta_1, \dots, x_n \in \Delta_n \rangle$ in $U(\mathbb{C})$ then for each $j, 1 \le j \le m$, $J(R_{t_j}) \in Arr_{\mathbb{C}}(J(R_{t_{j-1}})^* \dots J(R_{t_1})^* p(J(R_{\Delta_n}), 1)^* J(R_{\Omega_j}))$. Where $R_{t_j}$ is the rule $\frac{x_1 \in \Delta_1, \dots, x_n \in \Delta_n}{t_j \in \Omega_j[t_1|y_1, \dots, t_{j-1}|y_{j-1}]}$.

(iii) If $\langle x_1 \in \Delta_1, \dots, x_n \in \Delta_n \rangle \equiv \langle x'_1 \in \Delta'_1, \dots, x'_n \in \Delta'_n \rangle$ then $J(R_{\Delta_n}) = J(R_{\Delta'_n})$.

(iv) If $\langle t_1, \dots, t_m \rangle \equiv \langle t'_1, \dots, t'_m \rangle$ then for each $j, 1 \le j \le m$, $J(R_{t_j}) = J(R_{t'_j})$.
\end{corollary}

By corollary 14(ii) and by lemma 3(iii) of \S 2.3, whenever $\langle t_1, \dots, t_m \rangle$ is a realisation of $\langle y_1 \in \Omega_1, \dots, y_m \in \Omega_m \rangle$ wrt $\langle x_1 \in \Delta_1, \dots, x_n \in \Delta_n \rangle$ in $U(\mathbb{C})$ then there exists a unique $m$-tuple $\langle \gamma_1, \dots, \gamma_m \rangle$ of morphisms of $\mathbb{C}$ such that for each $j, 1 \le j \le m$, $\gamma_j : J(R_{\Delta_n}) \to J(R_{\Omega_j})$ and $\text{'}\gamma_j\text{'} = J(R_{t_j})$, and such that for each $j, 1 \le j < m$, $\gamma_{j+1} \circ p(J(R_{\Omega_{j+1}}), J(R_{\Omega_j})) = \gamma_j$. This last condition implies that the $m$-tuple $\langle \gamma_1, \dots, \gamma_m \rangle$ is determined by $\gamma_m : J(R_{\Delta_n}) \to J(R_{\Omega_m})$. So the statement can be reworded: for each such realisation there exists a unique $\gamma : J(R_{\Delta_n}) \to J(R_{\Omega_m})$ such that for each $j, 1 \le j \le m$, $\text{'}\gamma \circ p(J(R_{\Omega_m}), J(R_{\Omega_j}))\text{'} = J(R_{t_j})$. Thus we can define a function $\xi$ from objects and morphisms of $\mathbb{C}(U(\mathbb{C}))$ to objects and morphisms of $\mathbb{C}$ by
\begin{center}
\begin{tikzcd}
{[\langle x_1 \in \Delta_1{,} \dots{,} x_n \in \Delta_n \rangle]} \arrow[d, "{[\langle t_1{,} \dots{,} t_m \rangle]}"] \arrow[r, "\xi"] & J(R_{\Delta_n}) \arrow[d, "\gamma"] \\
{[\langle y_1 \in \Omega_1{,} \dots{,} y_m \in \Omega_m \rangle]} \arrow[r] & J(R_{\Omega_m})
\end{tikzcd}
\end{center}
where $\gamma$ is the unique map such that for all $j, 1 \le j \le m$, $\text{'}\gamma \circ p(J(R_{\Omega_m}), J(R_{\Omega_j}))\text{'} = J(R_{t_j})$.

Then $\xi$ is well defined by corollary 14(iii) and (iv).

We show that $\xi$ is an inverse to $\eta_{\mathbb{C}} : \mathbb{C} \to \mathbb{C}(U(\mathbb{C}))$. We need one last lemma.

\begin{lemma}
If $\frac{x_1 \in \Delta_1, \dots, x_{n-1} \in \Delta_{n-1}}{\Delta_n \text{ is a type}}$ is a derived rule of $U(\mathbb{C})$ then for all $i, 1 \le i \le n$, $\frac{x_1 \in \Delta_1, \dots, x_{i-1} \in \Delta_{i-1}}{\Delta_i = \overline{J(R_{\Delta_i})}(x_1, \dots, x_{i-1})}$ is a derived rule of $U(\mathbb{C})$.

If $\frac{x_1 \in \Delta_1, \dots, x_n \in \Delta_n}{t \in \Delta}$ is a derived rule of $U(\mathbb{C})$ then
\[ \frac{x_1 \in \Delta_1, \dots, x_n \in \Delta_n}{t = \overline{J(R_t)}(x_1, \dots, x_n)} \]
is a derived rule of $U(\mathbb{C})$.
\end{lemma}

\begin{proof}
By induction on derivations in $U(\mathbb{C})$. We show that each principle of derivation preserves the property. We have only to check those principles by which T and $\in$-rules are derived. These are T1, CF1, CF2(a) and CF2(b). In fact the checking of T1 is trivial. CF2(b) is very similar to CF2(a), so that leaves us to check CF1 and CF2(a).

\underline{CF1.} Suppose that for all $j, 1 \le j \le n+1$, $\frac{x_1 \in \Delta_1, \dots, x_{j-1} \in \Delta_{j-1}}{\Delta_j = \overline{J(R_{\Delta_j})}(x_1, \dots, x_{j-1})}$ is a derived rule of $U(\mathbb{C})$ and suppose that we have some $i, 1 \le i \le n+1$. We must show that
\[ \frac{x_1 \in \Delta_1, \dots, x_{n+1} \in \Delta_{n+1}}{x_i = \overline{\text{'p}(J(R_{\Delta_{n+1}}), J(R_{\Delta_i}))\text{'}}(x_1, \dots, x_{n+1}) \in \Delta_i} \]
is a derived rule of $U(\mathbb{C})$. But this easily follows from corollary 3 of this section and the axiom
\[ \frac{x_1 \in \overline{J(R_{\Delta_1})}, \dots, x_{n+1} \in \overline{J(R_{\Delta_{n+1}})}(x_1, \dots, x_n)}{\overline{p(J(R_{\Delta_{n+1}}), J(R_{\Delta_i}))}(x_1, \dots, x_{n+1}) = x_i \in \overline{J(R_{\Delta_i})}(x_1, \dots, x_{i-1})} \]

\underline{CF2(a).} Suppose that $\overline{B}$ is a sort symbol of $U(\mathbb{C})$ introduced by the rule $\frac{y_1 \in \overline{B_1}, \dots, y_m \in \overline{B_m}(y_1, \dots, y_{m-1})}{\overline{B}(y_1, \dots, y_m) \text{ is a type}}$ and suppose that for each $j, 1 \le j \le m$, $\frac{x_1 \in \Delta_1, \dots, x_n \in \Delta_n}{t_j = \overline{J(R_{t_j})}(x_1, \dots, x_n) \in \overline{B_j}(t_1, \dots, t_{j-1})}$ is a derived rule of $U(\mathbb{C})$.
We must show that the rule $\frac{x_1 \in \Delta_1, \dots, x_n \in \Delta_n}{\overline{B}(t_1, \dots, t_m) = \overline{J(R)}(x_1, \dots, x_n)}$ is a derived rule of $U(\mathbb{C})$, where $R$ is the rule $\frac{x_1 \in \Delta_1, \dots, x_n \in \Delta_n}{\overline{B}(t_1, \dots, t_m) \text{ is a type}}$.

Let $\gamma: J(R_{\Delta_n}) \to B_m$ be the unique map such that for all $j, 1 \le j \le m$, $\text{'}\gamma \circ p(B_m, B_j)\text{'} = J(R_{t_j})$. By lemma 5(i) of this section
\[ \frac{x_1 \in \overline{J(R_{\Delta_1})}, \dots, x_n \in \overline{J(R_{\Delta_n})}(x_1, \dots, x_{n-1})}{\overline{\gamma^*B}(x_1, \dots, x_n) = \overline{B}(\overline{J(R_{t_1})}(x_1, \dots, x_n), \dots, \overline{J(R_{t_m})}(x_1, \dots, x_n))} \]
is a derived rule of $U(\mathbb{C})$. Thus so is
\[ \frac{x_1 \in \Delta_1, \dots, x_n \in \Delta_n}{\overline{\gamma^*B}(x_1, \dots, x_n) = \overline{B}(t_1, \dots, t_m)}. \]
But by lemma 3(ii) of \S 2.3, $\gamma^*B = J(R_{t_m})^* \dots J(R_{t_1})^* p(J(R_{\Delta_n}), 1)^* B$, which by definition of $J$ is just $J\left( \frac{x_1 \in \Delta_1, \dots, x_n \in \Delta_n}{\overline{B}(t_1, \dots, t_m) \text{ is a type}} \right)$. Thus $\frac{x_1 \in \Delta_1, \dots, x_n \in \Delta_n}{\overline{B}(t_1, \dots, t_m) = \overline{J(R)}(x_1, \dots, x_n)}$ is a derived rule of $U(\mathbb{C})$, as required.
\end{proof}

\begin{corollary}
(i) $\xi_{\mathbb{C}} \circ \eta_{\mathbb{C}} = id_{\mathbb{C}}$. (ii) $\eta_{\mathbb{C}} \circ \xi_{\mathbb{C}} = id_{\mathbb{C}(U(\mathbb{C}))}$.
\end{corollary}

\begin{proof}
(i) Follows directly from lemma 12. (ii) Follows from lemma 15 and corollary 3.
\end{proof}

It now follows from quite general considerations that $\xi_{\mathbb{C}} : \mathbb{C}(U(\mathbb{C})) \to \mathbb{C}$ is a contextual functor. For example $\xi$ preserves composition because $\xi(f \circ g) = \xi(\eta_{\mathbb{C}}(\xi(f)) \circ \eta_{\mathbb{C}}(\xi(g))) = \xi(\eta_{\mathbb{C}}(\xi(f) \circ \xi(g))) = \xi(f) \circ \xi(g)$.

So that completes the proof that $\eta_{\mathbb{C}} : \mathbb{C} \to \mathbb{C}(U(\mathbb{C}))$ is an isomorphism for every contextual category $\mathbb{C}$. So completing the proof that the category \underline{Con} is equivalent to the category \underline{GAT}.

\section{Functorial Semantics, Universal Algebra}

\subsection{Functorial Semantics}

An algebraic semantics is an equivalence between a category of theories and a category of structures. We referred to several such in \S 2.1. In all cases so far considered there is a further equivalence. In all cases the usual definition of model of a theory can be replaced by a new definition which uses only the notion of structure. Lawvere has used the term functorial semantics in describing this kind of semantics. Functorial semantics depends on an equivalence between the category of models of a theory $U$ and the category of structure preserving morphisms from the structure $\mathbb{C}(U)$ corresponding to $U$ to a special canonical structure (the world structure?). For example the canonical structure is taken to be the category with products \underline{Set} in the case of algebraic theories (Lawvere [11]). Or in the case of classical propositional theories the canonical structure is taken to be the Boolean Algebra $\{0,1\}$.

The present situation is as well behaved as any if the canonical structure is taken to be the contextual category \underline{Fam}.

If $U$ is a generalised algebraic theory then the category of models of $U$ is equivalent to the category which has contextual functors $\mathbb{C}(U)$ to \underline{Fam} as objects and natural transformations as morphisms. Thus we can assert
\[ U\text{-alg} \cong \text{ConFunc}(\mathbb{C}(U), \underline{Fam}). \]

It has turned out that the inductive construction of $\mathbb{C}(U)$ from $U$ has enabled us to replace the usual inductive definition of model of $U$ by the definition "a model of $U$ is a contextual functor $M: \mathbb{C}(U) \to \underline{Fam}$".

Every interpretation $I: U \to U'$ induces a contextual functor $\mathbb{C}(I): \mathbb{C}(U) \to \mathbb{C}(U')$. Composition with $\mathbb{C}(I)$ is a functor from $\text{ConFunc}(\mathbb{C}(U'), \underline{Fam})$ to $\text{ConFunc}(\mathbb{C}(U), \underline{Fam})$. It is the functor $I\text{-alg}: U'\text{-alg} \to U\text{-alg}$. Those functors between categories of models which are induced in this way are called generalised algebraic functors. We can show that all such functors have left adjoints. But anyhow this is equivalent to a known generalisation of Lawvere [11]'s theorem all algebraic functors have left adjoint.

\subsection{Universal Algebra}

We have been able to prove a generalisation of Birkhoff's theorem. Previously this theorem has been proved for many sorted algebraic theories see Birkhoff and Lipson [3]. Birkhoff's theorem classifies those subcategories of a category $U\text{-alg}$ that arise as the category of models of an equational extension of $U$. The result that we describe characterises subcategories $U'\text{-alg}$ of $U\text{-alg}$ in cases where $U$ is any generalised algebraic theory and $U'$ is any $\in$-equational extension of $U$.

By an $\in$-equational extension $U'$ of $U$ we mean an extension by axioms all of which are $\in=$-rules. Thus $U'$ is not permitted to have any T=axioms which are not already in $U$.

If we state the result then we can explain the terms afterwards.

\begin{theorem}
If $U$ is a generalised algebraic theory and if $\Gamma$ is a subcategory of $U\text{-alg}$ then equivalent are:

i. $\Gamma$ is the subcategory determined by some $\in$-equational extension $U'$ of $U$.

ii. $\Gamma$ is a full subcategory of $U$ closed under products, homomorphic images and subalgebras and having the property that if $M$ is a $U$-algebra and if all the finitely generated subalgebras of $M$ belong to $\Gamma$ then $M$ belongs to $\Gamma$.
\end{theorem}

The $U$-algebra $M'$ is said to be a \underline{homomorphic image} of the $U$-algebra $M$ iff there exists a homomorphism $f: M \to M'$ having the property that for all $1 \triangleleft A_1 \triangleleft \dots \triangleleft A_n$ in $\mathbb{C}(U)$, for all $a'_1 \in M'(A_1)$, for all $a'_2 \in M'(A_2)(a'_1), \dots,$ for all $a'_n \in M'(A_n)(a'_1, \dots, a'_{n-1})$, there exists $a_1, \dots, a_n$ such that $a_1 \in M(A_1), \dots, a_n \in M(A_n)(a_1, \dots, a_{n-1})$ and such that $f(a_1) = a'_1, \dots, f(a_n) = a'_n$.

A $U$-algebra $M$ is to be a \underline{finitely generated} $U$-algebra iff it is the homomorphic image of some finitely generated free algebra.

Consider for a moment. Every theory $U$ has a minimal model denoted $K_U$ and built out of closed terms. Alternatively this minimal model is described just in terms of the structure $\mathbb{C}(U)$. For example if $1 \triangleleft A$ in $\mathbb{C}(U)$ then $K_U(A) = Hom(1,A)$, otherwise if $1 \triangleleft A_1 \dots \triangleleft A_n \triangleleft A$ in $\mathbb{C}(U)$ then if $a_1 \in K_U(A_1), \dots,$ if $a_n \in K_U(A_n)(a_1, \dots, a_{n-1})$ then $K_U(A)(a_1, \dots, a_n) = \{ a \in Hom_{\mathbb{C}(U)}(1,A) \mid a \circ p(A) = a_n \}$.

Now, the free $U$-algebras are the algebras $I\text{-alg}(K_{U'})$ for $I: U \to U'$ an extension of $U$ by constants alone. The finitely generated free $U$-algebras are these algebras where $U'$ in an extension by just finitely many constants.

For example, take $U$ to be the theory of categories. Take $U'$ to be the theory of categories +

\textbf{Symbol} \quad \textbf{Introductory Rule}
\begin{itemize}
    \item $a_1$: \quad $a_1 \in Ob$
    \item $a_2$: \quad $a_2 \in Ob$
    \item $b$: \quad $b \in Hom(a_1, a_2)$.
\end{itemize}

In this case $I\text{-alg}(K_{U'})$ is the category $\cdot \xrightarrow{} \cdot$. It is the free category with one morphism. It is a finitely generated free category.

\chapter{THE ALGEBRAIC SEMANTICS OF M-L TYPE THEORY}

We say what it is for a contextual category to have disjoint unions and a singleton object (comparable with a category with finite products and terminal object). The category of these structures is denoted \underline{$\Sigma$-Con}.

We then introduce a new notion of structure - the notion of a category with attributes. The category of these structures is denoted \underline{Attcat}. We show that the category \underline{$\Sigma$-Con} and the category \underline{Attcat} are equivalent categories.

In \S 1.5 we alluded to a contextual category of categories, category indexed families of categories and so on. The well known fibration construction induces a structure of disjoint unions on this contextual category. We use the \underline{Attcat}, \underline{$\Sigma$-Con} equivalence to give a fairly brief description of this structure. This is in \S 3.3.

\S 3.4 contains the definition of M-L structure intended as the model theory of a strengthened M-L type theory.

Also in \S 3.4 an equivalent notion of structure, based on the \underline{Attcat}, \underline{$\Sigma$-Con} equivalence, is put forward. Then in \S 3.5 we can briefly describe a new model of M-L type theory, we refer to it as the limit space model.

\section{Disjoint Unions and Singleton Object}

A \underline{contextual category with disjoint unions} $\langle \mathbb{C}, \Sigma, \alpha \rangle$ consists of a contextual category $\mathbb{C}$ and for every $Q \triangleleft A \triangleleft B$ in $\mathbb{C}$, an object $\Sigma B$ of $\mathbb{C}$ and a morphism $\alpha(B)$ of $\mathbb{C}$ such that $Q \triangleleft \Sigma B$ in $\mathbb{C}$, such that $\alpha(B)$ is an isomorphism in $\mathbb{C}$, $\alpha(B): B \to \Sigma B$ such that the diagram
\begin{center}
\begin{tikzcd}
& B \arrow[d, "\alpha(B)"] \arrow[dl, "p(B)"'] \\
A \arrow[r, "p(A)"'] & \Sigma B \arrow[d, "p(\Sigma B)"] \\
& Q
\end{tikzcd}
\end{center}
commutes,

subject to the condition: If $f: Q \to Q'$ in $\mathbb{C}$ and if $Q' \triangleleft A \triangleleft B$ then $f^* \Sigma B = \Sigma f^* B$ and $f^*(\alpha(B)) = \alpha(f^*B)$.

The category of contextual categories with disjoint unions has as morphisms $F: \langle \mathbb{C}, \Sigma, \alpha \rangle \to \langle \mathbb{C}', \Sigma', \alpha' \rangle$ those contextual functors $F: \mathbb{C} \to \mathbb{C}'$ such that for all $Q \triangleleft A \triangleleft B$ in $\mathbb{C}$, $F(\Sigma B) = \Sigma' F(B)$ and $F(\alpha(B)) = \alpha'(F(B))$.

It follows that if $\langle \mathbb{C}, \Sigma, \alpha \rangle$ is a contextual category with disjoint unions and if $f: A \to A'$ in $\mathbb{C}$ then $\mathbb{C}_f: \langle \mathbb{C}_{A'}, \Sigma, \alpha \rangle \to \langle \mathbb{C}_A, \Sigma, \alpha \rangle$ is a morphism of contextual categories with disjoint unions.

Actual disjoint unions of families of sets induce the structure of contextual category with disjoint unions on the contextual category \underline{Fam} of sets, families of sets and so on. This is as follows.

If $1 \triangleleft Q_1 \dots \triangleleft Q_n \triangleleft A \triangleleft B$ in \underline{Fam} then define $\Sigma B$, such that $Q_n \triangleleft \Sigma B$ in \underline{Fam}, by defining $\Sigma B(r_1, \dots, r_n) = \{ \langle a,b \rangle \mid a \in A(r_1, \dots, r_n) \text{ and } b \in B(r_1, \dots, r_n, a) \}$ whenever $r_1 \in Q_1, \dots, r_n \in Q_n(r_1, \dots, r_{n-1})$. Define $\alpha(B) : B \to \Sigma B$ in \underline{Fam} by defining $\alpha(B) = \langle f_1, \dots, f_n, g \rangle$ where $f_i, 1 \le i \le n$ is defined by $f_i(r_1, \dots, r_n, a, b) = r_i$ whenever $r_1 \in Q_1, \dots, r_n \in Q_n(r_1, \dots, r_{n-1}), a \in A(r_1, \dots, r_n)$ and $b \in B(r_1, \dots, r_n, a)$, and where $g$ is defined by $g(r_1, \dots, r_n, a, b) = \langle a, b \rangle$.

It is easy to see that $\alpha(B)$ has inverse $\alpha^{-1}(B)$ given by $\langle L_1, \dots, L_n, k_1, k_2 \rangle$ where $L_i, 1 \le i \le n$, is the operator given by $L_i(r_1, \dots, r_n, c) = r_i$ whenever $r_1 \in Q_1, \dots, r_n \in Q_n(r_1, \dots, r_{n-1})$ and $c \in \Sigma B(r_1, \dots, r_n)$, and where $k_1$ and $k_2$ are given by $k_i(r_1, \dots, r_n, c) =$ the $i$th component of the ordered pair $c$. The conditions $f^* \Sigma B = \Sigma f^* B$ and $f^*(\alpha(B)) = \alpha(f^*B)$ are automatically satisfied. Even without these conditions it is clear that the definition characterises disjoint unions in \underline{Fam} up to isomorphism.

\begin{lemma}
If $\langle \mathbb{C}, \Sigma, \alpha \rangle$ is a contextual category with disjoint unions then Base($\mathbb{C}$) is a category with products of pairs (and hence has finite products).
\end{lemma}

\begin{proof}
Suppose that $A$ and $A'$ are objects of Base($\mathbb{C}$). That is to say suppose $1 \triangleleft A$ and $1 \triangleleft A'$ in $\mathbb{C}$.

In any category, pulling back over the terminal object yields a product diagram. Since
\begin{center}
\begin{tikzcd}
p(A)^*A' \arrow[r, "q(p(A){,} A')"] \arrow[d, "p(A)^*A'"] & A' \arrow[d, "p(A')"] \\
A \arrow[r, "p(A)"] & 1
\end{tikzcd}
\end{center}
is a pullback diagram in $\mathbb{C}$,
\begin{center}
\begin{tikzcd}
& p(A)^*A' \arrow[dl, "p(p(A)^*A')"'] \arrow[dr, "q(p(A){,} A')"] & \\
A & & A'
\end{tikzcd}
\end{center}
is a product diagram in $\mathbb{C}$.

Since $\alpha(p(A)^*A') : p(A)^*A' \to \Sigma p(A)^*A'$ is an isomorphism
\begin{center}
\begin{tikzcd}
& \Sigma p(A)^*A' \arrow[dl, "p(p(A)^*A') \circ \alpha^{-1}(p(A)^*A')"'] \arrow[dr, "q(p(A){,} A') \circ \alpha^{-1}(p(A)^*A')"] & \\
A & & A'
\end{tikzcd}
\end{center}
is a product diagram in $\mathbb{C}$. $1 \triangleleft \Sigma p(A)^*A'$ in $\mathbb{C}$ and so this diagram is a product diagram in Base($\mathbb{C}$).
\end{proof}

If this proof is interpreted in $\mathbb{C} = \underline{Fam}$ then we have Base($\mathbb{C}$) = \underline{Set}, the category of sets and functions. If $A$ and $A'$ are sets then $p(A)^*A'$ is the constant $A$-indexed family with value $A'$. Thus $A \times A'$ in \underline{Set} is given by $\Sigma p(A)^*A'$ in \underline{Fam} which in turn is just $\{ \langle a, a' \rangle \mid a \in A \text{ and } a' \in A' \}$. All is as it should be.

A \underline{singleton object} of the contextual category $\mathbb{C}$ is defined to be an object of Base($\mathbb{C}$) that is terminal in $\mathbb{C}$. Equivalently it is an object of Base($\mathbb{C}$) that is isomorphic in $\mathbb{C}$ to the terminal object $1$. The singleton object, if there is one, will usually be denoted $\{ \cdot \}$. The unique morphism $1 \to \{ \cdot \}$ in $\mathbb{C}$ will be denoted $e$. Thus a contextual category with singleton object $\langle \mathbb{C}, \{ \cdot \}, e \rangle$ consists of a contextual category $\mathbb{C}$, an object $\{ \cdot \}$ of $\mathbb{C}$ such that $1 \triangleleft \{ \cdot \}$ in $\mathbb{C}$ and a morphism $e : 1 \to \{ \cdot \}$ such that $p(\{ \cdot \}) \circ e = id_{\{ \cdot \}}$. The morphisms between contextual categories with singleton objects are taken to be those contextual functors which preserve the singleton object.

It is not difficult to see that if $U$ is a generalised algebraic theory then the contextual category $\mathbb{C}(U)$ has a singleton object iff there exists expressions $t$ and $\Delta$ of $U$ such that $x \in \Delta : t = x \in \Delta$ is a derived rule of $U$.

\begin{lemma}
If $\mathbb{C}$ is a contextual category with singleton object $\{ \cdot \}$ and if $F: \mathbb{C} \to \mathbb{D}$ is a contextual functor then $F(\{ \cdot \})$ is a singleton object of $\mathbb{D}$.
\end{lemma}

\begin{proof}
$F$ is a functor, hence $F$ preserves isomorphisms. Thus $1 \triangleleft F(\{ \cdot \})$ and $F(\{ \cdot \}) \cong 1$ in $\mathbb{D}$.
\end{proof}

\begin{lemma}
If $\langle \mathbb{C}, \{ \cdot \} \rangle$ is a contextual category with singleton object and if $A$ is an object of $\mathbb{C}$ then $\langle \mathbb{C}_A, p(A,1)^* \{ \cdot \} \rangle$ is a contextual category with singleton object. If $f: A \to A'$ in $\mathbb{C}$ then $\mathbb{C}_f : \langle \mathbb{C}_{A'}, p(A',1)^* \{ \cdot \} \rangle \to \langle \mathbb{C}_A, p(A,1)^* \{ \cdot \} \rangle$ is a morphism of contextual categories with singleton objects.
\end{lemma}

\begin{proof}
If $A$ is an object of $\mathbb{C}$ then $p(A,1) : A \to 1$ in $\mathbb{C}$. Hence $p(A,1)^* : \mathbb{C}_1 \to \mathbb{C}_A$. Thus by lemma 2 $\langle \mathbb{C}_A, p(A,1)^* \{ \cdot \} \rangle$ is a contextual category with singleton object. Now if $f: A \to A'$ in $\mathbb{C}$ then $\mathbb{C}_f : \mathbb{C}_{A'} \to \mathbb{C}_A$ and $\mathbb{C}_f(p(A',1)^* \{ \cdot \}) = f^* p(A',1)^* \{ \cdot \} = (p(A',1) \circ f)^* \{ \cdot \} = p(A,1)^* \{ \cdot \}$. Thus $\mathbb{C}_f : \mathbb{C}_{A'} \to \mathbb{C}_A$ and preserves the singleton object.
\end{proof}

The category of contextual categories with disjoint unions and singleton objects is denoted \underline{$\Sigma$-Con}.

The point is that in a contextual category with disjoint unions and singleton object there is a lot of repetition of structure. For example if in such a structure $Q \triangleleft A \triangleleft B$ then $\mathbb{C}_{\alpha(B)}$ is an isomorphism of $\mathbb{C}_B$ with $\mathbb{C}_{\Sigma B}$. Thus the structure of $\mathbb{C}$ above $B$ is isomorphic to the structure of $\mathbb{C}$ above $\Sigma B$ and because $\Sigma B$ is at a lower level than $B$ it turns out that the whole structure of $\mathbb{C}$ is coded up as structure at a very low level. This leads to the notion of a category with attributes. We must first introduce some new notation.

For the remainder of this section we suppose $\langle \mathbb{C}, \Sigma, \alpha, \{ \cdot \} \rangle$ to be a contextual category with disjoint unions and singleton object.

\begin{lemma}
If $f: A \to A'$ in Base($\mathbb{C}$) and if $A' \triangleleft B \triangleleft D$ in $\mathbb{C}$ then the diagram
\begin{center}
\begin{tikzcd}
f^*D \arrow[r, "q(f{,}D)"] \arrow[d, "\alpha(f^*D)"'] & D \arrow[d, "\alpha(D)"] \\
f^* \Sigma D \arrow[r, "q(f{,} \Sigma D)"] & \Sigma D
\end{tikzcd}
\end{center}
commutes.
\end{lemma}

\begin{proof}
Since $f: A \to A'$ and $A' \triangleleft B \triangleleft D$ we have by definition that $f^* \Sigma D = \Sigma f^* D$ and $f^*(\alpha(D)) = \alpha(f^*D)$. But $f^*(\alpha(D))$ is defined to be the unique map: $f^*D \to f^* \Sigma D$ such that the diagrams
\begin{center}
\begin{minipage}{0.4\textwidth}
\begin{tikzcd}
f^*D \arrow[r, "q(f{,}D)"] \arrow[d] & D \arrow[d, "p(D)"] \\
f^*B \arrow[r, "q(f{,}B)"] & B
\end{tikzcd}
\end{minipage}
and
\begin{minipage}{0.4\textwidth}
\begin{tikzcd}
f^*D \arrow[r, "q(f{,}D)"] \arrow[d, "f^*(\alpha(D))"'] & D \arrow[d, "\alpha(D)"] \\
f^* \Sigma D \arrow[r, "q(f{,} \Sigma D)"] & \Sigma D
\end{tikzcd}
\end{minipage}
\end{center}
commute. Thus the statement of the lemma is a restatement of the condition $f^*(\alpha(D)) = \alpha(f^*D)$.
\end{proof}

If $1 \triangleleft A \triangleleft B$ in \underline{Fam} then we let $\rho(B) : \Sigma B \to A$ be the 1st projection function from $\{ \langle a,b \rangle \mid a \in A, b \in B(a) \}$ to $A$. More generally if $Q \triangleleft A \triangleleft B$ in $\mathbb{C}$ then define $\rho(B) : \Sigma B \to A$ to be $\alpha^{-1}(B) \circ p(B)$.
\begin{center}
\begin{tikzcd}
& B \arrow[dl, "\alpha(B)"'] \arrow[d, "p(B)"] \\
\Sigma B \arrow[r, "\rho(B)"] & A
\end{tikzcd}
\end{center}

Now suppose $f: A \to A'$ in Base($\mathbb{C}$) and $A' \triangleleft B$ in $\mathbb{C}$, define $\xi(f,B) : \Sigma f^*B \to \Sigma B$ to be the morphism $\alpha^{-1}(f^*B) \circ q(f,B) \circ \alpha(B)$.

\begin{lemma}
If $f: A \to A'$ in Base($\mathbb{C}$) and if $A' \triangleleft B$ in $\mathbb{C}$ then
\begin{center}
\begin{tikzcd}
\Sigma f^*B \arrow[r, "\xi(f{,}B)"] \arrow[d, "\rho(f^*B)"] & \Sigma B \arrow[d, "\rho(B)"] \\
A \arrow[r, "f"] & A'
\end{tikzcd}
\end{center}
is a pullback diagram in Base($\mathbb{C}$).
\end{lemma}

\begin{proof}
Just because
\begin{center}
\begin{tikzcd}
f^*B \arrow[r, "q(f{,}B)"] \arrow[d, "p(f^*B)"] & B \arrow[d, "p(B)"] \\
A \arrow[r, "f"] & A'
\end{tikzcd}
\end{center}
is a pullback diagram in $\mathbb{C}$ and $\alpha(f^*B)$ and $\alpha(B)$ are isomorphisms.
\end{proof}

Note that these pullback diagrams fit together in the sense that if $A \xrightarrow{f} A' \xrightarrow{f'} A''$ in $\mathbb{C}$ then $f^*f'^*B = (ff')^*B$ and $\xi(f, f'^*B) \circ \xi(f', B) = \xi(ff', B)$. We are at this moment collapsing the structure of $\mathbb{C}$ into Base($\mathbb{C}$).

We now introduce the $\#, \gamma$ notation. We have to think of this as a reflection at a low level of $\mathbb{C}$, structure at higher levels.

\begin{lemma}
If $f: A \to A'$ is an isomorphism in $\mathbb{C}$ and if $A' \triangleleft B$ then $q(f,B) : f^*B \to B$ is an isomorphism with inverse $q(f^{-1}, f^*B) : B \to f^*B$ in $\mathbb{C}$.
\end{lemma}

\begin{proof}
$q(f^{-1}, f^*B): B \to f^*B$ in $\mathbb{C}$ because $f^{-1*} f^*B = B$.
$q(f^{-1}, f^*B) \circ q(f,B) = q(f^{-1} \circ f, B) = q(id_{A'}, B) = id_B$.
$q(f,B) \circ q(f^{-1}, f^*B) = q(f \circ f^{-1}, f^*B) = q(ff^{-1}, f^*B) = q(id_A, f^*B) = id_{f^*B}$.
\end{proof}

In future if $1 \triangleleft A \triangleleft B$ in $\mathbb{C}$ and $\Sigma B \triangleleft C$ then the object $\Sigma \alpha(B)^*C$ will be denoted $\# C$. Thus $A \triangleleft \# C$ in $\mathbb{C}$. By lemma 6 $\alpha(B)^*C \cong C$. The composite isomorphism $C \xrightarrow{\alpha(B)} \alpha(B)^*C \xrightarrow{\alpha(\alpha(B)^*C)} \Sigma \alpha(B)^*C$ will be denoted $\sigma(C)$. Thus $\sigma(C) = q(\alpha(B)^{-1}, \alpha(B)^*C) \circ \alpha(\alpha(B)^*C)$.
Finally we define $\gamma(C) : \Sigma C \to \Sigma \# C$ to be the isomorphism $\alpha^{-1}(C) \circ \sigma(C) \circ \alpha(\# C)$.

\begin{center}
\begin{tikzcd}
C \arrow[r, "{q(\alpha(B)^{-1}{,}\alpha(B)^*C)}"] \arrow[d, "\alpha^{-1}(C)"] \arrow[dr, "\sigma(C)"] & \alpha(B)^*C \arrow[r, "\alpha(\alpha(B)^*C)"] & \Sigma \alpha(B)^*C = \# C \arrow[d, "\alpha(\# C)"] \\
\Sigma C \arrow[rr, "\gamma(C)"] & & \Sigma \# C
\end{tikzcd}
\end{center}

\begin{lemma}
If $1 \triangleleft A \triangleleft B$ and $\Sigma B \triangleleft C$ in $\mathbb{C}$ then the diagram
\begin{center}
\begin{tikzcd}
\Sigma C \arrow[r, "\gamma(C)"] \arrow[d, "\rho(C)"] & \Sigma \# C \arrow[d, "\rho(\# C)"] \\
\Sigma B \arrow[d, "\rho(B)"] & \Sigma \alpha(B)^*C \\
A
\end{tikzcd}
\end{center}
commutes in Base($\mathbb{C}$).
\end{lemma}

\begin{proof}
Use the definition of $\rho$ and $\gamma$ plus the commutativity of the diagrams
\begin{center}
\begin{minipage}{0.45\textwidth}
\begin{tikzcd}
C \arrow[r, "{q(\alpha^{-1}(B){,} \alpha(B)^*C)}"] \arrow[d, "p(C)"] & \alpha(B)^*C \arrow[d] \\
\Sigma B \arrow[r, "\alpha^{-1}(B)"] & B
\end{tikzcd}
\end{minipage}
and
\begin{minipage}{0.45\textwidth}
\begin{tikzcd}
\alpha(B)^*C \arrow[r, "\alpha(\alpha(B)^*C)"] \arrow[d] & \Sigma \alpha(B)^*C = \# C \\
B \arrow[d, "p(B)"] & \\
A
\end{tikzcd}
\end{minipage}
\end{center}
(The first diagram commutes because it is a pullback diagram, the second by definition of disjoint unions).
\end{proof}

\begin{lemma}
If $f: A \to A'$ in Base($\mathbb{C}$) and if $A' \triangleleft B$, $\Sigma B \triangleleft C$ in $\mathbb{C}$ then $\#(\xi(f,B)^*C) = f^* \# C$ and the diagram
\begin{center}
\begin{tikzcd}
\Sigma \xi(f{,}B)^*C \arrow[r, "\xi(\xi(f{,}B){,} C)"] \arrow[d, "\gamma(\xi(f{,}B)^*C)"] & \Sigma C \arrow[d, "\gamma(C)"] \\
\Sigma f^* \# C \arrow[r, "\xi(f{,} \# C)"] & \Sigma \# C
\end{tikzcd}
\end{center}
commutes.
\end{lemma}

\begin{proof}
The situation:
\begin{center}
\begin{tikzcd}
\xi(f{,}B)^*C \arrow[d, "\alpha(f^*B)"] & f^*B \arrow[d, "\alpha(f^*B)"] \arrow[r, "q(f{,}B)"] & B \arrow[d, "\alpha(B)"] & C \arrow[d, "\alpha(B)"] \\
\Sigma f^* B & A \arrow[r, "f"] & A' & \Sigma B
\end{tikzcd}
\end{center}

The identity $\#(\xi(f,B)^*C) = f^*B$ holds as follows:
$\#(\xi(f,B)^*C) = \Sigma(\alpha(f^*B)^* \xi(f,B)^* C)$ by def. of $\#$.
$= \Sigma(\alpha(f^*B)^* (\alpha^{-1}(f^*B) \circ q(f,B) \circ \alpha(B))^* C)$ by def. of $\xi$.
$= \Sigma(q(f,B)^* \alpha(B)^* C)$
$= \Sigma(f^* \alpha(B)^* C)$ * notation - see lemma 1 of 2.3. Since $A' \triangleleft B \triangleleft \alpha(B)^*$ and $f: A \to A'$ in $\mathbb{C}$.
$= f^* \Sigma(\alpha(B)^* C)$
$= f^* \# C$ by def. of $\#$.

Now if we take the diagram that we wish to show commutes and use the definitions of $\gamma$ and $\xi$ we see that we wish to show that the outer rectangle of the diagram
\begin{center}
\begin{tikzcd}
\Sigma \xi(f{,}B)^*C \arrow[r, "\alpha^{-1}(\xi(f{,}B)^*C)"] \arrow[d, "\alpha^{-1}(\xi(f{,}B)^*C)"] & \xi(f{,}B)^*C \arrow[r, "q(\xi(f{,}B){,} C)"] & C \arrow[r, "\alpha(C)"] \arrow[d, "\sigma(C)"] & \Sigma C \arrow[d, "\alpha^{-1}"] \\
& & \# C \arrow[d, "\alpha"] \\
& \# \xi(f{,}B)^*C \arrow[r] & f^* \# C \arrow[r, "q(f{,} \# C)"] & \Sigma \# C
\end{tikzcd}
\end{center}
commutes.

Thus it suffices to show that the diagram
\begin{center}
\begin{tikzcd}
\xi(f{,}B)^*C \arrow[r, "q(\xi(f{,}B){,} C)"] \arrow[d, "\sigma(\xi(f{,}B)^*C)"] & C \arrow[d, "\sigma(C)"] \\
f^* \# C = \#(\xi(f{,}B)^*C) \arrow[r, "q(f{,} \# C)"] & \# C
\end{tikzcd}
\end{center}
commutes.

Now $\sigma(C)$ is defined to be $q(\alpha^{-1}(B), \alpha(B)^*C) \circ \alpha(\alpha(B)^*C)$
and $\sigma(\xi(f,B)^*C) = q(\alpha^{-1}(f^*B), \alpha(f^*B)^* \xi(f,B)^*C) \circ \alpha(\alpha(f^*B)^* \xi(f,B)^*C) = q(\alpha^{-1}(f^*B), f^* \alpha(B)^* C) \circ \alpha(f^* \alpha(B)^* C)$, by definition of $\xi$. Thus we wish to show that the outer rectangle of the diagram
\begin{center}
\begin{tikzcd}
\xi(f{,}B)^*C \arrow[r, "q(\xi(f{,}B){,} C)"] \arrow[d] & C \arrow[d] \\
f^* \alpha(B)^* C \arrow[r, "q(f{,} \alpha(B)^*C)"] \arrow[d] & \alpha(B)^* C \arrow[d] \\
f^* \# C \arrow[r, "q(f{,} \# C)"] & \# C
\end{tikzcd}
\end{center}
commutes.

Well the lower rectangle commutes by lemma 4 of this section. To show that the upper rectangle commutes we replace the extended $^*, q$ notation (i.e. use lemma 1 of \S 2.3), use the fact that pullbacks fit together and use the definition of $\xi$. This is as follows
$q(\alpha^{-1}(f^*B), f^* \alpha(B)^* C) \circ q(f, \alpha(B)^* C) =$
$q(\alpha^{-1}(f^*B), q(f, B)^* \alpha(B)^* C) \circ q(q(f,B), \alpha(B)^* C) =$
$q(\alpha^{-1}(f^*B) \circ q(f,B), \alpha(B)^* C) = q(\xi(f,B) \circ \alpha^{-1}(B), \alpha(B)^* C) =$
$q(\xi(f,B), \alpha^{-1}(B)^* \alpha(B)^* C) \circ q(\alpha^{-1}(B), \alpha(B)^* C) =$
$q(\xi(f,B), C) \circ q(\alpha^{-1}(B), \alpha(B)^* C)$, as required.
\end{proof}

If $A \in |\text{Base}(\mathbb{C})|$ then $p(\{ \cdot \})^* A \cong A$ because $p(\{ \cdot \}) : \{ \cdot \} \to 1$ is an isomorphism. Hence $\Sigma p(\{ \cdot \})^* A \cong A$. We denote $p(\{ \cdot \})^* A$ by $L(A)$ and the isomorphism of $\Sigma L(A)$ with $A$ by $\Theta(A)$. Thus $\Theta(A) : \Sigma L(A) \to A$ is defined by $\Theta(A) = \alpha^{-1}(p(\{ \cdot \})^* A) \circ q(p(\{ \cdot \}), A)$.

\begin{lemma}
If $1 \triangleleft A \triangleleft B$ in $\mathbb{C}$ then $L(\Sigma B) = \#(\Theta(A)^* B)$ and the diagram
\begin{center}
\begin{tikzcd}
\Sigma B \arrow[d, "\Theta(\Sigma B)"] & \Sigma L(\Sigma B) \arrow[l, "\xi(\Theta(A){,} B)"'] \arrow[d, "\gamma(\Theta(A)^* B)"] \\
\Sigma L \Sigma B & \Sigma \# \Theta(A)^* B \arrow[l, "\Theta(\Sigma B)"]
\end{tikzcd}
\end{center}
commutes.
\end{lemma}

\begin{proof}
$\#((\alpha^{-1}(p(\{ \cdot \})^* A) \circ q(p(\{ \cdot \}), A))^* B = \Sigma q(p(\{ \cdot \}), A)^* B$ by definition of $\#$, $= \Sigma p(\{ \cdot \})^* B$, by replacing extended $^*, q$ notation, $= p(\{ \cdot \})^* \Sigma B$. That is $\# \Theta(A)^* B = L(\Sigma B)$.

As for the commuting triangle, if we cancel out the $\rho$'s and $\alpha^{-1}$'s, after substituting in for $\Theta, \gamma$ and $\xi$, then we see that we want to show that the diagram
\begin{center}
\begin{tikzcd}
q(p(\{ \cdot \}), \Sigma B) \arrow[d] & \Sigma B \arrow[l] \arrow[d] \\
p(\{ \cdot \})^* \Sigma B \arrow[r, equal] & \# \Theta(A)^* B
\end{tikzcd}
\end{center}
commutes. Now $q(\Theta(A), B) \circ \alpha(B) = q(\alpha^{-1}(p(\{ \cdot \})^* A) \circ q(p(\{ \cdot \}), A), B) \circ \alpha(B)$
$q(\alpha^{-1}(p(\{ \cdot \})^* A), p(\{ \cdot \})^* B) \circ q(p(\{ \cdot \}), B) \circ \alpha(B)$. Whereas
$\sigma(\Theta(A)^* B) = q(\alpha^{-1}(p(\{ \cdot \})^* A), p(\{ \cdot \})^* B) \circ \alpha(p(\{ \cdot \})^* B)$. Thus we must just show that the diagram
\begin{center}
\begin{tikzcd}
p(\{ \cdot \})^* B \arrow[r, "q(p(\{ \cdot \}){,} B)"] \arrow[d, "\alpha(p(\{ \cdot \})^* B)"] & B \arrow[d, "\alpha(B)"] \\
\Sigma p(\{ \cdot \})^* B \arrow[r, "q(p(\{ \cdot \}){,} \Sigma B)"] & \Sigma B
\end{tikzcd}
\end{center}
commutes. But this commutes by lemma 4.
\end{proof}
\end{document}
